% LaTeX Template for Project Report, Version 2.0
% (Abstracted from a Major Project Report at CSED, NIT Calicut but can be
% modified easily to use for other reports also.)
%
% Released under Creative Commons Attribution license (CC-BY)
% Info: http://creativecommons.org/licenses/by/3.0/
%
% Created by: Kartik Singhal
% BTech CSE Batch of 2009-13
% NIT Calicut
% Contact Info: kartiksinghal@gmail.com
%
% It is advisable to learn the basics of LaTeX before using this template.
% A good resource to start with is http://en.wikibooks.org/wiki/LaTeX/
%
% All template fields are marked with a pair of angular brackets e.g. <title here>
% except for the ones defining citation names in ref.tex.
%
% Empty space after chapter/section/subsection titles can be used to insert text.
%
% Just compile this file using pdflatex after making all required changes.

\documentclass[12pt,a4paper]{article}

\RequirePackage{fancyhdr}
\RequirePackage{lastpage}

\setlength\textwidth{165mm}
\setlength\textheight{240mm}
\setlength\topmargin{-10mm}
\setlength\oddsidemargin{0mm}
\setlength\parindent{0pt}
\setlength\parskip{1.7\medskipamount}

\sloppy\pagestyle{fancy}


%% rest

\usepackage{url} %for proper url entries
\usepackage[numbers]{natbib}
% for nice tables
\usepackage{longtable}
\usepackage{booktabs}
\usepackage{floatrow}
\floatsetup[table]{capposition=bottom}
% for nice code

\usepackage{amssymb,amsmath}
\usepackage{ifxetex,ifluatex}
\ifxetex
  \usepackage{fontspec,xltxtra,xunicode}
  \defaultfontfeatures{Mapping=tex-text,Scale=MatchLowercase}
\else
  \ifluatex
    \usepackage{fontspec}
    \defaultfontfeatures{Mapping=tex-text,Scale=MatchLowercase}
  \else
    \usepackage[utf8]{inputenc}
  \fi
\fi

\usepackage{color}
\usepackage{fancyvrb}
\DefineShortVerb[commandchars=\\\{\}]{\|}
\DefineVerbatimEnvironment{Highlighting}{Verbatim}{commandchars=\\\{\}}
% Add ',fontsize=\small' for more characters per line
\newenvironment{Shaded}{}{}
\newcommand{\KeywordTok}[1]{\textcolor[rgb]{0.00,0.44,0.13}{\textbf{{#1}}}}
\newcommand{\DataTypeTok}[1]{\textcolor[rgb]{0.56,0.13,0.00}{{#1}}}
\newcommand{\DecValTok}[1]{\textcolor[rgb]{0.25,0.63,0.44}{{#1}}}
\newcommand{\BaseNTok}[1]{\textcolor[rgb]{0.25,0.63,0.44}{{#1}}}
\newcommand{\FloatTok}[1]{\textcolor[rgb]{0.25,0.63,0.44}{{#1}}}
\newcommand{\CharTok}[1]{\textcolor[rgb]{0.25,0.44,0.63}{{#1}}}
\newcommand{\StringTok}[1]{\textcolor[rgb]{0.25,0.44,0.63}{{#1}}}
\newcommand{\CommentTok}[1]{\textcolor[rgb]{0.38,0.63,0.69}{\textit{{#1}}}}
\newcommand{\OtherTok}[1]{\textcolor[rgb]{0.00,0.44,0.13}{{#1}}}
\newcommand{\AlertTok}[1]{\textcolor[rgb]{1.00,0.00,0.00}{\textbf{{#1}}}}
\newcommand{\FunctionTok}[1]{\textcolor[rgb]{0.02,0.16,0.49}{{#1}}}
\newcommand{\RegionMarkerTok}[1]{{#1}}
\newcommand{\ErrorTok}[1]{\textcolor[rgb]{1.00,0.00,0.00}{\textbf{{#1}}}}
\newcommand{\NormalTok}[1]{{#1}}
\usepackage[pdftex]{graphicx}
\setkeys{Gin}{width=\textwidth}
\usepackage[Export]{adjustbox}
\usepackage[unicode=true]{hyperref}
\hypersetup{breaklinks=true, pdfborder={0 0 0}}
\setlength{\parindent}{0pt}
\setlength{\parskip}{6pt plus 2pt minus 1pt}
\setlength{\emergencystretch}{3em}  % prevent overfull lines
%\setcounter{secnumdepth}{0}


\begin{document}
\renewcommand\refname{References} %Renames "Bibliography" to "References" on ref page

%include other pages
\begin{titlepage}

\begin{center}

\textup{\small {\bf Statistical Computing in R} \\ Report}\\[0.2in]

% Title
\Large \textbf {Implementation of NMMSO in R}\\[0.5in]

 % Submitted by
\normalsize Submitted by \\
\begin{table}[h]
\centering
\begin{tabular}{lr}
425699 & Jannik Hoffjann \\
425699 & Daniel Carriola \\ 
\end{tabular}
\end{table}

\vspace{.1in}
Under the guidance of\\
{\textbf{Dr. Mike Preuss}}\\[0.2in]

\vfill

% Bottom of the page
\includegraphics[width=0.18\textwidth]{./assets/wwu-logo}\\[0.1in]
\Large{Information Systems and Statistics}\\
\normalsize
\textsc{ERCIS}\\
Münster - NRW - Germany \\
\vspace{0.2cm}
Winter Semester 2015/16

\end{center}

\end{titlepage}

\newpage
\pagenumbering{roman} %numbering before main content starts
\tableofcontents
\newpage

\pagenumbering{arabic} %reset numbering to normal for the main content

\section{Introduction}\label{introduction}

In the recent years, R has become the statistical programming language
of choice for many scientists. The strength of R, being a domain
specific language, has also become one of its weaknesses. Since new
research findings in statistical computing are split up over several
languages like R, Matlab or SciPy\footnote{SciPy is a common library for
  the Python Programming language which brings Statistical Computing
  capabilities to the language. \newpage}, it often becomes difficult to
compare new methods with established ones. Because it is also hard to
interface those languages due to different architectures and data
storage mechanisms there is often no other way than to reimplement new
methods in a different programming language to create a common scope.

An example for a well perceived new finding in statistical computing is
the NMMSO-Algorithm by Jonathan E. Fieldsend (Fieldsend 2014). It won
the niching competition in 2015 held by the CEC and is only written in
Matlab. Since the chair `Information Systems and Statistics' at the
Westfälische Wilhelms-Universität Münster, Germany, is mainly
concentrating its work on Statistical Computing in R, an implementation
of this algorithm became interesting.

As part of this Seminar Project in the context of the Seminar
`Statistical Computing in R', a reimplementation of the NMMSO algorithm
in R (nmmso.R) will be presented. During this technical documentation,
the general function of the algorithm and the used test cases by the CEC
will be shown. Afterwards the structure and used techniques and
libraries, as well as problems and pitfalls due to the different
behaviors of R and Matlab, will be shown. The documentation will be
closed by the benchmarking results and several test cases.

It was the goal of this project to keep up high comparability with the
original code, to ensure the correct functionality and easily implement
changes to the original codebase in this program. To reach this, unit
tests were used where possible and continuous comparison between interim
results of the original implementation and nmmso.R where used to ensure
functioning. Additionally a benchmarking suite, which builds on the CEC
Benchmarking Suite for Niching Algorithm was implemented to evaluate and
test the functioning of nmmso.R with the same characteristics as in the
original implementation.

\section{General Function}\label{general-function}

The starting point of the project was the paper provided by Dr.~Jonathan
E. Fieldsend (Fieldsend 2014) on the Niching Migratory Multi-Swarm
Optimiser (NMMSO) algorithm. NMMSO is a multi-modal optimiser which
relies heavily on multiple swarms that are generated on the landscape of
a function in order to find the global optima. It is build around three
main pillars: (1) dynamic in the numbers of dimensions, (2)
self-adaptive without any special preparation and (3) exploitative local
search to quickly find peak estimates (Fieldsend 2014, p. 1).

Multi-modal optimisation in general, does not differ very much from well
known and widely discussed single-objective optimisation, but but unlike
it, the goal of the algorithms in the multi-modal perspective is not to
find just one single optimising point but all possible points (Fieldsend
2014, p. 1). In order to do so, many early multi-modal optimisation
algorithms needed defined parameters (Fieldsend 2014, p. 1).

\textbf{Maybe write a bit more about multi-modal algorithms in general}

Newer algorithms fall in the field of self-tuning and try to use
different mathematical paradigms like nearest-best clustering with
covariance matrices (Preuss 2012) and strategies like storing the so far
best found global optima estimators to provide them as parameters for
new optimisation runs (Epitropakis et al. 2013). Contradictory to that
NMMSO goes the way of many early algorithms and uses the swarm strategy
in order to find which store their current (Fieldsend 2014).

In order to do so, NMMSO follow a strict structure, which can be seen in
the following pseudo-code

\begin{verbatim}
nmmso (max_evals, tol, n, max_inc, c_1, c_2, omega)
    S: initialise_swarm(1)
    evaluations := 1
    while evaluations < max_evals:
        while flagged_swarms(S) == true:
            {S, m} := attempt_merge(S, n, tol)
            evals := evals + m
        S := increment(S, n, max_inc, c_1, c_2, omega)
        evals := evals + min(|S|, max_inc)
        {S, k} := attempt_separation(S, tol)
        evals := evals + k
        S := add_new_swarm(S)
        evals := evals + 1
    {X*, Y*} := extract_gbest(S)
    return X*,Y*
\end{verbatim}

This structure wasn't modified during the reimplementation of NMMSO to
keep comparability and the possibility to fix bugs at a high level. The
only newly introduced setting was the possibility to modify c\_1, c\_2
and omega as parameters from the outside. In the original version those
parameters are part of the program code.

\begin{longtable}[c]{@{}lll@{}}
\toprule
& standard value & used value\tabularnewline
\midrule
\endhead
evaluations & 0 & 0\tabularnewline
max\_evol & 100 & 100\tabularnewline
tol\_val & 10\^{}-6 & 10\^{}-6\tabularnewline
c\_1 & 2.0 & 2.0\tabularnewline
c\_2 & 2.0 & 2.0\tabularnewline
omega & 0.1 & 0.1\tabularnewline
\bottomrule
\end{longtable}

\begin{center}\rule{0.5\linewidth}{\linethickness}\end{center}

\section{CEC Algorithms}\label{cec-algorithms}

\subsection{CEC}\label{cec}

The IEEE Congress of Evolutionary Computation (CEC) is one of the
largest, most important and recognized conferences within Evolutionary
Computation (EC). It is organised by the IEEE Computational Intelligence
Society in cooperation with the Evolutionary Programming Society and
covers most of the subtopics of the EC.

In order to validate the potential of the NMMSO algorithm, it was
submitted to the IEEE CEC 2015 held in Sendai, Japan. Here,
Dr.~Fieldsend was provided with some multimodal benchmark test functions
with different dimension sizes and characteristics, for evaluating
niching algorithms developed by Dr.~Xiaodong Li, Dr.~Andries Engelbrecht
and Dr.~Michael G. Epitropakis (Epitropakis et al. 2013). They state
that even if several niching methods have been around for many years,
further advances in this area have been hindered by several obstacles;
most of the studies focus on very low dimensional multi-modal problems
(2 or 3 dimensions) making this more complicated to asses theses
methods' scalability to high dimensions with better performance. The
benchmark tool includes 20 test functions (in some cases the same
function but with different dimension sizes), which includes 10 simple,
well-known and widely used benchmark functions, based on recent studies,
and more complex functions following the paradigm of composition
functions. In the following section, they will be briefly explained:

\begin{verbatim}
•   F1: Five-Uneven-Peak Trap (1D)
•   F2: Equal Maxima (1D)
•   F3: Uneven Decreasing Maxima (1D)
•   F4: Himmelblau (2D)
•   F5: Six-Hump Camel Back (2D)
•   F6: Shubert (2D, 3D)
•   F7: Vincent (2D, 3D)
•   F8: Modified Rastrigin - All Global Optima (2D)
•   F9: Composition Function 1 (2D)
•   F10: Composition Function 2 (2D)
•   F11: Composition Function 3 (2D, 3D, 5D, 10D)
•   F12: Composition Function 4 (3D, 5D, 10D, 20D)
\end{verbatim}

All of the test functions are formulated as maximisation problems. F1,
F2 and F3 are simple 1D multimodal functions, while F4 and F5 are simple
2D functions and not scalable. F6 to F8 are scalable multimodal
functions. The number of global optima for F6 and F7 are determined by
the dimension. However, for F8, the number of global optima is
independent of the dimension, therefore it can be controlled by the
user. F9 to F12 are scalable multimodal functions constructed by several
basic functions with different properties (Sphere function, Grienwank,
Rastrigin, Weierstrass and the Expanded Griewank's plus Rosenbrock's
function). F9 and F10 are separable, and non-symmetric, while F11 and
F12 are non-separable, non-symmetric complex multimodal functions. The
number of global optima in all of the composition functions is
independent of the number of dimensions, therefore can be controlled by
the user (Epitropakis et al. 2013).

\textbf{Maybe write each math equation or the R code}

Besides the 12 different functions, also, a \(count_goptima\) function
was included in order to find the optimal value for each evaluated
function in each iteration. Here, some optimal values are already given
as well as the number of each value to find. Together with an accuracy
rate, the evaluation starts in a cycle and stores the possible optimal
values in order to be compared with the expected values using the
accuracy rates of 0.1, 0.01, 0.001, 0.0001 and 0.00001 for each
different iteration. Once the optimal values are found and after
comparing them with the same results of the Matlab implementation, it
can be concluded that the new implementation works as expected and is
ready for submission.

\textbf{Explain maybe a little bit more, maybe a pseudocode}

https://en.wikipedia.org/wiki/IEEE\_Congress\_on\_Evolutionary\_Computation

\subsection{Implementation and
Pitfalls}\label{implementation-and-pitfalls}

During the developing time, an issue raised with the CEC benchmark tool.
In order to compare the R implementation of the NMMSO algorithm with the
original one, it was mandatory to use this tool to test each of its
functions with the new algorithm and compare results. After several
complications with the original test suite (these complications will be
addressed in the pitfalls' section), it was decided to recode each of
the functions as an independent R package to avoid any further
complication and having an easier and more trustworthy comparison of the
NMMSO algorithm in R.

Pitfalls

When the decision of re-writing the whole benchmark tool in the R
programming language was made, some issues came up regarding the way
Matlab handles the matrices and vectors. After analysing the Matlab
implementation,

\textbf{Results in R}\\
 • f\_ 1 : f(1\ldots{}1) = 120 • f\_ 2 : f(1\ldots{}1) = 5.270904e-92 •
f\_ 3 : f(1\ldots{}1) = 0.02501472 • f\_ 4 : f(1\ldots{}1) = 94 • f\_ 5
: f(1\ldots{}1) = -3.233333 • f\_ 6 : f(1\ldots{}1) = -3.180351 • f\_ 7
: f(1\ldots{}1) = 0 • f\_ 8 : f(1\ldots{}1) = 5.671692 • f\_ 9 :
f(1\ldots{}1) = 0 • f\_ 10 : f(1\ldots{}1) = -38 • f\_ 11 :
f(1\ldots{}1) = -268.6638 • f\_ 12 : f(1\ldots{}1) = -758.9333 • f\_ 13
: f(1\ldots{}1) = -613.5412 • f\_ 14 : f(1\ldots{}1) = -1838.556 • f\_
15 : f(1\ldots{}1) = -1049.86 • f\_ 16 : f(1\ldots{}1) = -2149.558 • f\_
17 : f(1\ldots{}1) = -1238.211 • f\_ 18 : f(1\ldots{}1) = -1683.184 •
f\_ 19 : f(1\ldots{}1) = -1342.819 • f\_ 20 : f(1\ldots{}1) = -1337.852

\begin{center}\rule{0.5\linewidth}{\linethickness}\end{center}

\section{The Implementation}\label{the-implementation}

\subsection{Structure of the project}\label{structure-of-the-project}

In difference to the original implementation it was chosen to split up
all single functions of the nmmso algorithm into single files. These
were bundled into the standard R package structure to give the
possibility to make it available over CRAN\footnote{Comprehensive R
  Archive Network.} in the future. To and give the possibilty to
collaborate the project was managed and versioned via Github. The
package was tested with testthat\footnote{https://cran.r-project.org/web/packages/testthat/index.html}
and documented with roxygen2\footnote{https://cran.r-project.org/web/packages/roxygen2/index.html}.
To assure functioning the package was continuously tested with Travis
CI.

After analysing the algorithm provided in Matlab by Dr.~Fieldsend, it
was decided to first translate each of the functions into the R
programming language. At first instance, this task seemed to be simple
because most of the functions were basically managing matrices and
vectors, but later this became a problem that will be covered in the
pitfalls' section (4.2) of this paper.

Once all the NMMSO functions existed in R and having the input data, the
testing phase started. It has been said, that one of the biggest
problems when coding an already existing program into another
programming language, is the different behaviours corresponding to each
object (in case of an object-oriented language) or its main structure.
The first runs came with several errors regarding the matrix generation
and handling, slowing down the project in a near future. Using GitHub,
it was easier to attack these problems in parallel, having one developer
reviewing different functions and the other one, fixing other bugs and
continue the testing phase. Also, this was achieved in an easier way,
thanks to that each function was coded in an independent R file, making
easier and faster the debugging and the fixing of each problem.

\subsection{Pitfalls and Problems}\label{pitfalls-and-problems}

In the beginning of the project, after translating each function into
the R programming language and in order to start testing each one of
them, some data was missing. The algorithm runs with some parameters
that had to be given by the CEC so it was necessary to contact the
organisation and ask for the files including the benchmarking tool.
Also, Dr.~Fieldsend was contacted in order to have a better
understanding of the general structure of his algorithm. Once the data
was completed, the test phase started, trying to get the same results as
the original one, but this was not possible at first instance because
the output was not generalised and most of the data generated inside the
algorithm were random values to be evaluated.

During the implementation of the NMMSO.R algorithm, some problems
regarding the language structure came up immediately. After some first
test cases, it was discovered how different Matlab and R work with
matrices and vectors. In the case of Matlab, every time a value wanted
to be added for an inexistent index, this was created automatically and
added to the data structure, where no row was existing before, instead
of throwing the common ``index out of bounds'' error. First, it was
necessary to compare all these possible behaviors in Matlab so a general
way to attack them could be implemented and after few tries, two
function-files were created. ``add\_col.R'', as its name says, is in
charge to add a new column into a structure, it is just necessary to
specify the original structure where it will be added to, the index and
the new object containing the information to add. This function
considers the cases if the new object is a vector, a matrix or just a
single value, so the original matrix could be modified and returned as
desired. ``add\_row.R'', on the other hand, simply imitates the behavior
of ``add\_col.R'' but as a transpose matrix in order to add the rows.

Once the data was completed in order to run the algorithm and get at
least a similar output as the original version, the CEC benchmarking
tool was intended to be used for testing. Thanks to the CEC
organisation, these tools were provided in C++, Java and Matlab to check
which one could be the best implementation for this case. After some
trials with the C++ version, some issues regarding its implementation
and missing documentation lead the development team to re-write the
complete benchmarking tool in the R programming language, in order to
avoid mixing code and be sure to get the results expected since the
beginning of this project. Also, it was thought in a way to supply the
research community with a new implementation of this tool for future
projects.

After solving the previous issues and starting to run the whole
algorithm, a memory problem came up. Mainly because of the continuous
problem between R and Matlab, this was caused, because one matrix,
originally containing only integer values, was added a float value in a
certain point within the process. R automatically cloned the integer
matrix, creating it with a float type in order to continue without
errors for every iteration. In the end having an unnecessary number of
matrices allocated in memory, making impossible the computation of the
algorithm with a size of 7.9 GB during the iteration 700 out of 50,000
(depending on the function to be evaluated in CEC benchmark suite, the
total number of iterations would vary). Even if this issue was solved
only by changing the original matrix to contain float values since the
beginning, it caused several problems and time consumption during the
testing phase.

Finally, after completing and fixing the R implementation, and trying to
test it with the real number of iterations for each of the twenty CEC
benchmarking functions, a bigger computation power was needed. Having
only students' computers for development, it was necessary a bigger
source of computation power and with the help of the R library
``BatchJobs'' together with the Palma Cluster, property of the
Westfälische Wilhelms-Universität located in Münster, Germany, it was
possible to run the algorithm in different batch jobs within the cluster
and just printing output files in order to analyse the data in the end.

\section{Benchmark and Comparison}\label{benchmark-and-comparison}

To compare nmmso.R with the original NMMSO the CEC test cases were used
to run the same benchmarks as in the original submission (Fieldsend
2014). There 4 different Ratios were used to measure the performance of
certain algorithms. Three of those measures (Peak Ratio, Success Ratio
and Convergence Speed) have been introduced in (Epitropakis et al. 2013,
pp. 6--7) to create a common point of comparison. The fourth ratio is
special for the nmmso algorithm since it tracks the number of swarms
over the iterations of the algorithm. Nmmso.R uses the same measures to
reach the highest comparability possible.

The first measure used is the Success Ratio (SR). The Success Ratio is
defined as the percentage of successful runs (runs that found all global
optima) over all runs (Li et al. 2013, p. 7). As for the other ratios
this measure was taken over several independent runs and collectively
evaluated. The taken measures for the Success Ratio can be found in
Table 2. \[\frac{successful\ runs}{NR} = SR \] Here \(NR\) denotes the
Number of runs done to reach this measure. \newline

\begin{longtable}[c]{@{}crrrrrr@{}}
\caption{Success Ratio over given runs (Measure of share of runs which
found all global optima)}\tabularnewline
\toprule
\begin{minipage}[b]{0.11\columnwidth}\centering\strut
~
\strut\end{minipage} &
\begin{minipage}[b]{0.07\columnwidth}\raggedleft\strut
0.1
\strut\end{minipage} &
\begin{minipage}[b]{0.08\columnwidth}\raggedleft\strut
0.01
\strut\end{minipage} &
\begin{minipage}[b]{0.09\columnwidth}\raggedleft\strut
0.001
\strut\end{minipage} &
\begin{minipage}[b]{0.10\columnwidth}\raggedleft\strut
0.0001
\strut\end{minipage} &
\begin{minipage}[b]{0.11\columnwidth}\raggedleft\strut
0.00001
\strut\end{minipage} &
\begin{minipage}[b]{0.07\columnwidth}\raggedleft\strut
runs
\strut\end{minipage}\tabularnewline
\midrule
\endfirsthead
\toprule
\begin{minipage}[b]{0.11\columnwidth}\centering\strut
~
\strut\end{minipage} &
\begin{minipage}[b]{0.07\columnwidth}\raggedleft\strut
0.1
\strut\end{minipage} &
\begin{minipage}[b]{0.08\columnwidth}\raggedleft\strut
0.01
\strut\end{minipage} &
\begin{minipage}[b]{0.09\columnwidth}\raggedleft\strut
0.001
\strut\end{minipage} &
\begin{minipage}[b]{0.10\columnwidth}\raggedleft\strut
0.0001
\strut\end{minipage} &
\begin{minipage}[b]{0.11\columnwidth}\raggedleft\strut
0.00001
\strut\end{minipage} &
\begin{minipage}[b]{0.07\columnwidth}\raggedleft\strut
runs
\strut\end{minipage}\tabularnewline
\midrule
\endhead
\begin{minipage}[t]{0.11\columnwidth}\centering\strut
\textbf{F1}
\strut\end{minipage} &
\begin{minipage}[t]{0.07\columnwidth}\raggedleft\strut
1
\strut\end{minipage} &
\begin{minipage}[t]{0.08\columnwidth}\raggedleft\strut
1
\strut\end{minipage} &
\begin{minipage}[t]{0.09\columnwidth}\raggedleft\strut
1
\strut\end{minipage} &
\begin{minipage}[t]{0.10\columnwidth}\raggedleft\strut
1
\strut\end{minipage} &
\begin{minipage}[t]{0.11\columnwidth}\raggedleft\strut
1
\strut\end{minipage} &
\begin{minipage}[t]{0.07\columnwidth}\raggedleft\strut
45
\strut\end{minipage}\tabularnewline
\begin{minipage}[t]{0.11\columnwidth}\centering\strut
\textbf{F2}
\strut\end{minipage} &
\begin{minipage}[t]{0.07\columnwidth}\raggedleft\strut
1
\strut\end{minipage} &
\begin{minipage}[t]{0.08\columnwidth}\raggedleft\strut
1
\strut\end{minipage} &
\begin{minipage}[t]{0.09\columnwidth}\raggedleft\strut
1
\strut\end{minipage} &
\begin{minipage}[t]{0.10\columnwidth}\raggedleft\strut
1
\strut\end{minipage} &
\begin{minipage}[t]{0.11\columnwidth}\raggedleft\strut
1
\strut\end{minipage} &
\begin{minipage}[t]{0.07\columnwidth}\raggedleft\strut
42
\strut\end{minipage}\tabularnewline
\begin{minipage}[t]{0.11\columnwidth}\centering\strut
\textbf{F3}
\strut\end{minipage} &
\begin{minipage}[t]{0.07\columnwidth}\raggedleft\strut
1
\strut\end{minipage} &
\begin{minipage}[t]{0.08\columnwidth}\raggedleft\strut
1
\strut\end{minipage} &
\begin{minipage}[t]{0.09\columnwidth}\raggedleft\strut
1
\strut\end{minipage} &
\begin{minipage}[t]{0.10\columnwidth}\raggedleft\strut
1
\strut\end{minipage} &
\begin{minipage}[t]{0.11\columnwidth}\raggedleft\strut
1
\strut\end{minipage} &
\begin{minipage}[t]{0.07\columnwidth}\raggedleft\strut
44
\strut\end{minipage}\tabularnewline
\begin{minipage}[t]{0.11\columnwidth}\centering\strut
\textbf{F4}
\strut\end{minipage} &
\begin{minipage}[t]{0.07\columnwidth}\raggedleft\strut
1
\strut\end{minipage} &
\begin{minipage}[t]{0.08\columnwidth}\raggedleft\strut
1
\strut\end{minipage} &
\begin{minipage}[t]{0.09\columnwidth}\raggedleft\strut
1
\strut\end{minipage} &
\begin{minipage}[t]{0.10\columnwidth}\raggedleft\strut
1
\strut\end{minipage} &
\begin{minipage}[t]{0.11\columnwidth}\raggedleft\strut
1
\strut\end{minipage} &
\begin{minipage}[t]{0.07\columnwidth}\raggedleft\strut
44
\strut\end{minipage}\tabularnewline
\begin{minipage}[t]{0.11\columnwidth}\centering\strut
\textbf{F5}
\strut\end{minipage} &
\begin{minipage}[t]{0.07\columnwidth}\raggedleft\strut
1
\strut\end{minipage} &
\begin{minipage}[t]{0.08\columnwidth}\raggedleft\strut
1
\strut\end{minipage} &
\begin{minipage}[t]{0.09\columnwidth}\raggedleft\strut
1
\strut\end{minipage} &
\begin{minipage}[t]{0.10\columnwidth}\raggedleft\strut
1
\strut\end{minipage} &
\begin{minipage}[t]{0.11\columnwidth}\raggedleft\strut
1
\strut\end{minipage} &
\begin{minipage}[t]{0.07\columnwidth}\raggedleft\strut
41
\strut\end{minipage}\tabularnewline
\begin{minipage}[t]{0.11\columnwidth}\centering\strut
\textbf{F6}
\strut\end{minipage} &
\begin{minipage}[t]{0.07\columnwidth}\raggedleft\strut
1
\strut\end{minipage} &
\begin{minipage}[t]{0.08\columnwidth}\raggedleft\strut
1
\strut\end{minipage} &
\begin{minipage}[t]{0.09\columnwidth}\raggedleft\strut
1
\strut\end{minipage} &
\begin{minipage}[t]{0.10\columnwidth}\raggedleft\strut
1
\strut\end{minipage} &
\begin{minipage}[t]{0.11\columnwidth}\raggedleft\strut
0
\strut\end{minipage} &
\begin{minipage}[t]{0.07\columnwidth}\raggedleft\strut
40
\strut\end{minipage}\tabularnewline
\begin{minipage}[t]{0.11\columnwidth}\centering\strut
\textbf{F7}
\strut\end{minipage} &
\begin{minipage}[t]{0.07\columnwidth}\raggedleft\strut
1
\strut\end{minipage} &
\begin{minipage}[t]{0.08\columnwidth}\raggedleft\strut
1
\strut\end{minipage} &
\begin{minipage}[t]{0.09\columnwidth}\raggedleft\strut
1
\strut\end{minipage} &
\begin{minipage}[t]{0.10\columnwidth}\raggedleft\strut
1
\strut\end{minipage} &
\begin{minipage}[t]{0.11\columnwidth}\raggedleft\strut
1
\strut\end{minipage} &
\begin{minipage}[t]{0.07\columnwidth}\raggedleft\strut
41
\strut\end{minipage}\tabularnewline
\begin{minipage}[t]{0.11\columnwidth}\centering\strut
\textbf{F8}
\strut\end{minipage} &
\begin{minipage}[t]{0.07\columnwidth}\raggedleft\strut
1
\strut\end{minipage} &
\begin{minipage}[t]{0.08\columnwidth}\raggedleft\strut
1
\strut\end{minipage} &
\begin{minipage}[t]{0.09\columnwidth}\raggedleft\strut
1
\strut\end{minipage} &
\begin{minipage}[t]{0.10\columnwidth}\raggedleft\strut
0.78
\strut\end{minipage} &
\begin{minipage}[t]{0.11\columnwidth}\raggedleft\strut
0.61
\strut\end{minipage} &
\begin{minipage}[t]{0.07\columnwidth}\raggedleft\strut
23
\strut\end{minipage}\tabularnewline
\begin{minipage}[t]{0.11\columnwidth}\centering\strut
\textbf{F9}
\strut\end{minipage} &
\begin{minipage}[t]{0.07\columnwidth}\raggedleft\strut
0.96
\strut\end{minipage} &
\begin{minipage}[t]{0.08\columnwidth}\raggedleft\strut
0.96
\strut\end{minipage} &
\begin{minipage}[t]{0.09\columnwidth}\raggedleft\strut
0.96
\strut\end{minipage} &
\begin{minipage}[t]{0.10\columnwidth}\raggedleft\strut
0.93
\strut\end{minipage} &
\begin{minipage}[t]{0.11\columnwidth}\raggedleft\strut
0.93
\strut\end{minipage} &
\begin{minipage}[t]{0.07\columnwidth}\raggedleft\strut
27
\strut\end{minipage}\tabularnewline
\begin{minipage}[t]{0.11\columnwidth}\centering\strut
\textbf{F10}
\strut\end{minipage} &
\begin{minipage}[t]{0.07\columnwidth}\raggedleft\strut
1
\strut\end{minipage} &
\begin{minipage}[t]{0.08\columnwidth}\raggedleft\strut
1
\strut\end{minipage} &
\begin{minipage}[t]{0.09\columnwidth}\raggedleft\strut
1
\strut\end{minipage} &
\begin{minipage}[t]{0.10\columnwidth}\raggedleft\strut
1
\strut\end{minipage} &
\begin{minipage}[t]{0.11\columnwidth}\raggedleft\strut
1
\strut\end{minipage} &
\begin{minipage}[t]{0.07\columnwidth}\raggedleft\strut
41
\strut\end{minipage}\tabularnewline
\begin{minipage}[t]{0.11\columnwidth}\centering\strut
\textbf{F11}
\strut\end{minipage} &
\begin{minipage}[t]{0.07\columnwidth}\raggedleft\strut
1
\strut\end{minipage} &
\begin{minipage}[t]{0.08\columnwidth}\raggedleft\strut
1
\strut\end{minipage} &
\begin{minipage}[t]{0.09\columnwidth}\raggedleft\strut
1
\strut\end{minipage} &
\begin{minipage}[t]{0.10\columnwidth}\raggedleft\strut
1
\strut\end{minipage} &
\begin{minipage}[t]{0.11\columnwidth}\raggedleft\strut
1
\strut\end{minipage} &
\begin{minipage}[t]{0.07\columnwidth}\raggedleft\strut
40
\strut\end{minipage}\tabularnewline
\begin{minipage}[t]{0.11\columnwidth}\centering\strut
\textbf{F12}
\strut\end{minipage} &
\begin{minipage}[t]{0.07\columnwidth}\raggedleft\strut
1
\strut\end{minipage} &
\begin{minipage}[t]{0.08\columnwidth}\raggedleft\strut
1
\strut\end{minipage} &
\begin{minipage}[t]{0.09\columnwidth}\raggedleft\strut
1
\strut\end{minipage} &
\begin{minipage}[t]{0.10\columnwidth}\raggedleft\strut
1
\strut\end{minipage} &
\begin{minipage}[t]{0.11\columnwidth}\raggedleft\strut
1
\strut\end{minipage} &
\begin{minipage}[t]{0.07\columnwidth}\raggedleft\strut
41
\strut\end{minipage}\tabularnewline
\begin{minipage}[t]{0.11\columnwidth}\centering\strut
\textbf{F13}
\strut\end{minipage} &
\begin{minipage}[t]{0.07\columnwidth}\raggedleft\strut
1
\strut\end{minipage} &
\begin{minipage}[t]{0.08\columnwidth}\raggedleft\strut
1
\strut\end{minipage} &
\begin{minipage}[t]{0.09\columnwidth}\raggedleft\strut
1
\strut\end{minipage} &
\begin{minipage}[t]{0.10\columnwidth}\raggedleft\strut
1
\strut\end{minipage} &
\begin{minipage}[t]{0.11\columnwidth}\raggedleft\strut
1
\strut\end{minipage} &
\begin{minipage}[t]{0.07\columnwidth}\raggedleft\strut
41
\strut\end{minipage}\tabularnewline
\begin{minipage}[t]{0.11\columnwidth}\centering\strut
\textbf{F14}
\strut\end{minipage} &
\begin{minipage}[t]{0.07\columnwidth}\raggedleft\strut
1
\strut\end{minipage} &
\begin{minipage}[t]{0.08\columnwidth}\raggedleft\strut
1
\strut\end{minipage} &
\begin{minipage}[t]{0.09\columnwidth}\raggedleft\strut
1
\strut\end{minipage} &
\begin{minipage}[t]{0.10\columnwidth}\raggedleft\strut
1
\strut\end{minipage} &
\begin{minipage}[t]{0.11\columnwidth}\raggedleft\strut
1
\strut\end{minipage} &
\begin{minipage}[t]{0.07\columnwidth}\raggedleft\strut
39
\strut\end{minipage}\tabularnewline
\begin{minipage}[t]{0.11\columnwidth}\centering\strut
\textbf{F15}
\strut\end{minipage} &
\begin{minipage}[t]{0.07\columnwidth}\raggedleft\strut
0.97
\strut\end{minipage} &
\begin{minipage}[t]{0.08\columnwidth}\raggedleft\strut
0.97
\strut\end{minipage} &
\begin{minipage}[t]{0.09\columnwidth}\raggedleft\strut
0.97
\strut\end{minipage} &
\begin{minipage}[t]{0.10\columnwidth}\raggedleft\strut
0.97
\strut\end{minipage} &
\begin{minipage}[t]{0.11\columnwidth}\raggedleft\strut
0.97
\strut\end{minipage} &
\begin{minipage}[t]{0.07\columnwidth}\raggedleft\strut
31
\strut\end{minipage}\tabularnewline
\begin{minipage}[t]{0.11\columnwidth}\centering\strut
\textbf{F16}
\strut\end{minipage} &
\begin{minipage}[t]{0.07\columnwidth}\raggedleft\strut
0
\strut\end{minipage} &
\begin{minipage}[t]{0.08\columnwidth}\raggedleft\strut
0
\strut\end{minipage} &
\begin{minipage}[t]{0.09\columnwidth}\raggedleft\strut
0
\strut\end{minipage} &
\begin{minipage}[t]{0.10\columnwidth}\raggedleft\strut
0
\strut\end{minipage} &
\begin{minipage}[t]{0.11\columnwidth}\raggedleft\strut
0
\strut\end{minipage} &
\begin{minipage}[t]{0.07\columnwidth}\raggedleft\strut
20
\strut\end{minipage}\tabularnewline
\begin{minipage}[t]{0.11\columnwidth}\centering\strut
\textbf{F17}
\strut\end{minipage} &
\begin{minipage}[t]{0.07\columnwidth}\raggedleft\strut
0.21
\strut\end{minipage} &
\begin{minipage}[t]{0.08\columnwidth}\raggedleft\strut
0.11
\strut\end{minipage} &
\begin{minipage}[t]{0.09\columnwidth}\raggedleft\strut
0.11
\strut\end{minipage} &
\begin{minipage}[t]{0.10\columnwidth}\raggedleft\strut
0.11
\strut\end{minipage} &
\begin{minipage}[t]{0.11\columnwidth}\raggedleft\strut
0.11
\strut\end{minipage} &
\begin{minipage}[t]{0.07\columnwidth}\raggedleft\strut
19
\strut\end{minipage}\tabularnewline
\begin{minipage}[t]{0.11\columnwidth}\centering\strut
\textbf{F18}
\strut\end{minipage} &
\begin{minipage}[t]{0.07\columnwidth}\raggedleft\strut
0.39
\strut\end{minipage} &
\begin{minipage}[t]{0.08\columnwidth}\raggedleft\strut
0.39
\strut\end{minipage} &
\begin{minipage}[t]{0.09\columnwidth}\raggedleft\strut
0.35
\strut\end{minipage} &
\begin{minipage}[t]{0.10\columnwidth}\raggedleft\strut
0.3
\strut\end{minipage} &
\begin{minipage}[t]{0.11\columnwidth}\raggedleft\strut
0.3
\strut\end{minipage} &
\begin{minipage}[t]{0.07\columnwidth}\raggedleft\strut
23
\strut\end{minipage}\tabularnewline
\begin{minipage}[t]{0.11\columnwidth}\centering\strut
\textbf{F19}
\strut\end{minipage} &
\begin{minipage}[t]{0.07\columnwidth}\raggedleft\strut
0
\strut\end{minipage} &
\begin{minipage}[t]{0.08\columnwidth}\raggedleft\strut
0
\strut\end{minipage} &
\begin{minipage}[t]{0.09\columnwidth}\raggedleft\strut
0
\strut\end{minipage} &
\begin{minipage}[t]{0.10\columnwidth}\raggedleft\strut
0
\strut\end{minipage} &
\begin{minipage}[t]{0.11\columnwidth}\raggedleft\strut
0
\strut\end{minipage} &
\begin{minipage}[t]{0.07\columnwidth}\raggedleft\strut
18
\strut\end{minipage}\tabularnewline
\begin{minipage}[t]{0.11\columnwidth}\centering\strut
\textbf{F20}
\strut\end{minipage} &
\begin{minipage}[t]{0.07\columnwidth}\raggedleft\strut
0
\strut\end{minipage} &
\begin{minipage}[t]{0.08\columnwidth}\raggedleft\strut
0
\strut\end{minipage} &
\begin{minipage}[t]{0.09\columnwidth}\raggedleft\strut
0
\strut\end{minipage} &
\begin{minipage}[t]{0.10\columnwidth}\raggedleft\strut
0
\strut\end{minipage} &
\begin{minipage}[t]{0.11\columnwidth}\raggedleft\strut
0
\strut\end{minipage} &
\begin{minipage}[t]{0.07\columnwidth}\raggedleft\strut
17
\strut\end{minipage}\tabularnewline
\bottomrule
\end{longtable}

The second measure introduced by the CEC committee and also used by
Dr.~Fieldsend is the Convergence Rate. The Convergence Rate (CR)
measures the needed evaluations per Accuracy and Function to find all
global optima (Li et al. 2013, p. 7). This measure takes the mean of
evaluations over all runs. The results of this measure can be found in
Table 3.

\[\frac{\sum\nolimits_{n=1}^{NR} evals_{n}}{NR} = CR\] In this measure,
\(evals\) denotes the number of evaluations done. \newline

\begin{longtable}[c]{@{}crrrrrr@{}}
\caption{Convergence Rates over given runs (Mean of evaluations needed
to find all global optima, if all optima have never been found the
maximum allowed evaluations for that function were
taken.)}\tabularnewline
\toprule
\begin{minipage}[b]{0.11\columnwidth}\centering\strut
~
\strut\end{minipage} &
\begin{minipage}[b]{0.08\columnwidth}\raggedleft\strut
0.1
\strut\end{minipage} &
\begin{minipage}[b]{0.08\columnwidth}\raggedleft\strut
0.01
\strut\end{minipage} &
\begin{minipage}[b]{0.09\columnwidth}\raggedleft\strut
0.001
\strut\end{minipage} &
\begin{minipage}[b]{0.10\columnwidth}\raggedleft\strut
0.0001
\strut\end{minipage} &
\begin{minipage}[b]{0.11\columnwidth}\raggedleft\strut
0.00001
\strut\end{minipage} &
\begin{minipage}[b]{0.07\columnwidth}\raggedleft\strut
runs
\strut\end{minipage}\tabularnewline
\midrule
\endfirsthead
\toprule
\begin{minipage}[b]{0.11\columnwidth}\centering\strut
~
\strut\end{minipage} &
\begin{minipage}[b]{0.08\columnwidth}\raggedleft\strut
0.1
\strut\end{minipage} &
\begin{minipage}[b]{0.08\columnwidth}\raggedleft\strut
0.01
\strut\end{minipage} &
\begin{minipage}[b]{0.09\columnwidth}\raggedleft\strut
0.001
\strut\end{minipage} &
\begin{minipage}[b]{0.10\columnwidth}\raggedleft\strut
0.0001
\strut\end{minipage} &
\begin{minipage}[b]{0.11\columnwidth}\raggedleft\strut
0.00001
\strut\end{minipage} &
\begin{minipage}[b]{0.07\columnwidth}\raggedleft\strut
runs
\strut\end{minipage}\tabularnewline
\midrule
\endhead
\begin{minipage}[t]{0.11\columnwidth}\centering\strut
\textbf{F1}
\strut\end{minipage} &
\begin{minipage}[t]{0.08\columnwidth}\raggedleft\strut
610
\strut\end{minipage} &
\begin{minipage}[t]{0.08\columnwidth}\raggedleft\strut
814
\strut\end{minipage} &
\begin{minipage}[t]{0.09\columnwidth}\raggedleft\strut
1010
\strut\end{minipage} &
\begin{minipage}[t]{0.10\columnwidth}\raggedleft\strut
1204
\strut\end{minipage} &
\begin{minipage}[t]{0.11\columnwidth}\raggedleft\strut
1434
\strut\end{minipage} &
\begin{minipage}[t]{0.07\columnwidth}\raggedleft\strut
45
\strut\end{minipage}\tabularnewline
\begin{minipage}[t]{0.11\columnwidth}\centering\strut
\textbf{F2}
\strut\end{minipage} &
\begin{minipage}[t]{0.08\columnwidth}\raggedleft\strut
181
\strut\end{minipage} &
\begin{minipage}[t]{0.08\columnwidth}\raggedleft\strut
265
\strut\end{minipage} &
\begin{minipage}[t]{0.09\columnwidth}\raggedleft\strut
390
\strut\end{minipage} &
\begin{minipage}[t]{0.10\columnwidth}\raggedleft\strut
560
\strut\end{minipage} &
\begin{minipage}[t]{0.11\columnwidth}\raggedleft\strut
677
\strut\end{minipage} &
\begin{minipage}[t]{0.07\columnwidth}\raggedleft\strut
42
\strut\end{minipage}\tabularnewline
\begin{minipage}[t]{0.11\columnwidth}\centering\strut
\textbf{F3}
\strut\end{minipage} &
\begin{minipage}[t]{0.08\columnwidth}\raggedleft\strut
33
\strut\end{minipage} &
\begin{minipage}[t]{0.08\columnwidth}\raggedleft\strut
164
\strut\end{minipage} &
\begin{minipage}[t]{0.09\columnwidth}\raggedleft\strut
269
\strut\end{minipage} &
\begin{minipage}[t]{0.10\columnwidth}\raggedleft\strut
382
\strut\end{minipage} &
\begin{minipage}[t]{0.11\columnwidth}\raggedleft\strut
504
\strut\end{minipage} &
\begin{minipage}[t]{0.07\columnwidth}\raggedleft\strut
44
\strut\end{minipage}\tabularnewline
\begin{minipage}[t]{0.11\columnwidth}\centering\strut
\textbf{F4}
\strut\end{minipage} &
\begin{minipage}[t]{0.08\columnwidth}\raggedleft\strut
501
\strut\end{minipage} &
\begin{minipage}[t]{0.08\columnwidth}\raggedleft\strut
728
\strut\end{minipage} &
\begin{minipage}[t]{0.09\columnwidth}\raggedleft\strut
951
\strut\end{minipage} &
\begin{minipage}[t]{0.10\columnwidth}\raggedleft\strut
1202
\strut\end{minipage} &
\begin{minipage}[t]{0.11\columnwidth}\raggedleft\strut
1458
\strut\end{minipage} &
\begin{minipage}[t]{0.07\columnwidth}\raggedleft\strut
44
\strut\end{minipage}\tabularnewline
\begin{minipage}[t]{0.11\columnwidth}\centering\strut
\textbf{F5}
\strut\end{minipage} &
\begin{minipage}[t]{0.08\columnwidth}\raggedleft\strut
80
\strut\end{minipage} &
\begin{minipage}[t]{0.08\columnwidth}\raggedleft\strut
194
\strut\end{minipage} &
\begin{minipage}[t]{0.09\columnwidth}\raggedleft\strut
312
\strut\end{minipage} &
\begin{minipage}[t]{0.10\columnwidth}\raggedleft\strut
514
\strut\end{minipage} &
\begin{minipage}[t]{0.11\columnwidth}\raggedleft\strut
748
\strut\end{minipage} &
\begin{minipage}[t]{0.07\columnwidth}\raggedleft\strut
41
\strut\end{minipage}\tabularnewline
\begin{minipage}[t]{0.11\columnwidth}\centering\strut
\textbf{F6}
\strut\end{minipage} &
\begin{minipage}[t]{0.08\columnwidth}\raggedleft\strut
19124
\strut\end{minipage} &
\begin{minipage}[t]{0.08\columnwidth}\raggedleft\strut
24077
\strut\end{minipage} &
\begin{minipage}[t]{0.09\columnwidth}\raggedleft\strut
29970
\strut\end{minipage} &
\begin{minipage}[t]{0.10\columnwidth}\raggedleft\strut
41858
\strut\end{minipage} &
\begin{minipage}[t]{0.11\columnwidth}\raggedleft\strut
200001
\strut\end{minipage} &
\begin{minipage}[t]{0.07\columnwidth}\raggedleft\strut
40
\strut\end{minipage}\tabularnewline
\begin{minipage}[t]{0.11\columnwidth}\centering\strut
\textbf{F7}
\strut\end{minipage} &
\begin{minipage}[t]{0.08\columnwidth}\raggedleft\strut
8453
\strut\end{minipage} &
\begin{minipage}[t]{0.08\columnwidth}\raggedleft\strut
9073
\strut\end{minipage} &
\begin{minipage}[t]{0.09\columnwidth}\raggedleft\strut
10539
\strut\end{minipage} &
\begin{minipage}[t]{0.10\columnwidth}\raggedleft\strut
12097
\strut\end{minipage} &
\begin{minipage}[t]{0.11\columnwidth}\raggedleft\strut
14434
\strut\end{minipage} &
\begin{minipage}[t]{0.07\columnwidth}\raggedleft\strut
41
\strut\end{minipage}\tabularnewline
\begin{minipage}[t]{0.11\columnwidth}\centering\strut
\textbf{F8}
\strut\end{minipage} &
\begin{minipage}[t]{0.08\columnwidth}\raggedleft\strut
199028
\strut\end{minipage} &
\begin{minipage}[t]{0.08\columnwidth}\raggedleft\strut
239062
\strut\end{minipage} &
\begin{minipage}[t]{0.09\columnwidth}\raggedleft\strut
283957
\strut\end{minipage} &
\begin{minipage}[t]{0.10\columnwidth}\raggedleft\strut
331813
\strut\end{minipage} &
\begin{minipage}[t]{0.11\columnwidth}\raggedleft\strut
358588
\strut\end{minipage} &
\begin{minipage}[t]{0.07\columnwidth}\raggedleft\strut
23
\strut\end{minipage}\tabularnewline
\begin{minipage}[t]{0.11\columnwidth}\centering\strut
\textbf{F9}
\strut\end{minipage} &
\begin{minipage}[t]{0.08\columnwidth}\raggedleft\strut
190205
\strut\end{minipage} &
\begin{minipage}[t]{0.08\columnwidth}\raggedleft\strut
197213
\strut\end{minipage} &
\begin{minipage}[t]{0.09\columnwidth}\raggedleft\strut
213486
\strut\end{minipage} &
\begin{minipage}[t]{0.10\columnwidth}\raggedleft\strut
227226
\strut\end{minipage} &
\begin{minipage}[t]{0.11\columnwidth}\raggedleft\strut
235417
\strut\end{minipage} &
\begin{minipage}[t]{0.07\columnwidth}\raggedleft\strut
27
\strut\end{minipage}\tabularnewline
\begin{minipage}[t]{0.11\columnwidth}\centering\strut
\textbf{F10}
\strut\end{minipage} &
\begin{minipage}[t]{0.08\columnwidth}\raggedleft\strut
871
\strut\end{minipage} &
\begin{minipage}[t]{0.08\columnwidth}\raggedleft\strut
1309
\strut\end{minipage} &
\begin{minipage}[t]{0.09\columnwidth}\raggedleft\strut
1716
\strut\end{minipage} &
\begin{minipage}[t]{0.10\columnwidth}\raggedleft\strut
2252
\strut\end{minipage} &
\begin{minipage}[t]{0.11\columnwidth}\raggedleft\strut
2749
\strut\end{minipage} &
\begin{minipage}[t]{0.07\columnwidth}\raggedleft\strut
41
\strut\end{minipage}\tabularnewline
\begin{minipage}[t]{0.11\columnwidth}\centering\strut
\textbf{F11}
\strut\end{minipage} &
\begin{minipage}[t]{0.08\columnwidth}\raggedleft\strut
3743
\strut\end{minipage} &
\begin{minipage}[t]{0.08\columnwidth}\raggedleft\strut
5652
\strut\end{minipage} &
\begin{minipage}[t]{0.09\columnwidth}\raggedleft\strut
7159
\strut\end{minipage} &
\begin{minipage}[t]{0.10\columnwidth}\raggedleft\strut
8309
\strut\end{minipage} &
\begin{minipage}[t]{0.11\columnwidth}\raggedleft\strut
9173
\strut\end{minipage} &
\begin{minipage}[t]{0.07\columnwidth}\raggedleft\strut
40
\strut\end{minipage}\tabularnewline
\begin{minipage}[t]{0.11\columnwidth}\centering\strut
\textbf{F12}
\strut\end{minipage} &
\begin{minipage}[t]{0.08\columnwidth}\raggedleft\strut
17240
\strut\end{minipage} &
\begin{minipage}[t]{0.08\columnwidth}\raggedleft\strut
25441
\strut\end{minipage} &
\begin{minipage}[t]{0.09\columnwidth}\raggedleft\strut
36657
\strut\end{minipage} &
\begin{minipage}[t]{0.10\columnwidth}\raggedleft\strut
43765
\strut\end{minipage} &
\begin{minipage}[t]{0.11\columnwidth}\raggedleft\strut
50048
\strut\end{minipage} &
\begin{minipage}[t]{0.07\columnwidth}\raggedleft\strut
41
\strut\end{minipage}\tabularnewline
\begin{minipage}[t]{0.11\columnwidth}\centering\strut
\textbf{F13}
\strut\end{minipage} &
\begin{minipage}[t]{0.08\columnwidth}\raggedleft\strut
10221
\strut\end{minipage} &
\begin{minipage}[t]{0.08\columnwidth}\raggedleft\strut
14763
\strut\end{minipage} &
\begin{minipage}[t]{0.09\columnwidth}\raggedleft\strut
18100
\strut\end{minipage} &
\begin{minipage}[t]{0.10\columnwidth}\raggedleft\strut
21782
\strut\end{minipage} &
\begin{minipage}[t]{0.11\columnwidth}\raggedleft\strut
26547
\strut\end{minipage} &
\begin{minipage}[t]{0.07\columnwidth}\raggedleft\strut
41
\strut\end{minipage}\tabularnewline
\begin{minipage}[t]{0.11\columnwidth}\centering\strut
\textbf{F14}
\strut\end{minipage} &
\begin{minipage}[t]{0.08\columnwidth}\raggedleft\strut
28185
\strut\end{minipage} &
\begin{minipage}[t]{0.08\columnwidth}\raggedleft\strut
34579
\strut\end{minipage} &
\begin{minipage}[t]{0.09\columnwidth}\raggedleft\strut
47325
\strut\end{minipage} &
\begin{minipage}[t]{0.10\columnwidth}\raggedleft\strut
57497
\strut\end{minipage} &
\begin{minipage}[t]{0.11\columnwidth}\raggedleft\strut
65873
\strut\end{minipage} &
\begin{minipage}[t]{0.07\columnwidth}\raggedleft\strut
39
\strut\end{minipage}\tabularnewline
\begin{minipage}[t]{0.11\columnwidth}\centering\strut
\textbf{F15}
\strut\end{minipage} &
\begin{minipage}[t]{0.08\columnwidth}\raggedleft\strut
106836
\strut\end{minipage} &
\begin{minipage}[t]{0.08\columnwidth}\raggedleft\strut
125918
\strut\end{minipage} &
\begin{minipage}[t]{0.09\columnwidth}\raggedleft\strut
144279
\strut\end{minipage} &
\begin{minipage}[t]{0.10\columnwidth}\raggedleft\strut
165564
\strut\end{minipage} &
\begin{minipage}[t]{0.11\columnwidth}\raggedleft\strut
180095
\strut\end{minipage} &
\begin{minipage}[t]{0.07\columnwidth}\raggedleft\strut
31
\strut\end{minipage}\tabularnewline
\begin{minipage}[t]{0.11\columnwidth}\centering\strut
\textbf{F16}
\strut\end{minipage} &
\begin{minipage}[t]{0.08\columnwidth}\raggedleft\strut
400001
\strut\end{minipage} &
\begin{minipage}[t]{0.08\columnwidth}\raggedleft\strut
400001
\strut\end{minipage} &
\begin{minipage}[t]{0.09\columnwidth}\raggedleft\strut
400001
\strut\end{minipage} &
\begin{minipage}[t]{0.10\columnwidth}\raggedleft\strut
400001
\strut\end{minipage} &
\begin{minipage}[t]{0.11\columnwidth}\raggedleft\strut
400001
\strut\end{minipage} &
\begin{minipage}[t]{0.07\columnwidth}\raggedleft\strut
20
\strut\end{minipage}\tabularnewline
\begin{minipage}[t]{0.11\columnwidth}\centering\strut
\textbf{F17}
\strut\end{minipage} &
\begin{minipage}[t]{0.08\columnwidth}\raggedleft\strut
352657
\strut\end{minipage} &
\begin{minipage}[t]{0.08\columnwidth}\raggedleft\strut
368393
\strut\end{minipage} &
\begin{minipage}[t]{0.09\columnwidth}\raggedleft\strut
372283
\strut\end{minipage} &
\begin{minipage}[t]{0.10\columnwidth}\raggedleft\strut
373597
\strut\end{minipage} &
\begin{minipage}[t]{0.11\columnwidth}\raggedleft\strut
374462
\strut\end{minipage} &
\begin{minipage}[t]{0.07\columnwidth}\raggedleft\strut
19
\strut\end{minipage}\tabularnewline
\begin{minipage}[t]{0.11\columnwidth}\centering\strut
\textbf{F18}
\strut\end{minipage} &
\begin{minipage}[t]{0.08\columnwidth}\raggedleft\strut
299275
\strut\end{minipage} &
\begin{minipage}[t]{0.08\columnwidth}\raggedleft\strut
302915
\strut\end{minipage} &
\begin{minipage}[t]{0.09\columnwidth}\raggedleft\strut
313231
\strut\end{minipage} &
\begin{minipage}[t]{0.10\columnwidth}\raggedleft\strut
318487
\strut\end{minipage} &
\begin{minipage}[t]{0.11\columnwidth}\raggedleft\strut
320204
\strut\end{minipage} &
\begin{minipage}[t]{0.07\columnwidth}\raggedleft\strut
23
\strut\end{minipage}\tabularnewline
\begin{minipage}[t]{0.11\columnwidth}\centering\strut
\textbf{F19}
\strut\end{minipage} &
\begin{minipage}[t]{0.08\columnwidth}\raggedleft\strut
400001
\strut\end{minipage} &
\begin{minipage}[t]{0.08\columnwidth}\raggedleft\strut
400001
\strut\end{minipage} &
\begin{minipage}[t]{0.09\columnwidth}\raggedleft\strut
400001
\strut\end{minipage} &
\begin{minipage}[t]{0.10\columnwidth}\raggedleft\strut
400001
\strut\end{minipage} &
\begin{minipage}[t]{0.11\columnwidth}\raggedleft\strut
400001
\strut\end{minipage} &
\begin{minipage}[t]{0.07\columnwidth}\raggedleft\strut
18
\strut\end{minipage}\tabularnewline
\begin{minipage}[t]{0.11\columnwidth}\centering\strut
\textbf{F20}
\strut\end{minipage} &
\begin{minipage}[t]{0.08\columnwidth}\raggedleft\strut
400001
\strut\end{minipage} &
\begin{minipage}[t]{0.08\columnwidth}\raggedleft\strut
400001
\strut\end{minipage} &
\begin{minipage}[t]{0.09\columnwidth}\raggedleft\strut
400001
\strut\end{minipage} &
\begin{minipage}[t]{0.10\columnwidth}\raggedleft\strut
400001
\strut\end{minipage} &
\begin{minipage}[t]{0.11\columnwidth}\raggedleft\strut
400001
\strut\end{minipage} &
\begin{minipage}[t]{0.07\columnwidth}\raggedleft\strut
17
\strut\end{minipage}\tabularnewline
\bottomrule
\end{longtable}

The third measure is the Peak Ratio (PR). It measures the share of found
global optima over all runs (Li et al. 2013, p. 7). The results of this
evaluation can be found in Table 4.

\[\frac{\sum\nolimits_{n=1}^{NR} NOF_{n}}{NKO * NR} = PR\] \newline
In this measure \(NOF\) denotes the number of found optima per run and
\(NKO\) the number of known optima for the function. \newline

\begin{longtable}[c]{@{}crrrrrr@{}}
\caption{Peak Ratio over given runs (Share of found global optima over
all runs)}\tabularnewline
\toprule
\begin{minipage}[b]{0.11\columnwidth}\centering\strut
~
\strut\end{minipage} &
\begin{minipage}[b]{0.07\columnwidth}\raggedleft\strut
0.1
\strut\end{minipage} &
\begin{minipage}[b]{0.08\columnwidth}\raggedleft\strut
0.01
\strut\end{minipage} &
\begin{minipage}[b]{0.09\columnwidth}\raggedleft\strut
0.001
\strut\end{minipage} &
\begin{minipage}[b]{0.10\columnwidth}\raggedleft\strut
0.0001
\strut\end{minipage} &
\begin{minipage}[b]{0.11\columnwidth}\raggedleft\strut
0.00001
\strut\end{minipage} &
\begin{minipage}[b]{0.07\columnwidth}\raggedleft\strut
runs
\strut\end{minipage}\tabularnewline
\midrule
\endfirsthead
\toprule
\begin{minipage}[b]{0.11\columnwidth}\centering\strut
~
\strut\end{minipage} &
\begin{minipage}[b]{0.07\columnwidth}\raggedleft\strut
0.1
\strut\end{minipage} &
\begin{minipage}[b]{0.08\columnwidth}\raggedleft\strut
0.01
\strut\end{minipage} &
\begin{minipage}[b]{0.09\columnwidth}\raggedleft\strut
0.001
\strut\end{minipage} &
\begin{minipage}[b]{0.10\columnwidth}\raggedleft\strut
0.0001
\strut\end{minipage} &
\begin{minipage}[b]{0.11\columnwidth}\raggedleft\strut
0.00001
\strut\end{minipage} &
\begin{minipage}[b]{0.07\columnwidth}\raggedleft\strut
runs
\strut\end{minipage}\tabularnewline
\midrule
\endhead
\begin{minipage}[t]{0.11\columnwidth}\centering\strut
\textbf{F1}
\strut\end{minipage} &
\begin{minipage}[t]{0.07\columnwidth}\raggedleft\strut
1
\strut\end{minipage} &
\begin{minipage}[t]{0.08\columnwidth}\raggedleft\strut
1
\strut\end{minipage} &
\begin{minipage}[t]{0.09\columnwidth}\raggedleft\strut
1
\strut\end{minipage} &
\begin{minipage}[t]{0.10\columnwidth}\raggedleft\strut
1
\strut\end{minipage} &
\begin{minipage}[t]{0.11\columnwidth}\raggedleft\strut
1
\strut\end{minipage} &
\begin{minipage}[t]{0.07\columnwidth}\raggedleft\strut
45
\strut\end{minipage}\tabularnewline
\begin{minipage}[t]{0.11\columnwidth}\centering\strut
\textbf{F2}
\strut\end{minipage} &
\begin{minipage}[t]{0.07\columnwidth}\raggedleft\strut
1
\strut\end{minipage} &
\begin{minipage}[t]{0.08\columnwidth}\raggedleft\strut
1
\strut\end{minipage} &
\begin{minipage}[t]{0.09\columnwidth}\raggedleft\strut
1
\strut\end{minipage} &
\begin{minipage}[t]{0.10\columnwidth}\raggedleft\strut
1
\strut\end{minipage} &
\begin{minipage}[t]{0.11\columnwidth}\raggedleft\strut
1
\strut\end{minipage} &
\begin{minipage}[t]{0.07\columnwidth}\raggedleft\strut
42
\strut\end{minipage}\tabularnewline
\begin{minipage}[t]{0.11\columnwidth}\centering\strut
\textbf{F3}
\strut\end{minipage} &
\begin{minipage}[t]{0.07\columnwidth}\raggedleft\strut
1
\strut\end{minipage} &
\begin{minipage}[t]{0.08\columnwidth}\raggedleft\strut
1
\strut\end{minipage} &
\begin{minipage}[t]{0.09\columnwidth}\raggedleft\strut
1
\strut\end{minipage} &
\begin{minipage}[t]{0.10\columnwidth}\raggedleft\strut
1
\strut\end{minipage} &
\begin{minipage}[t]{0.11\columnwidth}\raggedleft\strut
1
\strut\end{minipage} &
\begin{minipage}[t]{0.07\columnwidth}\raggedleft\strut
44
\strut\end{minipage}\tabularnewline
\begin{minipage}[t]{0.11\columnwidth}\centering\strut
\textbf{F4}
\strut\end{minipage} &
\begin{minipage}[t]{0.07\columnwidth}\raggedleft\strut
1
\strut\end{minipage} &
\begin{minipage}[t]{0.08\columnwidth}\raggedleft\strut
1
\strut\end{minipage} &
\begin{minipage}[t]{0.09\columnwidth}\raggedleft\strut
1
\strut\end{minipage} &
\begin{minipage}[t]{0.10\columnwidth}\raggedleft\strut
1
\strut\end{minipage} &
\begin{minipage}[t]{0.11\columnwidth}\raggedleft\strut
1
\strut\end{minipage} &
\begin{minipage}[t]{0.07\columnwidth}\raggedleft\strut
44
\strut\end{minipage}\tabularnewline
\begin{minipage}[t]{0.11\columnwidth}\centering\strut
\textbf{F5}
\strut\end{minipage} &
\begin{minipage}[t]{0.07\columnwidth}\raggedleft\strut
1
\strut\end{minipage} &
\begin{minipage}[t]{0.08\columnwidth}\raggedleft\strut
1
\strut\end{minipage} &
\begin{minipage}[t]{0.09\columnwidth}\raggedleft\strut
1
\strut\end{minipage} &
\begin{minipage}[t]{0.10\columnwidth}\raggedleft\strut
1
\strut\end{minipage} &
\begin{minipage}[t]{0.11\columnwidth}\raggedleft\strut
1
\strut\end{minipage} &
\begin{minipage}[t]{0.07\columnwidth}\raggedleft\strut
41
\strut\end{minipage}\tabularnewline
\begin{minipage}[t]{0.11\columnwidth}\centering\strut
\textbf{F6}
\strut\end{minipage} &
\begin{minipage}[t]{0.07\columnwidth}\raggedleft\strut
1
\strut\end{minipage} &
\begin{minipage}[t]{0.08\columnwidth}\raggedleft\strut
1
\strut\end{minipage} &
\begin{minipage}[t]{0.09\columnwidth}\raggedleft\strut
1
\strut\end{minipage} &
\begin{minipage}[t]{0.10\columnwidth}\raggedleft\strut
1
\strut\end{minipage} &
\begin{minipage}[t]{0.11\columnwidth}\raggedleft\strut
0
\strut\end{minipage} &
\begin{minipage}[t]{0.07\columnwidth}\raggedleft\strut
40
\strut\end{minipage}\tabularnewline
\begin{minipage}[t]{0.11\columnwidth}\centering\strut
\textbf{F7}
\strut\end{minipage} &
\begin{minipage}[t]{0.07\columnwidth}\raggedleft\strut
1
\strut\end{minipage} &
\begin{minipage}[t]{0.08\columnwidth}\raggedleft\strut
1
\strut\end{minipage} &
\begin{minipage}[t]{0.09\columnwidth}\raggedleft\strut
1
\strut\end{minipage} &
\begin{minipage}[t]{0.10\columnwidth}\raggedleft\strut
1
\strut\end{minipage} &
\begin{minipage}[t]{0.11\columnwidth}\raggedleft\strut
1
\strut\end{minipage} &
\begin{minipage}[t]{0.07\columnwidth}\raggedleft\strut
41
\strut\end{minipage}\tabularnewline
\begin{minipage}[t]{0.11\columnwidth}\centering\strut
\textbf{F8}
\strut\end{minipage} &
\begin{minipage}[t]{0.07\columnwidth}\raggedleft\strut
1
\strut\end{minipage} &
\begin{minipage}[t]{0.08\columnwidth}\raggedleft\strut
1
\strut\end{minipage} &
\begin{minipage}[t]{0.09\columnwidth}\raggedleft\strut
1
\strut\end{minipage} &
\begin{minipage}[t]{0.10\columnwidth}\raggedleft\strut
0.99
\strut\end{minipage} &
\begin{minipage}[t]{0.11\columnwidth}\raggedleft\strut
0.97
\strut\end{minipage} &
\begin{minipage}[t]{0.07\columnwidth}\raggedleft\strut
23
\strut\end{minipage}\tabularnewline
\begin{minipage}[t]{0.11\columnwidth}\centering\strut
\textbf{F9}
\strut\end{minipage} &
\begin{minipage}[t]{0.07\columnwidth}\raggedleft\strut
1
\strut\end{minipage} &
\begin{minipage}[t]{0.08\columnwidth}\raggedleft\strut
1
\strut\end{minipage} &
\begin{minipage}[t]{0.09\columnwidth}\raggedleft\strut
1
\strut\end{minipage} &
\begin{minipage}[t]{0.10\columnwidth}\raggedleft\strut
1
\strut\end{minipage} &
\begin{minipage}[t]{0.11\columnwidth}\raggedleft\strut
1
\strut\end{minipage} &
\begin{minipage}[t]{0.07\columnwidth}\raggedleft\strut
27
\strut\end{minipage}\tabularnewline
\begin{minipage}[t]{0.11\columnwidth}\centering\strut
\textbf{F10}
\strut\end{minipage} &
\begin{minipage}[t]{0.07\columnwidth}\raggedleft\strut
1
\strut\end{minipage} &
\begin{minipage}[t]{0.08\columnwidth}\raggedleft\strut
1
\strut\end{minipage} &
\begin{minipage}[t]{0.09\columnwidth}\raggedleft\strut
1
\strut\end{minipage} &
\begin{minipage}[t]{0.10\columnwidth}\raggedleft\strut
1
\strut\end{minipage} &
\begin{minipage}[t]{0.11\columnwidth}\raggedleft\strut
1
\strut\end{minipage} &
\begin{minipage}[t]{0.07\columnwidth}\raggedleft\strut
41
\strut\end{minipage}\tabularnewline
\begin{minipage}[t]{0.11\columnwidth}\centering\strut
\textbf{F11}
\strut\end{minipage} &
\begin{minipage}[t]{0.07\columnwidth}\raggedleft\strut
1
\strut\end{minipage} &
\begin{minipage}[t]{0.08\columnwidth}\raggedleft\strut
1
\strut\end{minipage} &
\begin{minipage}[t]{0.09\columnwidth}\raggedleft\strut
1
\strut\end{minipage} &
\begin{minipage}[t]{0.10\columnwidth}\raggedleft\strut
1
\strut\end{minipage} &
\begin{minipage}[t]{0.11\columnwidth}\raggedleft\strut
1
\strut\end{minipage} &
\begin{minipage}[t]{0.07\columnwidth}\raggedleft\strut
40
\strut\end{minipage}\tabularnewline
\begin{minipage}[t]{0.11\columnwidth}\centering\strut
\textbf{F12}
\strut\end{minipage} &
\begin{minipage}[t]{0.07\columnwidth}\raggedleft\strut
1
\strut\end{minipage} &
\begin{minipage}[t]{0.08\columnwidth}\raggedleft\strut
1
\strut\end{minipage} &
\begin{minipage}[t]{0.09\columnwidth}\raggedleft\strut
1
\strut\end{minipage} &
\begin{minipage}[t]{0.10\columnwidth}\raggedleft\strut
1
\strut\end{minipage} &
\begin{minipage}[t]{0.11\columnwidth}\raggedleft\strut
1
\strut\end{minipage} &
\begin{minipage}[t]{0.07\columnwidth}\raggedleft\strut
41
\strut\end{minipage}\tabularnewline
\begin{minipage}[t]{0.11\columnwidth}\centering\strut
\textbf{F13}
\strut\end{minipage} &
\begin{minipage}[t]{0.07\columnwidth}\raggedleft\strut
1
\strut\end{minipage} &
\begin{minipage}[t]{0.08\columnwidth}\raggedleft\strut
1
\strut\end{minipage} &
\begin{minipage}[t]{0.09\columnwidth}\raggedleft\strut
1
\strut\end{minipage} &
\begin{minipage}[t]{0.10\columnwidth}\raggedleft\strut
1
\strut\end{minipage} &
\begin{minipage}[t]{0.11\columnwidth}\raggedleft\strut
1
\strut\end{minipage} &
\begin{minipage}[t]{0.07\columnwidth}\raggedleft\strut
41
\strut\end{minipage}\tabularnewline
\begin{minipage}[t]{0.11\columnwidth}\centering\strut
\textbf{F14}
\strut\end{minipage} &
\begin{minipage}[t]{0.07\columnwidth}\raggedleft\strut
1
\strut\end{minipage} &
\begin{minipage}[t]{0.08\columnwidth}\raggedleft\strut
1
\strut\end{minipage} &
\begin{minipage}[t]{0.09\columnwidth}\raggedleft\strut
1
\strut\end{minipage} &
\begin{minipage}[t]{0.10\columnwidth}\raggedleft\strut
1
\strut\end{minipage} &
\begin{minipage}[t]{0.11\columnwidth}\raggedleft\strut
1
\strut\end{minipage} &
\begin{minipage}[t]{0.07\columnwidth}\raggedleft\strut
39
\strut\end{minipage}\tabularnewline
\begin{minipage}[t]{0.11\columnwidth}\centering\strut
\textbf{F15}
\strut\end{minipage} &
\begin{minipage}[t]{0.07\columnwidth}\raggedleft\strut
1
\strut\end{minipage} &
\begin{minipage}[t]{0.08\columnwidth}\raggedleft\strut
1
\strut\end{minipage} &
\begin{minipage}[t]{0.09\columnwidth}\raggedleft\strut
1
\strut\end{minipage} &
\begin{minipage}[t]{0.10\columnwidth}\raggedleft\strut
1
\strut\end{minipage} &
\begin{minipage}[t]{0.11\columnwidth}\raggedleft\strut
1
\strut\end{minipage} &
\begin{minipage}[t]{0.07\columnwidth}\raggedleft\strut
31
\strut\end{minipage}\tabularnewline
\begin{minipage}[t]{0.11\columnwidth}\centering\strut
\textbf{F16}
\strut\end{minipage} &
\begin{minipage}[t]{0.07\columnwidth}\raggedleft\strut
0.01
\strut\end{minipage} &
\begin{minipage}[t]{0.08\columnwidth}\raggedleft\strut
0
\strut\end{minipage} &
\begin{minipage}[t]{0.09\columnwidth}\raggedleft\strut
0
\strut\end{minipage} &
\begin{minipage}[t]{0.10\columnwidth}\raggedleft\strut
0
\strut\end{minipage} &
\begin{minipage}[t]{0.11\columnwidth}\raggedleft\strut
0
\strut\end{minipage} &
\begin{minipage}[t]{0.07\columnwidth}\raggedleft\strut
20
\strut\end{minipage}\tabularnewline
\begin{minipage}[t]{0.11\columnwidth}\centering\strut
\textbf{F17}
\strut\end{minipage} &
\begin{minipage}[t]{0.07\columnwidth}\raggedleft\strut
0.78
\strut\end{minipage} &
\begin{minipage}[t]{0.08\columnwidth}\raggedleft\strut
0.75
\strut\end{minipage} &
\begin{minipage}[t]{0.09\columnwidth}\raggedleft\strut
0.73
\strut\end{minipage} &
\begin{minipage}[t]{0.10\columnwidth}\raggedleft\strut
0.72
\strut\end{minipage} &
\begin{minipage}[t]{0.11\columnwidth}\raggedleft\strut
0.69
\strut\end{minipage} &
\begin{minipage}[t]{0.07\columnwidth}\raggedleft\strut
19
\strut\end{minipage}\tabularnewline
\begin{minipage}[t]{0.11\columnwidth}\centering\strut
\textbf{F18}
\strut\end{minipage} &
\begin{minipage}[t]{0.07\columnwidth}\raggedleft\strut
0.84
\strut\end{minipage} &
\begin{minipage}[t]{0.08\columnwidth}\raggedleft\strut
0.83
\strut\end{minipage} &
\begin{minipage}[t]{0.09\columnwidth}\raggedleft\strut
0.83
\strut\end{minipage} &
\begin{minipage}[t]{0.10\columnwidth}\raggedleft\strut
0.79
\strut\end{minipage} &
\begin{minipage}[t]{0.11\columnwidth}\raggedleft\strut
0.79
\strut\end{minipage} &
\begin{minipage}[t]{0.07\columnwidth}\raggedleft\strut
23
\strut\end{minipage}\tabularnewline
\begin{minipage}[t]{0.11\columnwidth}\centering\strut
\textbf{F19}
\strut\end{minipage} &
\begin{minipage}[t]{0.07\columnwidth}\raggedleft\strut
0.34
\strut\end{minipage} &
\begin{minipage}[t]{0.08\columnwidth}\raggedleft\strut
0.33
\strut\end{minipage} &
\begin{minipage}[t]{0.09\columnwidth}\raggedleft\strut
0.32
\strut\end{minipage} &
\begin{minipage}[t]{0.10\columnwidth}\raggedleft\strut
0.32
\strut\end{minipage} &
\begin{minipage}[t]{0.11\columnwidth}\raggedleft\strut
0.31
\strut\end{minipage} &
\begin{minipage}[t]{0.07\columnwidth}\raggedleft\strut
18
\strut\end{minipage}\tabularnewline
\begin{minipage}[t]{0.11\columnwidth}\centering\strut
\textbf{F20}
\strut\end{minipage} &
\begin{minipage}[t]{0.07\columnwidth}\raggedleft\strut
0.15
\strut\end{minipage} &
\begin{minipage}[t]{0.08\columnwidth}\raggedleft\strut
0.15
\strut\end{minipage} &
\begin{minipage}[t]{0.09\columnwidth}\raggedleft\strut
0.15
\strut\end{minipage} &
\begin{minipage}[t]{0.10\columnwidth}\raggedleft\strut
0.15
\strut\end{minipage} &
\begin{minipage}[t]{0.11\columnwidth}\raggedleft\strut
0.13
\strut\end{minipage} &
\begin{minipage}[t]{0.07\columnwidth}\raggedleft\strut
17
\strut\end{minipage}\tabularnewline
\bottomrule
\end{longtable}

As a fourth measure, which wasn't introduced by the CEC committee, but
used in the original nmmso implementation (Fieldsend 2014), the Number
of Swarms was chosen. Since this is a continuous measure and, therefore
no calculation is needed, this measure is pictured as graphs. The graphs
can be found in Figure 1. They show the development of \(number\) \(of\)
\(swarms\) kept by nmmso.R over all iterations. Important to notice here
is that \(iterations\) is different from the \(evaluations\) referenced
in the other measures. Iterations are calls to start single runs of
nmmso.R and, therefore, are different from the evaluations taken within
the program.

Additionally, a fifth measure was introduced which denotes the runtime
of nmmso.R for the single functions. These times were taken on the
ZIVHPC, a scientific High Perfomance Computing Cluster by Westfälische
Wilhelms-Universität Münster. Since the nmmso.R is a strictly sequential
algorithm the runtimes for single runs will be comparable on common
computers. The ZIVHPC was only used to parallelise the single runs.

\begin{figure}[htbp]
\centering
\includegraphics{figure/trend\%20curve\%20of\%20kept\%20swarms\%20over\%20all\%2020\%20functions.\%20The\%20red\%20curves\%20show\%20the\%20number\%20of\%20swarms\%20kept\%20for\%20each\%20single\%20run.\%20The\%20black\%20line\%20shows\%20the\%20mean\%20of\%20kept\%20swarms\%20over\%20these\%20runs.-1.pdf}
\caption{plot of chunk trend curve of kept swarms over all 20 functions.
The red curves show the number of swarms kept for each single run. The
black line shows the mean of kept swarms over these runs.}
\end{figure}

When comparing those measures with the ones given in the original paper
(Fieldsend 2014), it can be seen that the reimplementation nmmso.R is an
overall good resemblance of the original algorithm. The three CEC
measures are close to the original taken measures and the trend curves
for the number of kept swarms have similar trends.

The biggest differences between the benchmarking results of the two
implementations can be seen in the general results of function 14, 15,
16 and 18, as well as in the number of created swarms for the
n-dimensional functions:

\begin{enumerate}
\def\labelenumi{(\arabic{enumi})}
\item
  Function 14 and 15 have a \(Success\ Ratio\) of \(1\) aswell as
  \(Peak\) \(Ratio\) of one \(1\) for all accuracy levels. Additionally
  nmmso.R sometimes found all global optima for Function 18. In
  contradiction of all three function almost never result in the finding
  of global optima in the evaluation of the original implementation.
  Only at the lowest accuracy, the original implementation is able to
  find all global optima for Function 14 (Fieldsend 2014, p. 16). It is
  hard to say if this difference is equal to an error in the
  implementation of nmmso.R or if an error in the original
  implementation was fixed. Also, this could be a difference in the
  reimplementation of the CEC Benchmarking Tool. Nevertheless, this is
  an interesting point of discussion and worth evaluating.
\item
  nmmso.R performs noticeable worse for Function 16 than for the
  original function. While nmmso.R has a \(Peak\) \(Ratio\) of \(0.01\)
  for an accuracy of \(0.1\) and \(0\) for all others, the original
  implementation reaches a \(Peak\) \(Ratio\) of around \(0.6\) for all
  accuracies. This might be an implementation error in the CEC
  Benchmarking Tool and not in nmmso.R since it is so significantly
  worse that it is unlikely that this difference would only occur in one
  test function.
\item
  Almost all algorithm runs on high-dimensional functions (F12-F20)
  result in a high number of swarms while all other results regarding
  this functions are comparable to the original results. This difference
  becomes very clear in the case of Functions 17-20. In the paper
  addressing the original implementation the x-axis rank from 0-40,000
  iterations while for the reimplementation limit of 4,000 for Function
  17, of 20,000 for Function 18, 6,000 for Function 19 and of 30,000 for
  Function 20 is enough to show all data sets. This is connected to the
  creation of much more swarms, which leads to an earlier depletion of
  the maximum allowed number of evaluations.
\end{enumerate}

Additionally, the time of all algorithms was taken. Even though this
measure widely varies depending on the computer's architecture it can
show the different complexity of all 20 functions.

\begin{longtable}[c]{@{}crr@{}}
\caption{Taken time of nmmso.R for all 20 functions. All times are in
seconds.}\tabularnewline
\toprule
\begin{minipage}[b]{0.13\columnwidth}\centering\strut
~
\strut\end{minipage} &
\begin{minipage}[b]{0.09\columnwidth}\raggedleft\strut
Mean
\strut\end{minipage} &
\begin{minipage}[b]{0.25\columnwidth}\raggedleft\strut
Standard Deviation
\strut\end{minipage}\tabularnewline
\midrule
\endfirsthead
\toprule
\begin{minipage}[b]{0.13\columnwidth}\centering\strut
~
\strut\end{minipage} &
\begin{minipage}[b]{0.09\columnwidth}\raggedleft\strut
Mean
\strut\end{minipage} &
\begin{minipage}[b]{0.25\columnwidth}\raggedleft\strut
Standard Deviation
\strut\end{minipage}\tabularnewline
\midrule
\endhead
\begin{minipage}[t]{0.13\columnwidth}\centering\strut
\textbf{F1}
\strut\end{minipage} &
\begin{minipage}[t]{0.09\columnwidth}\raggedleft\strut
13
\strut\end{minipage} &
\begin{minipage}[t]{0.25\columnwidth}\raggedleft\strut
3.8
\strut\end{minipage}\tabularnewline
\begin{minipage}[t]{0.13\columnwidth}\centering\strut
\textbf{F2}
\strut\end{minipage} &
\begin{minipage}[t]{0.09\columnwidth}\raggedleft\strut
8.6
\strut\end{minipage} &
\begin{minipage}[t]{0.25\columnwidth}\raggedleft\strut
6.5
\strut\end{minipage}\tabularnewline
\begin{minipage}[t]{0.13\columnwidth}\centering\strut
\textbf{F3}
\strut\end{minipage} &
\begin{minipage}[t]{0.09\columnwidth}\raggedleft\strut
4.8
\strut\end{minipage} &
\begin{minipage}[t]{0.25\columnwidth}\raggedleft\strut
2.6
\strut\end{minipage}\tabularnewline
\begin{minipage}[t]{0.13\columnwidth}\centering\strut
\textbf{F4}
\strut\end{minipage} &
\begin{minipage}[t]{0.09\columnwidth}\raggedleft\strut
27
\strut\end{minipage} &
\begin{minipage}[t]{0.25\columnwidth}\raggedleft\strut
6.1
\strut\end{minipage}\tabularnewline
\begin{minipage}[t]{0.13\columnwidth}\centering\strut
\textbf{F5}
\strut\end{minipage} &
\begin{minipage}[t]{0.09\columnwidth}\raggedleft\strut
10
\strut\end{minipage} &
\begin{minipage}[t]{0.25\columnwidth}\raggedleft\strut
4.4
\strut\end{minipage}\tabularnewline
\begin{minipage}[t]{0.13\columnwidth}\centering\strut
\textbf{F6}
\strut\end{minipage} &
\begin{minipage}[t]{0.09\columnwidth}\raggedleft\strut
7341
\strut\end{minipage} &
\begin{minipage}[t]{0.25\columnwidth}\raggedleft\strut
807
\strut\end{minipage}\tabularnewline
\begin{minipage}[t]{0.13\columnwidth}\centering\strut
\textbf{F7}
\strut\end{minipage} &
\begin{minipage}[t]{0.09\columnwidth}\raggedleft\strut
740
\strut\end{minipage} &
\begin{minipage}[t]{0.25\columnwidth}\raggedleft\strut
350
\strut\end{minipage}\tabularnewline
\begin{minipage}[t]{0.13\columnwidth}\centering\strut
\textbf{F8}
\strut\end{minipage} &
\begin{minipage}[t]{0.09\columnwidth}\raggedleft\strut
32089
\strut\end{minipage} &
\begin{minipage}[t]{0.25\columnwidth}\raggedleft\strut
5379
\strut\end{minipage}\tabularnewline
\begin{minipage}[t]{0.13\columnwidth}\centering\strut
\textbf{F9}
\strut\end{minipage} &
\begin{minipage}[t]{0.09\columnwidth}\raggedleft\strut
23513
\strut\end{minipage} &
\begin{minipage}[t]{0.25\columnwidth}\raggedleft\strut
4169
\strut\end{minipage}\tabularnewline
\begin{minipage}[t]{0.13\columnwidth}\centering\strut
\textbf{F10}
\strut\end{minipage} &
\begin{minipage}[t]{0.09\columnwidth}\raggedleft\strut
123
\strut\end{minipage} &
\begin{minipage}[t]{0.25\columnwidth}\raggedleft\strut
26
\strut\end{minipage}\tabularnewline
\begin{minipage}[t]{0.13\columnwidth}\centering\strut
\textbf{F11}
\strut\end{minipage} &
\begin{minipage}[t]{0.09\columnwidth}\raggedleft\strut
660
\strut\end{minipage} &
\begin{minipage}[t]{0.25\columnwidth}\raggedleft\strut
439
\strut\end{minipage}\tabularnewline
\begin{minipage}[t]{0.13\columnwidth}\centering\strut
\textbf{F12}
\strut\end{minipage} &
\begin{minipage}[t]{0.09\columnwidth}\raggedleft\strut
2182
\strut\end{minipage} &
\begin{minipage}[t]{0.25\columnwidth}\raggedleft\strut
517
\strut\end{minipage}\tabularnewline
\begin{minipage}[t]{0.13\columnwidth}\centering\strut
\textbf{F13}
\strut\end{minipage} &
\begin{minipage}[t]{0.09\columnwidth}\raggedleft\strut
1746
\strut\end{minipage} &
\begin{minipage}[t]{0.25\columnwidth}\raggedleft\strut
1359
\strut\end{minipage}\tabularnewline
\begin{minipage}[t]{0.13\columnwidth}\centering\strut
\textbf{F14}
\strut\end{minipage} &
\begin{minipage}[t]{0.09\columnwidth}\raggedleft\strut
4519
\strut\end{minipage} &
\begin{minipage}[t]{0.25\columnwidth}\raggedleft\strut
1940
\strut\end{minipage}\tabularnewline
\begin{minipage}[t]{0.13\columnwidth}\centering\strut
\textbf{F15}
\strut\end{minipage} &
\begin{minipage}[t]{0.09\columnwidth}\raggedleft\strut
12282
\strut\end{minipage} &
\begin{minipage}[t]{0.25\columnwidth}\raggedleft\strut
7859
\strut\end{minipage}\tabularnewline
\begin{minipage}[t]{0.13\columnwidth}\centering\strut
\textbf{F16}
\strut\end{minipage} &
\begin{minipage}[t]{0.09\columnwidth}\raggedleft\strut
40958
\strut\end{minipage} &
\begin{minipage}[t]{0.25\columnwidth}\raggedleft\strut
1837
\strut\end{minipage}\tabularnewline
\begin{minipage}[t]{0.13\columnwidth}\centering\strut
\textbf{F17}
\strut\end{minipage} &
\begin{minipage}[t]{0.09\columnwidth}\raggedleft\strut
33240
\strut\end{minipage} &
\begin{minipage}[t]{0.25\columnwidth}\raggedleft\strut
12200
\strut\end{minipage}\tabularnewline
\begin{minipage}[t]{0.13\columnwidth}\centering\strut
\textbf{F18}
\strut\end{minipage} &
\begin{minipage}[t]{0.09\columnwidth}\raggedleft\strut
61631
\strut\end{minipage} &
\begin{minipage}[t]{0.25\columnwidth}\raggedleft\strut
27491
\strut\end{minipage}\tabularnewline
\begin{minipage}[t]{0.13\columnwidth}\centering\strut
\textbf{F19}
\strut\end{minipage} &
\begin{minipage}[t]{0.09\columnwidth}\raggedleft\strut
63224
\strut\end{minipage} &
\begin{minipage}[t]{0.25\columnwidth}\raggedleft\strut
15799
\strut\end{minipage}\tabularnewline
\begin{minipage}[t]{0.13\columnwidth}\centering\strut
\textbf{F20}
\strut\end{minipage} &
\begin{minipage}[t]{0.09\columnwidth}\raggedleft\strut
97631
\strut\end{minipage} &
\begin{minipage}[t]{0.25\columnwidth}\raggedleft\strut
14844
\strut\end{minipage}\tabularnewline
\bottomrule
\end{longtable}

\section{Conclusion}\label{conclusion}

In this project a reimplementation of the NMMSO algorithm in R was
shown. As part of this project also a reimplementation of the CEC
Benchmarking Suite was created. It was shown that it is possible to
translate a Matlab program into R

\section{Acknowledgements}\label{acknowledgements}

We want to thank Dr.~Jonathan Fieldsend for his continuous help via mail
during this seminar. Also, the committee of the CEC was always available
for questions and concerns during our work. Furthermore, a special
thanks go to all employees of the chair for `Information Systems and
Statistics' including Dr.~Mike Preuß, Jakob Bossek and Pascal Kerschke
who were available for any questions regarding the implementation and
this report at all times. \newpage

\hypertarget{refs}{}
\hypertarget{ref-epitropakisux5f2013}{}
Epitropakis, M. G., Li, X., and Burke, E. K. 2013. ``A dynamic archive
niching differential evolution algorithm for multimodal optimization,''
in \emph{Evolutionary computation (cEC), 2013 iEEE congress on}, IEEE,
pp. 79--86.

\hypertarget{ref-fieldsendux5f2014}{}
Fieldsend, J. E. 2014. ``Running up those hills: Multi-modal search with
the niching migratory multi-swarm optimiser,'' in \emph{Evolutionary
computation (cEC), 2014 iEEE congress on}, IEEE, pp. 2593--2600.

\hypertarget{ref-liux5f2013}{}
Li, X., Engelbrecht, A., and Epitropakis, M. G. 2013. ``Benchmark
functions for cEC'2013 special session and competition on niching
methods for multimodal function optimization,'' \emph{RMIT University,
Evolutionary Computation and Machine Learning Group, Australia, Tech.
Rep}.

\hypertarget{ref-preussux5f2012}{}
Preuss, M. 2012. ``Improved topological niching for real-valued global
optimization,'' in \emph{Applications of evolutionary computation},
Springer, pp. 386--395.


\end{document}
% LaTeX Template for Project Report, Version 2.0
% (Abstracted from a Major Project Report at CSED, NIT Calicut but can be
% modified easily to use for other reports also.)
%
% Released under Creative Commons Attribution license (CC-BY)
% Info: http://creativecommons.org/licenses/by/3.0/
%
% Created by: Kartik Singhal
% BTech CSE Batch of 2009-13
% NIT Calicut
% Contact Info: kartiksinghal@gmail.com
%
% It is advisable to learn the basics of LaTeX before using this template.
% A good resource to start with is http://en.wikibooks.org/wiki/LaTeX/
%
% All template fields are marked with a pair of angular brackets e.g. <title here>
% except for the ones defining citation names in ref.tex.
%
% Empty space after chapter/section/subsection titles can be used to insert text.
%
% Just compile this file using pdflatex after making all required changes.

\documentclass[12pt,a4paper]{article}

\RequirePackage{fancyhdr}
\RequirePackage{lastpage}

\setlength\textwidth{165mm}
\setlength\textheight{240mm}
\setlength\topmargin{-10mm}
\setlength\oddsidemargin{0mm}
\setlength\parindent{0pt}
\setlength\parskip{1.7\medskipamount}

\sloppy\pagestyle{fancy}


%% rest

\usepackage{url} %for proper url entries
\usepackage[numbers]{natbib}
% for nice tables
\usepackage{longtable}
\usepackage{booktabs}

\usepackage{amssymb,amsmath}
\usepackage{ifxetex,ifluatex}
\ifxetex
  \usepackage{fontspec,xltxtra,xunicode}
  \defaultfontfeatures{Mapping=tex-text,Scale=MatchLowercase}
\else
  \ifluatex
    \usepackage{fontspec}
    \defaultfontfeatures{Mapping=tex-text,Scale=MatchLowercase}
  \else
    \usepackage[utf8]{inputenc}
  \fi
\fi

\usepackage{color}
\usepackage{fancyvrb}
\DefineShortVerb[commandchars=\\\{\}]{\|}
\DefineVerbatimEnvironment{Highlighting}{Verbatim}{commandchars=\\\{\}}
% Add ',fontsize=\small' for more characters per line
\newenvironment{Shaded}{}{}
\newcommand{\KeywordTok}[1]{\textcolor[rgb]{0.00,0.44,0.13}{\textbf{{#1}}}}
\newcommand{\DataTypeTok}[1]{\textcolor[rgb]{0.56,0.13,0.00}{{#1}}}
\newcommand{\DecValTok}[1]{\textcolor[rgb]{0.25,0.63,0.44}{{#1}}}
\newcommand{\BaseNTok}[1]{\textcolor[rgb]{0.25,0.63,0.44}{{#1}}}
\newcommand{\FloatTok}[1]{\textcolor[rgb]{0.25,0.63,0.44}{{#1}}}
\newcommand{\CharTok}[1]{\textcolor[rgb]{0.25,0.44,0.63}{{#1}}}
\newcommand{\StringTok}[1]{\textcolor[rgb]{0.25,0.44,0.63}{{#1}}}
\newcommand{\CommentTok}[1]{\textcolor[rgb]{0.38,0.63,0.69}{\textit{{#1}}}}
\newcommand{\OtherTok}[1]{\textcolor[rgb]{0.00,0.44,0.13}{{#1}}}
\newcommand{\AlertTok}[1]{\textcolor[rgb]{1.00,0.00,0.00}{\textbf{{#1}}}}
\newcommand{\FunctionTok}[1]{\textcolor[rgb]{0.02,0.16,0.49}{{#1}}}
\newcommand{\RegionMarkerTok}[1]{{#1}}
\newcommand{\ErrorTok}[1]{\textcolor[rgb]{1.00,0.00,0.00}{\textbf{{#1}}}}
\newcommand{\NormalTok}[1]{{#1}}
\usepackage[pdftex]{graphicx}
\setkeys{Gin}{width=\textwidth}
\usepackage[Export]{adjustbox}
\usepackage[unicode=true]{hyperref}
\hypersetup{breaklinks=true, pdfborder={0 0 0}}
\setlength{\parindent}{0pt}
\setlength{\parskip}{6pt plus 2pt minus 1pt}
\setlength{\emergencystretch}{3em}  % prevent overfull lines
%\setcounter{secnumdepth}{0}


\begin{document}
\renewcommand\refname{References} %Renames "Bibliography" to "References" on ref page

%include other pages
\begin{titlepage}

\begin{center}

\textup{\small {\bf Statistical Computing in R} \\ Report}\\[0.2in]

% Title
\Large \textbf {Implementation of NMMSO in R}\\[0.5in]

 % Submitted by
\normalsize Submitted by \\
\begin{table}[h]
\centering
\begin{tabular}{lr}
425699 & Jannik Hoffjann \\
425699 & Daniel Carriola \\ 
\end{tabular}
\end{table}

\vspace{.1in}
Under the guidance of\\
{\textbf{Dr. Mike Preuss}}\\[0.2in]

\vfill

% Bottom of the page
\includegraphics[width=0.18\textwidth]{./assets/wwu-logo}\\[0.1in]
\Large{Information Systems and Statistics}\\
\normalsize
\textsc{ERCIS}\\
Münster - NRW - Germany \\
\vspace{0.2cm}
Winter Semester 2015/16

\end{center}

\end{titlepage}

\newpage
\pagenumbering{roman} %numbering before main content starts
\tableofcontents
\newpage
\listoffigures

\newpage
\pagenumbering{arabic} %reset numbering to normal for the main content

\section{Introduction}\label{introduction}

\begin{verbatim}
\end{verbatim}

In the recent years R has become the statistical programming language of
choice for many people. Since its introduction in 1992.

\begin{center}\rule{0.5\linewidth}{\linethickness}\end{center}

\section{The Algorithm}\label{the-algorithm}

\subsection{General Function}\label{general-function}

Starting point of the project was the paper provided by Jonathen E.
Fieldsend (Fieldsend 2014) on the Niching Migratoy Multi-Swarm Optimiser
(NMMSO) algorithm. NMMSO is a multi-modal optimiser which relies heavily
on multiple swarms which are generated on the landscape of an algorithm
in order to find the global optimum. It is build around three main
pillars: (1) dynamic in the numbers of dimensions, (2) self-adaptive
without any special preparation and (3) exploitative local search to
quickly find peak estimates (Fieldsend 2014, 1).

Multi-modal optimization in general is not that different from well
known and widely discussed single-objective optimisation, but in
difference to it the goal of the algorithms in the multi-modal is not to
find just one single optimizing point but all possible points (Fieldsend
2014, 1). In order to do so, many early multi-modal optimization
algorithms needed highly defined parameters {[}TODO: quote needed{]}.

Newer algorithms fall in the field of self-tuning and try to use
different mathematical paradigms like nearest-best clustering with
covariance matrices (Preuss 2010) and strategies like storing the so far
best found global optima estimators to provide them as parameters for
new optimization runs (Epitropakis, Li, and Burke 2013). Contradictory
to that NMMSO goes another way and uses the the swarm strategy in order
to find which store their current (Fieldsend 2014)

In order to do so NMMSO follow a strict structure which can be seen in
the following pseudo-code

\begin{verbatim}
nmmso(max_evals, tol, n, max_inc, c_1, c_2, chi, w)
    S: initialise_swarm(1)
    evaluations := 1
    while evaluations < max_evals:
        while flagged_swarms(S) == true:
            {S, m} := attempt_merge(S, n, tol)
            evals := evals + m
        S := increment(S, n, max_inc, c_1, c_2, chi, w)
        evals := evals + min(|S|, max_inc)
        {S, k} := attempt_separation(S, tol)
        evals := evals + k
        S := add_new_swarm(S)
        evals := evals + 1
    {X*, Y*} := extract_gebsest(S)
    return X*,Y*
\end{verbatim}

\subsection{CEC}\label{cec}

test

\begin{center}\rule{0.5\linewidth}{\linethickness}\end{center}

\section{The Implementation}\label{the-implementation}

\subsection{Structure of the project}\label{structure-of-the-project}

test {[}\#fieldsend\_2014{]}

\subsection{Pitfalls and Problems}\label{pitfalls-and-problems}

test

\subsection{Benchmark and Comparison}\label{benchmark-and-comparison}

\begin{longtable}[c]{@{}rrrrr@{}}
\toprule
0.1 & 0.01 & 0.001 & 0.0001 & 0.00001\tabularnewline
\midrule
\endhead
1 & 1 & 1 & 1 & 1\tabularnewline
1 & 1 & 1 & 1 & 1\tabularnewline
1 & 1 & 1 & 1 & 1\tabularnewline
1 & 1 & 1 & 1 & 1\tabularnewline
1 & 1 & 1 & 1 & 1\tabularnewline
6 & 6 & 6 & 6 & 6\tabularnewline
7 & 7 & 7 & 7 & 7\tabularnewline
8 & 8 & 8 & 8 & 8\tabularnewline
9 & 9 & 9 & 9 & 9\tabularnewline
10 & 10 & 10 & 10 & 10\tabularnewline
11 & 11 & 11 & 11 & 11\tabularnewline
12 & 12 & 12 & 12 & 12\tabularnewline
13 & 13 & 13 & 13 & 13\tabularnewline
14 & 14 & 14 & 14 & 14\tabularnewline
15 & 15 & 15 & 15 & 15\tabularnewline
16 & 16 & 16 & 16 & 16\tabularnewline
17 & 17 & 17 & 17 & 17\tabularnewline
18 & 18 & 18 & 18 & 18\tabularnewline
19 & 19 & 19 & 19 & 19\tabularnewline
20 & 20 & 20 & 20 & 20\tabularnewline
\bottomrule
\end{longtable}

\begin{verbatim}
## Warning in file(filename, "r", encoding = encoding): cannot open file 'R/
## cec_2015_problem_data.R': No such file or directory
\end{verbatim}

\begin{verbatim}
## Error in file(filename, "r", encoding = encoding): cannot open the connection
\end{verbatim}

\begin{figure}[htbp]
\centering
\includegraphics{figure/Results\%20of\%20Function\%205-1.png}
\caption{plot of chunk Results of Function 5}
\end{figure}

\subsection{Testing and alternative parameter
settings}\label{testing-and-alternative-parameter-settings}

test

\section{Discussion}\label{discussion}

test

\section{Conclusion}\label{conclusion}

test \newpage

\hypertarget{refs}{}
\hypertarget{ref-epitropakisux5f2013}{}
Epitropakis, Michael G, Xiaodong Li, and Edmund K Burke. 2013. ``A
dynamic archive niching differential evolution algorithm for multimodal
optimization.'' In \emph{Evolutionary computation (cEC), 2013 iEEE
congress on}, IEEE, p. 79--86.

\hypertarget{ref-fieldsendux5f2014}{}
Fieldsend, Jonathan E. 2014. ``Running up those hills: Multi-modal
search with the niching migratory multi-swarm optimiser.'' In
\emph{Evolutionary computation (cEC), 2014 iEEE congress on}, IEEE, p.
2593--2600.

\hypertarget{ref-preussux5f2010}{}
Preuss, Mike. 2010. ``Niching the cMA-eS via nearest-better
clustering.'' In \emph{Proceedings of the 12th annual conference
companion on genetic and evolutionary computation}, GECCO '10, New York,
NY, USA: ACM, p. 1711--1718.
\url{http://doi.acm.org/10.1145/1830761.1830793}.


\end{document}
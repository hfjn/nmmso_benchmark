% LaTeX Template for Project Report, Version 2.0
% (Abstracted from a Major Project Report at CSED, NIT Calicut but can be
% modified easily to use for other reports also.)
%
% Released under Creative Commons Attribution license (CC-BY)
% Info: http://creativecommons.org/licenses/by/3.0/
%
% Created by: Kartik Singhal
% BTech CSE Batch of 2009-13
% NIT Calicut
% Contact Info: kartiksinghal@gmail.com
%
% It is advisable to learn the basics of LaTeX before using this template.
% A good resource to start with is http://en.wikibooks.org/wiki/LaTeX/
%
% All template fields are marked with a pair of angular brackets e.g. <title here>
% except for the ones defining citation names in ref.tex.
%
% Empty space after chapter/section/subsection titles can be used to insert text.
%
% Just compile this file using pdflatex after making all required changes.

\documentclass[12pt,a4paper]{article}

\RequirePackage{fancyhdr}
\RequirePackage{lastpage}

\setlength\textwidth{165mm}
\setlength\textheight{240mm}
\setlength\topmargin{-10mm}
\setlength\oddsidemargin{0mm}
\setlength\parindent{0pt}
\setlength\parskip{1.7\medskipamount}

\sloppy\pagestyle{fancy}


%% rest

\usepackage{url} %for proper url entries
\usepackage[numbers]{natbib}
% for nice tables
\usepackage{longtable}
\usepackage{booktabs}
\usepackage{floatrow}
\floatsetup[table]{capposition=bottom}
% for nice code

\usepackage{amssymb,amsmath}
\usepackage{ifxetex,ifluatex}
\ifxetex
  \usepackage{fontspec,xltxtra,xunicode}
  \defaultfontfeatures{Mapping=tex-text,Scale=MatchLowercase}
\else
  \ifluatex
    \usepackage{fontspec}
    \defaultfontfeatures{Mapping=tex-text,Scale=MatchLowercase}
  \else
    \usepackage[utf8]{inputenc}
  \fi
\fi

\usepackage{color}
\usepackage{fancyvrb}
\DefineShortVerb[commandchars=\\\{\}]{\|}
\DefineVerbatimEnvironment{Highlighting}{Verbatim}{commandchars=\\\{\}}
% Add ',fontsize=\small' for more characters per line
\newenvironment{Shaded}{}{}
\newcommand{\KeywordTok}[1]{\textcolor[rgb]{0.00,0.44,0.13}{\textbf{{#1}}}}
\newcommand{\DataTypeTok}[1]{\textcolor[rgb]{0.56,0.13,0.00}{{#1}}}
\newcommand{\DecValTok}[1]{\textcolor[rgb]{0.25,0.63,0.44}{{#1}}}
\newcommand{\BaseNTok}[1]{\textcolor[rgb]{0.25,0.63,0.44}{{#1}}}
\newcommand{\FloatTok}[1]{\textcolor[rgb]{0.25,0.63,0.44}{{#1}}}
\newcommand{\CharTok}[1]{\textcolor[rgb]{0.25,0.44,0.63}{{#1}}}
\newcommand{\StringTok}[1]{\textcolor[rgb]{0.25,0.44,0.63}{{#1}}}
\newcommand{\CommentTok}[1]{\textcolor[rgb]{0.38,0.63,0.69}{\textit{{#1}}}}
\newcommand{\OtherTok}[1]{\textcolor[rgb]{0.00,0.44,0.13}{{#1}}}
\newcommand{\AlertTok}[1]{\textcolor[rgb]{1.00,0.00,0.00}{\textbf{{#1}}}}
\newcommand{\FunctionTok}[1]{\textcolor[rgb]{0.02,0.16,0.49}{{#1}}}
\newcommand{\RegionMarkerTok}[1]{{#1}}
\newcommand{\ErrorTok}[1]{\textcolor[rgb]{1.00,0.00,0.00}{\textbf{{#1}}}}
\newcommand{\NormalTok}[1]{{#1}}
\usepackage[pdftex]{graphicx}
\setkeys{Gin}{width=\textwidth}
\usepackage[Export]{adjustbox}
\usepackage[unicode=true]{hyperref}
\hypersetup{breaklinks=true, pdfborder={0 0 0}}
\setlength{\parindent}{0pt}
\setlength{\parskip}{6pt plus 2pt minus 1pt}
\setlength{\emergencystretch}{3em}  % prevent overfull lines
%\setcounter{secnumdepth}{0}


\begin{document}
\renewcommand\refname{References} %Renames "Bibliography" to "References" on ref page

%include other pages
\begin{titlepage}

\begin{center}

\textup{\small {\bf Statistical Computing in R} \\ Report}\\[0.2in]

% Title
\Large \textbf {Implementation of NMMSO in R}\\[0.5in]

 % Submitted by
\normalsize Submitted by \\
\begin{table}[h]
\centering
\begin{tabular}{lr}
425699 & Jannik Hoffjann \\
425699 & Daniel Carriola \\ 
\end{tabular}
\end{table}

\vspace{.1in}
Under the guidance of\\
{\textbf{Dr. Mike Preuss}}\\[0.2in]

\vfill

% Bottom of the page
\includegraphics[width=0.18\textwidth]{./assets/wwu-logo}\\[0.1in]
\Large{Information Systems and Statistics}\\
\normalsize
\textsc{ERCIS}\\
Münster - NRW - Germany \\
\vspace{0.2cm}
Winter Semester 2015/16

\end{center}

\end{titlepage}

\newpage
\pagenumbering{roman} %numbering before main content starts
\tableofcontents
\newpage

\pagenumbering{arabic} %reset numbering to normal for the main content

\section{Introduction}\label{introduction}

In the recent years R has become the statistical programming language of
choice for many scientist. The strength of R of being a domain specific
language has also become one of its weaknesses. Since new research
findings in statistical computing are split up over several languages
like R, Matlab or SciPy\footnote{SciPy is a common library for the
  Python Programming language which brings Statistical Computing
  capabilities to the language. \newpage} it often becomes difficult to
compare new methods with established ones. Since it is also hard to
interface those languages due to different architectures, data storage
mechanisms there is often no other way than to reimplement new methods
in a different programming language to create a common scope.

An example for a well perceived new finding in statistical computing is
the NMMSO-Algorithm by Jonathan E. Fieldsend (Fieldsend 2014). It won
the niching competition in 2015 held by the CEC and is only written in
Matlab. Since the chair `Information Systems and Statistics' at the
Westfälische Wilhelms-Universität Münster, Germany is mainly
concentrating its work on Statistical Computing in R an implementation
of this algorithm became interesting.

As part of this Seminar Project in the context of the Seminar
`Statistical Computing in R' a reimplementation of the NMMSO algorithm
in R (nmmso.R) will be presented. During this technical documentation,
the general function of the algorithm and the used test cases by the CEC
will be shown. Afterwards the structure and used techniques and
libraries, as well as problems and pitfalls due to the different
behaviours of R and Matlab, will be shown. The documentation will be
closed by the benchmarking results and different test cases.

It was the goal of this project to keep up high comparability with the
original code, to ensure the correct functionality and easily implement
changes to the original codebase in this program. To reach this, unit
tests were used where possible and continouous comparison between part
results of the original implementation and nmmso.R where used to ensure
functioning. Additionally a benchmarking suite, which builds on the CEC
Benchmarking Suite for Niching Algorithm was implemented to evaluate and
test the functioning of nmmso.R with the same characteristics as in the
original implementation.

\section{General Function}\label{general-function}

The starting point of the project was the paper provided by Dr.~Jonathen
E. Fieldsend (Fieldsend 2014) on the Niching Migratory Multi-Swarm
Optimiser (NMMSO) algorithm. NMMSO is a multi-modal optimiser which
relies heavily on multiple swarms which are generated on the landscape
of an function in order to find the global optima. It is build around
three main pillars: (1) dynamic in the numbers of dimensions, (2)
self-adaptive without any special preparation and (3) exploitative local
search to quickly find peak estimates (Fieldsend 2014, p. 1).

Multi-modal optimisation in general is not that different from well
known and widely discussed single-objective optimisation, but in
difference to it the goal of the algorithms in the multi-modal is not to
find just one single optimising point but all possible points (Fieldsend
2014, p. 1). To reach this goal many multi-modal optimization algorithms
use strategies oriented on the biological world and utilize swarm
intelligence to find optima defined by the search parameters (Yang
2009). In order to do so, many early multi-modal optimisation algorithms
needed defined parameters (Fieldsend 2014, p. 1).

Newer algorithms fall in the field of self-tuning and try to use
different mathematical paradigms like nearest-best clustering with
covariance matrices (Preuss 2012) and strategies like storing the so far
best found global optima estimators to provide them as parameters for
new optimisation runs (Epitropakis et al. 2013). Contradictory to that
NMMSO goes another way and uses the the swarm strategy in order to find
which store their current (Fieldsend 2014).

In order to do so NMMSO follow a strict structure which can be seen in
the following pseudo-code

\begin{verbatim}
nmmso(max_evals, tol, n, max_inc, c_1, c_2, omega)
    S: initialise_swarm(1)
    evaluations := 1
    while evaluations < max_evals:
        while flagged_swarms(S) == true:
            {S, m} := attempt_merge(S, n, tol)
            evals := evals + m
        S := increment(S, n, max_inc, c_1, c_2, omega)
        evals := evals + min(|S|, max_inc)
        {S, k} := attempt_separation(S, tol)
        evals := evals + k
        S := add_new_swarm(S)
        evals := evals + 1
    {X*, Y*} := extract_gbest(S)
    return X*,Y*
\end{verbatim}

This structure wasn't modified during the reimplementation of NMMSO to
keep comparability and the possibility to fix bugs at a high level. The
only newly introduced setting was the possibility to modify the c\_1,
c\_2, w as parameters from the outside. In the original version those
parameters are part of the program code.

\begin{longtable}[c]{@{}lll@{}}
\toprule
& standard value & used value\tabularnewline
\midrule
\endhead
evaluations & 0 & 0\tabularnewline
max\_evol & 100 & 100\tabularnewline
tol\_val & 10\^{}-6 & 10\^{}-6\tabularnewline
c\_1 & 2.0 & 2.0\tabularnewline
c\_2 & 2.0 & 2.0\tabularnewline
omega & 0.1 & 0.1\tabularnewline
\bottomrule
\end{longtable}

\begin{center}\rule{0.5\linewidth}{\linethickness}\end{center}

\section{CEC Algorithms}\label{cec-algorithms}

\subsection{CEC}\label{cec}

The IEEE Congress of Evolutionary Computation (CEC) is one of the
largest, most important and recognised conferences within Evolutionary
Computation (EC). It is organised by the IEEE Computational Intelligence
Society in cooperation with the Evolutionary Programming Society, and
covers most of the subtopics of the EC.

In order to validate the potential of the NMMSO algorithm, it was
submitted to the IEEE CEC 2013 held in Cancun, Mexico. Here,
Dr.~Fieldsend was provided with some multimodal benchmark test functions
with different dimension sizes and characteristics, for evaluating
niching algorithms developed by Dr.~Xiaodong Li, Dr.~Andries Engelbrecht
and Dr.~Michael G. Epitropakis (Epitropakis et al. 2013). They state
that even if several niching methods have been around for many years,
further advances in this area have been hindered by several obstacles;
most of the studies focus on very low dimensional multimodal problems (2
or 3 dimensions) making this more complicated to asses theses methods'
scalability to high dimensions with better performance. The benchmark
tool includes 20 test functions (in some cases the same function but
with different dimension sizes), which includes 10 simple, well-known
and widely used benchmark functions, based on recent studies, and more
complex functions following the paradigm of composition functions. In
the following section, they will be briefly explained:

\begin{verbatim}
•   F1: Five-Uneven-Peak Trap (1D)
•   F2: Equal Maxima (1D)
•   F3: Uneven Decreasing Maxima (1D)
•   F4: Himmelblau (2D)
•   F5: Six-Hump Camel Back (2D)
•   F6: Shubert (2D, 3D)
•   F7: Vincent (2D, 3D)
•   F8: Modified Rastrigin - All Global Optima (2D)
•   F9: Composition Function 1 (2D)
•   F10: Composition Function 2 (2D)
•   F11: Composition Function 3 (2D, 3D, 5D, 10D)
•   F12: Composition Function 4 (3D, 5D, 10D, 20D)
\end{verbatim}

All of the test functions are formulated as maximisation problems. F1,
F2 and F3 are simple 1D multimodal functions, while F4 and F5 are simple
2D functions and not scalable. F6 to F8 are scalable multimodal
functions. The number of global optima for F6 and F7 are determined by
the dimension. However, for F8, the number of global optima is
independent from the dimension, therefore it can be controlled by the
user. F9 to F12 are scalable multimodal functions constructed by several
basic functions with different properties (Sphere function, Grienwank,
Rastrigin, Weierstrass and the Expanded Griewank's plus Rosenbrock's
function). F9 and F10 are separable, and non-symmetric, while F11 and
F12 are non-separable, non-symmetric complex multimodal functions. The
number of global optima in all of the composition functions is
independent from the number of dimensions, therefore can be controlled
by the user (Epitropakis et al. 2013).

\textbf{Maybe write each math equation or the R code}

https://en.wikipedia.org/wiki/IEEE\_Congress\_on\_Evolutionary\_Computation

\subsection{Implementation and
Pitfalls}\label{implementation-and-pitfalls}

\textbf{write also about the count\_goptima and so on}

\begin{center}\rule{0.5\linewidth}{\linethickness}\end{center}

\section{The Implementation}\label{the-implementation}

\subsection{Structure of the project}\label{structure-of-the-project}

After analysing the algorithm provided in Matlab by Dr.~Fieldsend, it
was decided to first translate each of the functions into the R
programming language. At first instance, this task seemed to be simple
because most of the functions were basically managing matrices and
vectors, but later this became a problem that will be addressed in the
pitfalls' section of this paper.

Once all the NMMSO functions existed in R and having the input data, the
testing phase started. It has be said, that one of the biggest problems
when you code an already existing program into another programming
language, is the different behaviours corresponding to each object (in
case of an object-oriented language) or its main structure. The first
runs came with several errors regarding the matrix generation and
handling, slowing down the project in a near future. Using GitHub, it
was easier to attack these problems in parallel, having one developer
reviewing different functions and the other one, fixing other bugs and
continue the testing phase. Also, this was achieved in an easier way,
thanks that each function was coded in an independent R file, making
easier and faster the debugging and the fixing of each problem.

During the developing time, an issue raised with the CEC benchmark tool.
In order to compare the R implementation of the NMMSO algorithm with the
original one, it was mandatory to use this tool to test each of its
functions with the new algorithm and compare results. After several
complications with the original test suite (these complications will be
addressed in the pitfalls' section), it was decided to recode each of
the functions as an independent R package to avoid any further
complication and having an easier and more trustworthy comparison of the
NMMSO algorithm in R.

\subsection{Pitfalls and Problems}\label{pitfalls-and-problems}

test

\section{Benchmark and Comparison}\label{benchmark-and-comparison}

To compare nmmso.R with the original NMMSO the CEC test cases were used
to run the same benchmarks as in the original submission (Fieldsend
2014). There 4 different Ratios were used to measure the performance of
certain algorithms. Three of those measures (Peak Ratio, Success Ratio
and Convergence Speed) have been introduced in (Epitropakis et al. 2013,
pp. 6--7) to create a common point of comparison. The fourth ratio is
special for the nmmso algorithm since it tracks the number of swarms
over the iterations of the algorithm. Nmmso.R uses the same measures to
reach the highest comparability possible.

The first measure used is the Success Ratio (SR). The Success Ratio is
defined as the percentage of Successful runs (runs that found all global
optima) over all runs (Li et al. 2013, p. 7). As for the other ratios
this measure was taken over several independent runs and collectively
evaluated. The taken measures for the Success Ratio can be found in
Table 2. \[\frac{successful\ runs}{NR} = SR \] Here \(NR\) denotes the
Number of runs done to reach this measure. \newline

\begin{longtable}[c]{@{}crrrrrr@{}}
\caption{Success Ratio over given runs (Measure of share of runs which
found all global optima)}\tabularnewline
\toprule
\begin{minipage}[b]{0.11\columnwidth}\centering\strut
~
\strut\end{minipage} &
\begin{minipage}[b]{0.07\columnwidth}\raggedleft\strut
0.1
\strut\end{minipage} &
\begin{minipage}[b]{0.08\columnwidth}\raggedleft\strut
0.01
\strut\end{minipage} &
\begin{minipage}[b]{0.09\columnwidth}\raggedleft\strut
0.001
\strut\end{minipage} &
\begin{minipage}[b]{0.10\columnwidth}\raggedleft\strut
0.0001
\strut\end{minipage} &
\begin{minipage}[b]{0.11\columnwidth}\raggedleft\strut
0.00001
\strut\end{minipage} &
\begin{minipage}[b]{0.07\columnwidth}\raggedleft\strut
runs
\strut\end{minipage}\tabularnewline
\midrule
\endfirsthead
\toprule
\begin{minipage}[b]{0.11\columnwidth}\centering\strut
~
\strut\end{minipage} &
\begin{minipage}[b]{0.07\columnwidth}\raggedleft\strut
0.1
\strut\end{minipage} &
\begin{minipage}[b]{0.08\columnwidth}\raggedleft\strut
0.01
\strut\end{minipage} &
\begin{minipage}[b]{0.09\columnwidth}\raggedleft\strut
0.001
\strut\end{minipage} &
\begin{minipage}[b]{0.10\columnwidth}\raggedleft\strut
0.0001
\strut\end{minipage} &
\begin{minipage}[b]{0.11\columnwidth}\raggedleft\strut
0.00001
\strut\end{minipage} &
\begin{minipage}[b]{0.07\columnwidth}\raggedleft\strut
runs
\strut\end{minipage}\tabularnewline
\midrule
\endhead
\begin{minipage}[t]{0.11\columnwidth}\centering\strut
\textbf{F1}
\strut\end{minipage} &
\begin{minipage}[t]{0.07\columnwidth}\raggedleft\strut
1
\strut\end{minipage} &
\begin{minipage}[t]{0.08\columnwidth}\raggedleft\strut
1
\strut\end{minipage} &
\begin{minipage}[t]{0.09\columnwidth}\raggedleft\strut
1
\strut\end{minipage} &
\begin{minipage}[t]{0.10\columnwidth}\raggedleft\strut
1
\strut\end{minipage} &
\begin{minipage}[t]{0.11\columnwidth}\raggedleft\strut
1
\strut\end{minipage} &
\begin{minipage}[t]{0.07\columnwidth}\raggedleft\strut
37
\strut\end{minipage}\tabularnewline
\begin{minipage}[t]{0.11\columnwidth}\centering\strut
\textbf{F2}
\strut\end{minipage} &
\begin{minipage}[t]{0.07\columnwidth}\raggedleft\strut
1
\strut\end{minipage} &
\begin{minipage}[t]{0.08\columnwidth}\raggedleft\strut
1
\strut\end{minipage} &
\begin{minipage}[t]{0.09\columnwidth}\raggedleft\strut
1
\strut\end{minipage} &
\begin{minipage}[t]{0.10\columnwidth}\raggedleft\strut
1
\strut\end{minipage} &
\begin{minipage}[t]{0.11\columnwidth}\raggedleft\strut
1
\strut\end{minipage} &
\begin{minipage}[t]{0.07\columnwidth}\raggedleft\strut
35
\strut\end{minipage}\tabularnewline
\begin{minipage}[t]{0.11\columnwidth}\centering\strut
\textbf{F3}
\strut\end{minipage} &
\begin{minipage}[t]{0.07\columnwidth}\raggedleft\strut
1
\strut\end{minipage} &
\begin{minipage}[t]{0.08\columnwidth}\raggedleft\strut
1
\strut\end{minipage} &
\begin{minipage}[t]{0.09\columnwidth}\raggedleft\strut
1
\strut\end{minipage} &
\begin{minipage}[t]{0.10\columnwidth}\raggedleft\strut
1
\strut\end{minipage} &
\begin{minipage}[t]{0.11\columnwidth}\raggedleft\strut
1
\strut\end{minipage} &
\begin{minipage}[t]{0.07\columnwidth}\raggedleft\strut
37
\strut\end{minipage}\tabularnewline
\begin{minipage}[t]{0.11\columnwidth}\centering\strut
\textbf{F4}
\strut\end{minipage} &
\begin{minipage}[t]{0.07\columnwidth}\raggedleft\strut
1
\strut\end{minipage} &
\begin{minipage}[t]{0.08\columnwidth}\raggedleft\strut
1
\strut\end{minipage} &
\begin{minipage}[t]{0.09\columnwidth}\raggedleft\strut
1
\strut\end{minipage} &
\begin{minipage}[t]{0.10\columnwidth}\raggedleft\strut
1
\strut\end{minipage} &
\begin{minipage}[t]{0.11\columnwidth}\raggedleft\strut
1
\strut\end{minipage} &
\begin{minipage}[t]{0.07\columnwidth}\raggedleft\strut
37
\strut\end{minipage}\tabularnewline
\begin{minipage}[t]{0.11\columnwidth}\centering\strut
\textbf{F5}
\strut\end{minipage} &
\begin{minipage}[t]{0.07\columnwidth}\raggedleft\strut
1
\strut\end{minipage} &
\begin{minipage}[t]{0.08\columnwidth}\raggedleft\strut
1
\strut\end{minipage} &
\begin{minipage}[t]{0.09\columnwidth}\raggedleft\strut
1
\strut\end{minipage} &
\begin{minipage}[t]{0.10\columnwidth}\raggedleft\strut
1
\strut\end{minipage} &
\begin{minipage}[t]{0.11\columnwidth}\raggedleft\strut
1
\strut\end{minipage} &
\begin{minipage}[t]{0.07\columnwidth}\raggedleft\strut
34
\strut\end{minipage}\tabularnewline
\begin{minipage}[t]{0.11\columnwidth}\centering\strut
\textbf{F6}
\strut\end{minipage} &
\begin{minipage}[t]{0.07\columnwidth}\raggedleft\strut
1
\strut\end{minipage} &
\begin{minipage}[t]{0.08\columnwidth}\raggedleft\strut
1
\strut\end{minipage} &
\begin{minipage}[t]{0.09\columnwidth}\raggedleft\strut
1
\strut\end{minipage} &
\begin{minipage}[t]{0.10\columnwidth}\raggedleft\strut
1
\strut\end{minipage} &
\begin{minipage}[t]{0.11\columnwidth}\raggedleft\strut
0
\strut\end{minipage} &
\begin{minipage}[t]{0.07\columnwidth}\raggedleft\strut
34
\strut\end{minipage}\tabularnewline
\begin{minipage}[t]{0.11\columnwidth}\centering\strut
\textbf{F7}
\strut\end{minipage} &
\begin{minipage}[t]{0.07\columnwidth}\raggedleft\strut
1
\strut\end{minipage} &
\begin{minipage}[t]{0.08\columnwidth}\raggedleft\strut
1
\strut\end{minipage} &
\begin{minipage}[t]{0.09\columnwidth}\raggedleft\strut
1
\strut\end{minipage} &
\begin{minipage}[t]{0.10\columnwidth}\raggedleft\strut
1
\strut\end{minipage} &
\begin{minipage}[t]{0.11\columnwidth}\raggedleft\strut
1
\strut\end{minipage} &
\begin{minipage}[t]{0.07\columnwidth}\raggedleft\strut
34
\strut\end{minipage}\tabularnewline
\begin{minipage}[t]{0.11\columnwidth}\centering\strut
\textbf{F8}
\strut\end{minipage} &
\begin{minipage}[t]{0.07\columnwidth}\raggedleft\strut
1
\strut\end{minipage} &
\begin{minipage}[t]{0.08\columnwidth}\raggedleft\strut
1
\strut\end{minipage} &
\begin{minipage}[t]{0.09\columnwidth}\raggedleft\strut
1
\strut\end{minipage} &
\begin{minipage}[t]{0.10\columnwidth}\raggedleft\strut
0.94
\strut\end{minipage} &
\begin{minipage}[t]{0.11\columnwidth}\raggedleft\strut
0.71
\strut\end{minipage} &
\begin{minipage}[t]{0.07\columnwidth}\raggedleft\strut
17
\strut\end{minipage}\tabularnewline
\begin{minipage}[t]{0.11\columnwidth}\centering\strut
\textbf{F9}
\strut\end{minipage} &
\begin{minipage}[t]{0.07\columnwidth}\raggedleft\strut
0.95
\strut\end{minipage} &
\begin{minipage}[t]{0.08\columnwidth}\raggedleft\strut
0.95
\strut\end{minipage} &
\begin{minipage}[t]{0.09\columnwidth}\raggedleft\strut
0.95
\strut\end{minipage} &
\begin{minipage}[t]{0.10\columnwidth}\raggedleft\strut
0.95
\strut\end{minipage} &
\begin{minipage}[t]{0.11\columnwidth}\raggedleft\strut
0.95
\strut\end{minipage} &
\begin{minipage}[t]{0.07\columnwidth}\raggedleft\strut
20
\strut\end{minipage}\tabularnewline
\begin{minipage}[t]{0.11\columnwidth}\centering\strut
\textbf{F10}
\strut\end{minipage} &
\begin{minipage}[t]{0.07\columnwidth}\raggedleft\strut
1
\strut\end{minipage} &
\begin{minipage}[t]{0.08\columnwidth}\raggedleft\strut
1
\strut\end{minipage} &
\begin{minipage}[t]{0.09\columnwidth}\raggedleft\strut
1
\strut\end{minipage} &
\begin{minipage}[t]{0.10\columnwidth}\raggedleft\strut
1
\strut\end{minipage} &
\begin{minipage}[t]{0.11\columnwidth}\raggedleft\strut
1
\strut\end{minipage} &
\begin{minipage}[t]{0.07\columnwidth}\raggedleft\strut
34
\strut\end{minipage}\tabularnewline
\begin{minipage}[t]{0.11\columnwidth}\centering\strut
\textbf{F11}
\strut\end{minipage} &
\begin{minipage}[t]{0.07\columnwidth}\raggedleft\strut
1
\strut\end{minipage} &
\begin{minipage}[t]{0.08\columnwidth}\raggedleft\strut
1
\strut\end{minipage} &
\begin{minipage}[t]{0.09\columnwidth}\raggedleft\strut
1
\strut\end{minipage} &
\begin{minipage}[t]{0.10\columnwidth}\raggedleft\strut
1
\strut\end{minipage} &
\begin{minipage}[t]{0.11\columnwidth}\raggedleft\strut
1
\strut\end{minipage} &
\begin{minipage}[t]{0.07\columnwidth}\raggedleft\strut
33
\strut\end{minipage}\tabularnewline
\begin{minipage}[t]{0.11\columnwidth}\centering\strut
\textbf{F12}
\strut\end{minipage} &
\begin{minipage}[t]{0.07\columnwidth}\raggedleft\strut
1
\strut\end{minipage} &
\begin{minipage}[t]{0.08\columnwidth}\raggedleft\strut
1
\strut\end{minipage} &
\begin{minipage}[t]{0.09\columnwidth}\raggedleft\strut
1
\strut\end{minipage} &
\begin{minipage}[t]{0.10\columnwidth}\raggedleft\strut
1
\strut\end{minipage} &
\begin{minipage}[t]{0.11\columnwidth}\raggedleft\strut
1
\strut\end{minipage} &
\begin{minipage}[t]{0.07\columnwidth}\raggedleft\strut
34
\strut\end{minipage}\tabularnewline
\begin{minipage}[t]{0.11\columnwidth}\centering\strut
\textbf{F13}
\strut\end{minipage} &
\begin{minipage}[t]{0.07\columnwidth}\raggedleft\strut
1
\strut\end{minipage} &
\begin{minipage}[t]{0.08\columnwidth}\raggedleft\strut
1
\strut\end{minipage} &
\begin{minipage}[t]{0.09\columnwidth}\raggedleft\strut
1
\strut\end{minipage} &
\begin{minipage}[t]{0.10\columnwidth}\raggedleft\strut
1
\strut\end{minipage} &
\begin{minipage}[t]{0.11\columnwidth}\raggedleft\strut
1
\strut\end{minipage} &
\begin{minipage}[t]{0.07\columnwidth}\raggedleft\strut
34
\strut\end{minipage}\tabularnewline
\begin{minipage}[t]{0.11\columnwidth}\centering\strut
\textbf{F14}
\strut\end{minipage} &
\begin{minipage}[t]{0.07\columnwidth}\raggedleft\strut
1
\strut\end{minipage} &
\begin{minipage}[t]{0.08\columnwidth}\raggedleft\strut
1
\strut\end{minipage} &
\begin{minipage}[t]{0.09\columnwidth}\raggedleft\strut
1
\strut\end{minipage} &
\begin{minipage}[t]{0.10\columnwidth}\raggedleft\strut
1
\strut\end{minipage} &
\begin{minipage}[t]{0.11\columnwidth}\raggedleft\strut
1
\strut\end{minipage} &
\begin{minipage}[t]{0.07\columnwidth}\raggedleft\strut
33
\strut\end{minipage}\tabularnewline
\begin{minipage}[t]{0.11\columnwidth}\centering\strut
\textbf{F15}
\strut\end{minipage} &
\begin{minipage}[t]{0.07\columnwidth}\raggedleft\strut
0.96
\strut\end{minipage} &
\begin{minipage}[t]{0.08\columnwidth}\raggedleft\strut
0.96
\strut\end{minipage} &
\begin{minipage}[t]{0.09\columnwidth}\raggedleft\strut
0.96
\strut\end{minipage} &
\begin{minipage}[t]{0.10\columnwidth}\raggedleft\strut
0.96
\strut\end{minipage} &
\begin{minipage}[t]{0.11\columnwidth}\raggedleft\strut
0.96
\strut\end{minipage} &
\begin{minipage}[t]{0.07\columnwidth}\raggedleft\strut
26
\strut\end{minipage}\tabularnewline
\begin{minipage}[t]{0.11\columnwidth}\centering\strut
\textbf{F16}
\strut\end{minipage} &
\begin{minipage}[t]{0.07\columnwidth}\raggedleft\strut
0
\strut\end{minipage} &
\begin{minipage}[t]{0.08\columnwidth}\raggedleft\strut
0
\strut\end{minipage} &
\begin{minipage}[t]{0.09\columnwidth}\raggedleft\strut
0
\strut\end{minipage} &
\begin{minipage}[t]{0.10\columnwidth}\raggedleft\strut
0
\strut\end{minipage} &
\begin{minipage}[t]{0.11\columnwidth}\raggedleft\strut
0
\strut\end{minipage} &
\begin{minipage}[t]{0.07\columnwidth}\raggedleft\strut
13
\strut\end{minipage}\tabularnewline
\begin{minipage}[t]{0.11\columnwidth}\centering\strut
\textbf{F17}
\strut\end{minipage} &
\begin{minipage}[t]{0.07\columnwidth}\raggedleft\strut
0.17
\strut\end{minipage} &
\begin{minipage}[t]{0.08\columnwidth}\raggedleft\strut
0
\strut\end{minipage} &
\begin{minipage}[t]{0.09\columnwidth}\raggedleft\strut
0
\strut\end{minipage} &
\begin{minipage}[t]{0.10\columnwidth}\raggedleft\strut
0
\strut\end{minipage} &
\begin{minipage}[t]{0.11\columnwidth}\raggedleft\strut
0
\strut\end{minipage} &
\begin{minipage}[t]{0.07\columnwidth}\raggedleft\strut
12
\strut\end{minipage}\tabularnewline
\begin{minipage}[t]{0.11\columnwidth}\centering\strut
\textbf{F18}
\strut\end{minipage} &
\begin{minipage}[t]{0.07\columnwidth}\raggedleft\strut
0.38
\strut\end{minipage} &
\begin{minipage}[t]{0.08\columnwidth}\raggedleft\strut
0.38
\strut\end{minipage} &
\begin{minipage}[t]{0.09\columnwidth}\raggedleft\strut
0.38
\strut\end{minipage} &
\begin{minipage}[t]{0.10\columnwidth}\raggedleft\strut
0.31
\strut\end{minipage} &
\begin{minipage}[t]{0.11\columnwidth}\raggedleft\strut
0.31
\strut\end{minipage} &
\begin{minipage}[t]{0.07\columnwidth}\raggedleft\strut
16
\strut\end{minipage}\tabularnewline
\begin{minipage}[t]{0.11\columnwidth}\centering\strut
\textbf{F19}
\strut\end{minipage} &
\begin{minipage}[t]{0.07\columnwidth}\raggedleft\strut
0
\strut\end{minipage} &
\begin{minipage}[t]{0.08\columnwidth}\raggedleft\strut
0
\strut\end{minipage} &
\begin{minipage}[t]{0.09\columnwidth}\raggedleft\strut
0
\strut\end{minipage} &
\begin{minipage}[t]{0.10\columnwidth}\raggedleft\strut
0
\strut\end{minipage} &
\begin{minipage}[t]{0.11\columnwidth}\raggedleft\strut
0
\strut\end{minipage} &
\begin{minipage}[t]{0.07\columnwidth}\raggedleft\strut
13
\strut\end{minipage}\tabularnewline
\begin{minipage}[t]{0.11\columnwidth}\centering\strut
\textbf{F20}
\strut\end{minipage} &
\begin{minipage}[t]{0.07\columnwidth}\raggedleft\strut
0
\strut\end{minipage} &
\begin{minipage}[t]{0.08\columnwidth}\raggedleft\strut
0
\strut\end{minipage} &
\begin{minipage}[t]{0.09\columnwidth}\raggedleft\strut
0
\strut\end{minipage} &
\begin{minipage}[t]{0.10\columnwidth}\raggedleft\strut
0
\strut\end{minipage} &
\begin{minipage}[t]{0.11\columnwidth}\raggedleft\strut
0
\strut\end{minipage} &
\begin{minipage}[t]{0.07\columnwidth}\raggedleft\strut
12
\strut\end{minipage}\tabularnewline
\bottomrule
\end{longtable}

The second measure introduced by the CEC committee and also used by
Dr.~Fieldsend is the Convergence Rate. The Convergence Rate (CR)
measures the needed evaluations per Accuracy and Function to find all
global optima (Li et al. 2013, p. 7). This measure takes the mean of
evaluations over all runs. The results of this measure can be found in
Table 3.

\[\frac{\sum\nolimits_{n=1}^{NR} evals_{n}}{NR} = CR\] In this measure
\(evals\) denotes the number of evaluations done. \newline

\begin{longtable}[c]{@{}crrrrrr@{}}
\caption{Convergence Rates over given runs (Mean of evaluations needed
to find all global optima, if all optima have never been found the
maximum allowed evaluations for that function were
taken.)}\tabularnewline
\toprule
\begin{minipage}[b]{0.11\columnwidth}\centering\strut
~
\strut\end{minipage} &
\begin{minipage}[b]{0.08\columnwidth}\raggedleft\strut
0.1
\strut\end{minipage} &
\begin{minipage}[b]{0.08\columnwidth}\raggedleft\strut
0.01
\strut\end{minipage} &
\begin{minipage}[b]{0.09\columnwidth}\raggedleft\strut
0.001
\strut\end{minipage} &
\begin{minipage}[b]{0.10\columnwidth}\raggedleft\strut
0.0001
\strut\end{minipage} &
\begin{minipage}[b]{0.11\columnwidth}\raggedleft\strut
0.00001
\strut\end{minipage} &
\begin{minipage}[b]{0.07\columnwidth}\raggedleft\strut
runs
\strut\end{minipage}\tabularnewline
\midrule
\endfirsthead
\toprule
\begin{minipage}[b]{0.11\columnwidth}\centering\strut
~
\strut\end{minipage} &
\begin{minipage}[b]{0.08\columnwidth}\raggedleft\strut
0.1
\strut\end{minipage} &
\begin{minipage}[b]{0.08\columnwidth}\raggedleft\strut
0.01
\strut\end{minipage} &
\begin{minipage}[b]{0.09\columnwidth}\raggedleft\strut
0.001
\strut\end{minipage} &
\begin{minipage}[b]{0.10\columnwidth}\raggedleft\strut
0.0001
\strut\end{minipage} &
\begin{minipage}[b]{0.11\columnwidth}\raggedleft\strut
0.00001
\strut\end{minipage} &
\begin{minipage}[b]{0.07\columnwidth}\raggedleft\strut
runs
\strut\end{minipage}\tabularnewline
\midrule
\endhead
\begin{minipage}[t]{0.11\columnwidth}\centering\strut
\textbf{F1}
\strut\end{minipage} &
\begin{minipage}[t]{0.08\columnwidth}\raggedleft\strut
622
\strut\end{minipage} &
\begin{minipage}[t]{0.08\columnwidth}\raggedleft\strut
815
\strut\end{minipage} &
\begin{minipage}[t]{0.09\columnwidth}\raggedleft\strut
1018
\strut\end{minipage} &
\begin{minipage}[t]{0.10\columnwidth}\raggedleft\strut
1205
\strut\end{minipage} &
\begin{minipage}[t]{0.11\columnwidth}\raggedleft\strut
1441
\strut\end{minipage} &
\begin{minipage}[t]{0.07\columnwidth}\raggedleft\strut
37
\strut\end{minipage}\tabularnewline
\begin{minipage}[t]{0.11\columnwidth}\centering\strut
\textbf{F2}
\strut\end{minipage} &
\begin{minipage}[t]{0.08\columnwidth}\raggedleft\strut
179
\strut\end{minipage} &
\begin{minipage}[t]{0.08\columnwidth}\raggedleft\strut
269
\strut\end{minipage} &
\begin{minipage}[t]{0.09\columnwidth}\raggedleft\strut
397
\strut\end{minipage} &
\begin{minipage}[t]{0.10\columnwidth}\raggedleft\strut
533
\strut\end{minipage} &
\begin{minipage}[t]{0.11\columnwidth}\raggedleft\strut
640
\strut\end{minipage} &
\begin{minipage}[t]{0.07\columnwidth}\raggedleft\strut
35
\strut\end{minipage}\tabularnewline
\begin{minipage}[t]{0.11\columnwidth}\centering\strut
\textbf{F3}
\strut\end{minipage} &
\begin{minipage}[t]{0.08\columnwidth}\raggedleft\strut
35
\strut\end{minipage} &
\begin{minipage}[t]{0.08\columnwidth}\raggedleft\strut
170
\strut\end{minipage} &
\begin{minipage}[t]{0.09\columnwidth}\raggedleft\strut
273
\strut\end{minipage} &
\begin{minipage}[t]{0.10\columnwidth}\raggedleft\strut
390
\strut\end{minipage} &
\begin{minipage}[t]{0.11\columnwidth}\raggedleft\strut
513
\strut\end{minipage} &
\begin{minipage}[t]{0.07\columnwidth}\raggedleft\strut
37
\strut\end{minipage}\tabularnewline
\begin{minipage}[t]{0.11\columnwidth}\centering\strut
\textbf{F4}
\strut\end{minipage} &
\begin{minipage}[t]{0.08\columnwidth}\raggedleft\strut
506
\strut\end{minipage} &
\begin{minipage}[t]{0.08\columnwidth}\raggedleft\strut
735
\strut\end{minipage} &
\begin{minipage}[t]{0.09\columnwidth}\raggedleft\strut
961
\strut\end{minipage} &
\begin{minipage}[t]{0.10\columnwidth}\raggedleft\strut
1201
\strut\end{minipage} &
\begin{minipage}[t]{0.11\columnwidth}\raggedleft\strut
1455
\strut\end{minipage} &
\begin{minipage}[t]{0.07\columnwidth}\raggedleft\strut
37
\strut\end{minipage}\tabularnewline
\begin{minipage}[t]{0.11\columnwidth}\centering\strut
\textbf{F5}
\strut\end{minipage} &
\begin{minipage}[t]{0.08\columnwidth}\raggedleft\strut
82
\strut\end{minipage} &
\begin{minipage}[t]{0.08\columnwidth}\raggedleft\strut
194
\strut\end{minipage} &
\begin{minipage}[t]{0.09\columnwidth}\raggedleft\strut
321
\strut\end{minipage} &
\begin{minipage}[t]{0.10\columnwidth}\raggedleft\strut
522
\strut\end{minipage} &
\begin{minipage}[t]{0.11\columnwidth}\raggedleft\strut
788
\strut\end{minipage} &
\begin{minipage}[t]{0.07\columnwidth}\raggedleft\strut
34
\strut\end{minipage}\tabularnewline
\begin{minipage}[t]{0.11\columnwidth}\centering\strut
\textbf{F6}
\strut\end{minipage} &
\begin{minipage}[t]{0.08\columnwidth}\raggedleft\strut
19519
\strut\end{minipage} &
\begin{minipage}[t]{0.08\columnwidth}\raggedleft\strut
24388
\strut\end{minipage} &
\begin{minipage}[t]{0.09\columnwidth}\raggedleft\strut
30499
\strut\end{minipage} &
\begin{minipage}[t]{0.10\columnwidth}\raggedleft\strut
42404
\strut\end{minipage} &
\begin{minipage}[t]{0.11\columnwidth}\raggedleft\strut
200001
\strut\end{minipage} &
\begin{minipage}[t]{0.07\columnwidth}\raggedleft\strut
34
\strut\end{minipage}\tabularnewline
\begin{minipage}[t]{0.11\columnwidth}\centering\strut
\textbf{F7}
\strut\end{minipage} &
\begin{minipage}[t]{0.08\columnwidth}\raggedleft\strut
8564
\strut\end{minipage} &
\begin{minipage}[t]{0.08\columnwidth}\raggedleft\strut
9211
\strut\end{minipage} &
\begin{minipage}[t]{0.09\columnwidth}\raggedleft\strut
10604
\strut\end{minipage} &
\begin{minipage}[t]{0.10\columnwidth}\raggedleft\strut
12328
\strut\end{minipage} &
\begin{minipage}[t]{0.11\columnwidth}\raggedleft\strut
14646
\strut\end{minipage} &
\begin{minipage}[t]{0.07\columnwidth}\raggedleft\strut
34
\strut\end{minipage}\tabularnewline
\begin{minipage}[t]{0.11\columnwidth}\centering\strut
\textbf{F8}
\strut\end{minipage} &
\begin{minipage}[t]{0.08\columnwidth}\raggedleft\strut
194031
\strut\end{minipage} &
\begin{minipage}[t]{0.08\columnwidth}\raggedleft\strut
231763
\strut\end{minipage} &
\begin{minipage}[t]{0.09\columnwidth}\raggedleft\strut
271058
\strut\end{minipage} &
\begin{minipage}[t]{0.10\columnwidth}\raggedleft\strut
320284
\strut\end{minipage} &
\begin{minipage}[t]{0.11\columnwidth}\raggedleft\strut
354533
\strut\end{minipage} &
\begin{minipage}[t]{0.07\columnwidth}\raggedleft\strut
17
\strut\end{minipage}\tabularnewline
\begin{minipage}[t]{0.11\columnwidth}\centering\strut
\textbf{F9}
\strut\end{minipage} &
\begin{minipage}[t]{0.08\columnwidth}\raggedleft\strut
185670
\strut\end{minipage} &
\begin{minipage}[t]{0.08\columnwidth}\raggedleft\strut
189360
\strut\end{minipage} &
\begin{minipage}[t]{0.09\columnwidth}\raggedleft\strut
204110
\strut\end{minipage} &
\begin{minipage}[t]{0.10\columnwidth}\raggedleft\strut
219559
\strut\end{minipage} &
\begin{minipage}[t]{0.11\columnwidth}\raggedleft\strut
228586
\strut\end{minipage} &
\begin{minipage}[t]{0.07\columnwidth}\raggedleft\strut
20
\strut\end{minipage}\tabularnewline
\begin{minipage}[t]{0.11\columnwidth}\centering\strut
\textbf{F10}
\strut\end{minipage} &
\begin{minipage}[t]{0.08\columnwidth}\raggedleft\strut
887
\strut\end{minipage} &
\begin{minipage}[t]{0.08\columnwidth}\raggedleft\strut
1326
\strut\end{minipage} &
\begin{minipage}[t]{0.09\columnwidth}\raggedleft\strut
1728
\strut\end{minipage} &
\begin{minipage}[t]{0.10\columnwidth}\raggedleft\strut
2254
\strut\end{minipage} &
\begin{minipage}[t]{0.11\columnwidth}\raggedleft\strut
2758
\strut\end{minipage} &
\begin{minipage}[t]{0.07\columnwidth}\raggedleft\strut
34
\strut\end{minipage}\tabularnewline
\begin{minipage}[t]{0.11\columnwidth}\centering\strut
\textbf{F11}
\strut\end{minipage} &
\begin{minipage}[t]{0.08\columnwidth}\raggedleft\strut
3692
\strut\end{minipage} &
\begin{minipage}[t]{0.08\columnwidth}\raggedleft\strut
5747
\strut\end{minipage} &
\begin{minipage}[t]{0.09\columnwidth}\raggedleft\strut
7347
\strut\end{minipage} &
\begin{minipage}[t]{0.10\columnwidth}\raggedleft\strut
8549
\strut\end{minipage} &
\begin{minipage}[t]{0.11\columnwidth}\raggedleft\strut
9164
\strut\end{minipage} &
\begin{minipage}[t]{0.07\columnwidth}\raggedleft\strut
33
\strut\end{minipage}\tabularnewline
\begin{minipage}[t]{0.11\columnwidth}\centering\strut
\textbf{F12}
\strut\end{minipage} &
\begin{minipage}[t]{0.08\columnwidth}\raggedleft\strut
17321
\strut\end{minipage} &
\begin{minipage}[t]{0.08\columnwidth}\raggedleft\strut
25823
\strut\end{minipage} &
\begin{minipage}[t]{0.09\columnwidth}\raggedleft\strut
38464
\strut\end{minipage} &
\begin{minipage}[t]{0.10\columnwidth}\raggedleft\strut
44660
\strut\end{minipage} &
\begin{minipage}[t]{0.11\columnwidth}\raggedleft\strut
51792
\strut\end{minipage} &
\begin{minipage}[t]{0.07\columnwidth}\raggedleft\strut
34
\strut\end{minipage}\tabularnewline
\begin{minipage}[t]{0.11\columnwidth}\centering\strut
\textbf{F13}
\strut\end{minipage} &
\begin{minipage}[t]{0.08\columnwidth}\raggedleft\strut
11338
\strut\end{minipage} &
\begin{minipage}[t]{0.08\columnwidth}\raggedleft\strut
15676
\strut\end{minipage} &
\begin{minipage}[t]{0.09\columnwidth}\raggedleft\strut
19107
\strut\end{minipage} &
\begin{minipage}[t]{0.10\columnwidth}\raggedleft\strut
23152
\strut\end{minipage} &
\begin{minipage}[t]{0.11\columnwidth}\raggedleft\strut
26802
\strut\end{minipage} &
\begin{minipage}[t]{0.07\columnwidth}\raggedleft\strut
34
\strut\end{minipage}\tabularnewline
\begin{minipage}[t]{0.11\columnwidth}\centering\strut
\textbf{F14}
\strut\end{minipage} &
\begin{minipage}[t]{0.08\columnwidth}\raggedleft\strut
28776
\strut\end{minipage} &
\begin{minipage}[t]{0.08\columnwidth}\raggedleft\strut
34298
\strut\end{minipage} &
\begin{minipage}[t]{0.09\columnwidth}\raggedleft\strut
48738
\strut\end{minipage} &
\begin{minipage}[t]{0.10\columnwidth}\raggedleft\strut
59775
\strut\end{minipage} &
\begin{minipage}[t]{0.11\columnwidth}\raggedleft\strut
68576
\strut\end{minipage} &
\begin{minipage}[t]{0.07\columnwidth}\raggedleft\strut
33
\strut\end{minipage}\tabularnewline
\begin{minipage}[t]{0.11\columnwidth}\centering\strut
\textbf{F15}
\strut\end{minipage} &
\begin{minipage}[t]{0.08\columnwidth}\raggedleft\strut
107225
\strut\end{minipage} &
\begin{minipage}[t]{0.08\columnwidth}\raggedleft\strut
129151
\strut\end{minipage} &
\begin{minipage}[t]{0.09\columnwidth}\raggedleft\strut
149266
\strut\end{minipage} &
\begin{minipage}[t]{0.10\columnwidth}\raggedleft\strut
172770
\strut\end{minipage} &
\begin{minipage}[t]{0.11\columnwidth}\raggedleft\strut
189144
\strut\end{minipage} &
\begin{minipage}[t]{0.07\columnwidth}\raggedleft\strut
26
\strut\end{minipage}\tabularnewline
\begin{minipage}[t]{0.11\columnwidth}\centering\strut
\textbf{F16}
\strut\end{minipage} &
\begin{minipage}[t]{0.08\columnwidth}\raggedleft\strut
400001
\strut\end{minipage} &
\begin{minipage}[t]{0.08\columnwidth}\raggedleft\strut
400001
\strut\end{minipage} &
\begin{minipage}[t]{0.09\columnwidth}\raggedleft\strut
400001
\strut\end{minipage} &
\begin{minipage}[t]{0.10\columnwidth}\raggedleft\strut
400001
\strut\end{minipage} &
\begin{minipage}[t]{0.11\columnwidth}\raggedleft\strut
400001
\strut\end{minipage} &
\begin{minipage}[t]{0.07\columnwidth}\raggedleft\strut
13
\strut\end{minipage}\tabularnewline
\begin{minipage}[t]{0.11\columnwidth}\centering\strut
\textbf{F17}
\strut\end{minipage} &
\begin{minipage}[t]{0.08\columnwidth}\raggedleft\strut
377470
\strut\end{minipage} &
\begin{minipage}[t]{0.08\columnwidth}\raggedleft\strut
400001
\strut\end{minipage} &
\begin{minipage}[t]{0.09\columnwidth}\raggedleft\strut
400001
\strut\end{minipage} &
\begin{minipage}[t]{0.10\columnwidth}\raggedleft\strut
400001
\strut\end{minipage} &
\begin{minipage}[t]{0.11\columnwidth}\raggedleft\strut
400001
\strut\end{minipage} &
\begin{minipage}[t]{0.07\columnwidth}\raggedleft\strut
12
\strut\end{minipage}\tabularnewline
\begin{minipage}[t]{0.11\columnwidth}\centering\strut
\textbf{F18}
\strut\end{minipage} &
\begin{minipage}[t]{0.08\columnwidth}\raggedleft\strut
299577
\strut\end{minipage} &
\begin{minipage}[t]{0.08\columnwidth}\raggedleft\strut
302391
\strut\end{minipage} &
\begin{minipage}[t]{0.09\columnwidth}\raggedleft\strut
304385
\strut\end{minipage} &
\begin{minipage}[t]{0.10\columnwidth}\raggedleft\strut
309050
\strut\end{minipage} &
\begin{minipage}[t]{0.11\columnwidth}\raggedleft\strut
309450
\strut\end{minipage} &
\begin{minipage}[t]{0.07\columnwidth}\raggedleft\strut
16
\strut\end{minipage}\tabularnewline
\begin{minipage}[t]{0.11\columnwidth}\centering\strut
\textbf{F19}
\strut\end{minipage} &
\begin{minipage}[t]{0.08\columnwidth}\raggedleft\strut
400001
\strut\end{minipage} &
\begin{minipage}[t]{0.08\columnwidth}\raggedleft\strut
400001
\strut\end{minipage} &
\begin{minipage}[t]{0.09\columnwidth}\raggedleft\strut
400001
\strut\end{minipage} &
\begin{minipage}[t]{0.10\columnwidth}\raggedleft\strut
400001
\strut\end{minipage} &
\begin{minipage}[t]{0.11\columnwidth}\raggedleft\strut
400001
\strut\end{minipage} &
\begin{minipage}[t]{0.07\columnwidth}\raggedleft\strut
13
\strut\end{minipage}\tabularnewline
\begin{minipage}[t]{0.11\columnwidth}\centering\strut
\textbf{F20}
\strut\end{minipage} &
\begin{minipage}[t]{0.08\columnwidth}\raggedleft\strut
400001
\strut\end{minipage} &
\begin{minipage}[t]{0.08\columnwidth}\raggedleft\strut
400001
\strut\end{minipage} &
\begin{minipage}[t]{0.09\columnwidth}\raggedleft\strut
400001
\strut\end{minipage} &
\begin{minipage}[t]{0.10\columnwidth}\raggedleft\strut
400001
\strut\end{minipage} &
\begin{minipage}[t]{0.11\columnwidth}\raggedleft\strut
400001
\strut\end{minipage} &
\begin{minipage}[t]{0.07\columnwidth}\raggedleft\strut
12
\strut\end{minipage}\tabularnewline
\bottomrule
\end{longtable}

The third measure is the Peak Ratio (PR). It measures the share of found
global optima over all runs (Li et al. 2013, p. 7). The results of this
evaluation can be found in Table 4.

\[\frac{\sum\nolimits_{n=1}^{NR} NOF_{n}}{NKO * NR} = PR\] \newline
In this measure \(NOF\) denotes the number of found optima per run and
\(NKO\) the number of known optima for the function. \newline

\begin{longtable}[c]{@{}crrrrrr@{}}
\caption{Peak Ratio over given runs (Share of found global optima over
all runs)}\tabularnewline
\toprule
\begin{minipage}[b]{0.11\columnwidth}\centering\strut
~
\strut\end{minipage} &
\begin{minipage}[b]{0.07\columnwidth}\raggedleft\strut
0.1
\strut\end{minipage} &
\begin{minipage}[b]{0.08\columnwidth}\raggedleft\strut
0.01
\strut\end{minipage} &
\begin{minipage}[b]{0.09\columnwidth}\raggedleft\strut
0.001
\strut\end{minipage} &
\begin{minipage}[b]{0.10\columnwidth}\raggedleft\strut
0.0001
\strut\end{minipage} &
\begin{minipage}[b]{0.11\columnwidth}\raggedleft\strut
0.00001
\strut\end{minipage} &
\begin{minipage}[b]{0.07\columnwidth}\raggedleft\strut
runs
\strut\end{minipage}\tabularnewline
\midrule
\endfirsthead
\toprule
\begin{minipage}[b]{0.11\columnwidth}\centering\strut
~
\strut\end{minipage} &
\begin{minipage}[b]{0.07\columnwidth}\raggedleft\strut
0.1
\strut\end{minipage} &
\begin{minipage}[b]{0.08\columnwidth}\raggedleft\strut
0.01
\strut\end{minipage} &
\begin{minipage}[b]{0.09\columnwidth}\raggedleft\strut
0.001
\strut\end{minipage} &
\begin{minipage}[b]{0.10\columnwidth}\raggedleft\strut
0.0001
\strut\end{minipage} &
\begin{minipage}[b]{0.11\columnwidth}\raggedleft\strut
0.00001
\strut\end{minipage} &
\begin{minipage}[b]{0.07\columnwidth}\raggedleft\strut
runs
\strut\end{minipage}\tabularnewline
\midrule
\endhead
\begin{minipage}[t]{0.11\columnwidth}\centering\strut
\textbf{F1}
\strut\end{minipage} &
\begin{minipage}[t]{0.07\columnwidth}\raggedleft\strut
1
\strut\end{minipage} &
\begin{minipage}[t]{0.08\columnwidth}\raggedleft\strut
1
\strut\end{minipage} &
\begin{minipage}[t]{0.09\columnwidth}\raggedleft\strut
1
\strut\end{minipage} &
\begin{minipage}[t]{0.10\columnwidth}\raggedleft\strut
1
\strut\end{minipage} &
\begin{minipage}[t]{0.11\columnwidth}\raggedleft\strut
1
\strut\end{minipage} &
\begin{minipage}[t]{0.07\columnwidth}\raggedleft\strut
37
\strut\end{minipage}\tabularnewline
\begin{minipage}[t]{0.11\columnwidth}\centering\strut
\textbf{F2}
\strut\end{minipage} &
\begin{minipage}[t]{0.07\columnwidth}\raggedleft\strut
1
\strut\end{minipage} &
\begin{minipage}[t]{0.08\columnwidth}\raggedleft\strut
1
\strut\end{minipage} &
\begin{minipage}[t]{0.09\columnwidth}\raggedleft\strut
1
\strut\end{minipage} &
\begin{minipage}[t]{0.10\columnwidth}\raggedleft\strut
1
\strut\end{minipage} &
\begin{minipage}[t]{0.11\columnwidth}\raggedleft\strut
1
\strut\end{minipage} &
\begin{minipage}[t]{0.07\columnwidth}\raggedleft\strut
35
\strut\end{minipage}\tabularnewline
\begin{minipage}[t]{0.11\columnwidth}\centering\strut
\textbf{F3}
\strut\end{minipage} &
\begin{minipage}[t]{0.07\columnwidth}\raggedleft\strut
1
\strut\end{minipage} &
\begin{minipage}[t]{0.08\columnwidth}\raggedleft\strut
1
\strut\end{minipage} &
\begin{minipage}[t]{0.09\columnwidth}\raggedleft\strut
1
\strut\end{minipage} &
\begin{minipage}[t]{0.10\columnwidth}\raggedleft\strut
1
\strut\end{minipage} &
\begin{minipage}[t]{0.11\columnwidth}\raggedleft\strut
1
\strut\end{minipage} &
\begin{minipage}[t]{0.07\columnwidth}\raggedleft\strut
37
\strut\end{minipage}\tabularnewline
\begin{minipage}[t]{0.11\columnwidth}\centering\strut
\textbf{F4}
\strut\end{minipage} &
\begin{minipage}[t]{0.07\columnwidth}\raggedleft\strut
1
\strut\end{minipage} &
\begin{minipage}[t]{0.08\columnwidth}\raggedleft\strut
1
\strut\end{minipage} &
\begin{minipage}[t]{0.09\columnwidth}\raggedleft\strut
1
\strut\end{minipage} &
\begin{minipage}[t]{0.10\columnwidth}\raggedleft\strut
1
\strut\end{minipage} &
\begin{minipage}[t]{0.11\columnwidth}\raggedleft\strut
1
\strut\end{minipage} &
\begin{minipage}[t]{0.07\columnwidth}\raggedleft\strut
37
\strut\end{minipage}\tabularnewline
\begin{minipage}[t]{0.11\columnwidth}\centering\strut
\textbf{F5}
\strut\end{minipage} &
\begin{minipage}[t]{0.07\columnwidth}\raggedleft\strut
1
\strut\end{minipage} &
\begin{minipage}[t]{0.08\columnwidth}\raggedleft\strut
1
\strut\end{minipage} &
\begin{minipage}[t]{0.09\columnwidth}\raggedleft\strut
1
\strut\end{minipage} &
\begin{minipage}[t]{0.10\columnwidth}\raggedleft\strut
1
\strut\end{minipage} &
\begin{minipage}[t]{0.11\columnwidth}\raggedleft\strut
1
\strut\end{minipage} &
\begin{minipage}[t]{0.07\columnwidth}\raggedleft\strut
34
\strut\end{minipage}\tabularnewline
\begin{minipage}[t]{0.11\columnwidth}\centering\strut
\textbf{F6}
\strut\end{minipage} &
\begin{minipage}[t]{0.07\columnwidth}\raggedleft\strut
1
\strut\end{minipage} &
\begin{minipage}[t]{0.08\columnwidth}\raggedleft\strut
1
\strut\end{minipage} &
\begin{minipage}[t]{0.09\columnwidth}\raggedleft\strut
1
\strut\end{minipage} &
\begin{minipage}[t]{0.10\columnwidth}\raggedleft\strut
1
\strut\end{minipage} &
\begin{minipage}[t]{0.11\columnwidth}\raggedleft\strut
0
\strut\end{minipage} &
\begin{minipage}[t]{0.07\columnwidth}\raggedleft\strut
34
\strut\end{minipage}\tabularnewline
\begin{minipage}[t]{0.11\columnwidth}\centering\strut
\textbf{F7}
\strut\end{minipage} &
\begin{minipage}[t]{0.07\columnwidth}\raggedleft\strut
1
\strut\end{minipage} &
\begin{minipage}[t]{0.08\columnwidth}\raggedleft\strut
1
\strut\end{minipage} &
\begin{minipage}[t]{0.09\columnwidth}\raggedleft\strut
1
\strut\end{minipage} &
\begin{minipage}[t]{0.10\columnwidth}\raggedleft\strut
1
\strut\end{minipage} &
\begin{minipage}[t]{0.11\columnwidth}\raggedleft\strut
1
\strut\end{minipage} &
\begin{minipage}[t]{0.07\columnwidth}\raggedleft\strut
34
\strut\end{minipage}\tabularnewline
\begin{minipage}[t]{0.11\columnwidth}\centering\strut
\textbf{F8}
\strut\end{minipage} &
\begin{minipage}[t]{0.07\columnwidth}\raggedleft\strut
1
\strut\end{minipage} &
\begin{minipage}[t]{0.08\columnwidth}\raggedleft\strut
1
\strut\end{minipage} &
\begin{minipage}[t]{0.09\columnwidth}\raggedleft\strut
1
\strut\end{minipage} &
\begin{minipage}[t]{0.10\columnwidth}\raggedleft\strut
1
\strut\end{minipage} &
\begin{minipage}[t]{0.11\columnwidth}\raggedleft\strut
0.98
\strut\end{minipage} &
\begin{minipage}[t]{0.07\columnwidth}\raggedleft\strut
17
\strut\end{minipage}\tabularnewline
\begin{minipage}[t]{0.11\columnwidth}\centering\strut
\textbf{F9}
\strut\end{minipage} &
\begin{minipage}[t]{0.07\columnwidth}\raggedleft\strut
1
\strut\end{minipage} &
\begin{minipage}[t]{0.08\columnwidth}\raggedleft\strut
1
\strut\end{minipage} &
\begin{minipage}[t]{0.09\columnwidth}\raggedleft\strut
1
\strut\end{minipage} &
\begin{minipage}[t]{0.10\columnwidth}\raggedleft\strut
1
\strut\end{minipage} &
\begin{minipage}[t]{0.11\columnwidth}\raggedleft\strut
1
\strut\end{minipage} &
\begin{minipage}[t]{0.07\columnwidth}\raggedleft\strut
20
\strut\end{minipage}\tabularnewline
\begin{minipage}[t]{0.11\columnwidth}\centering\strut
\textbf{F10}
\strut\end{minipage} &
\begin{minipage}[t]{0.07\columnwidth}\raggedleft\strut
1
\strut\end{minipage} &
\begin{minipage}[t]{0.08\columnwidth}\raggedleft\strut
1
\strut\end{minipage} &
\begin{minipage}[t]{0.09\columnwidth}\raggedleft\strut
1
\strut\end{minipage} &
\begin{minipage}[t]{0.10\columnwidth}\raggedleft\strut
1
\strut\end{minipage} &
\begin{minipage}[t]{0.11\columnwidth}\raggedleft\strut
1
\strut\end{minipage} &
\begin{minipage}[t]{0.07\columnwidth}\raggedleft\strut
34
\strut\end{minipage}\tabularnewline
\begin{minipage}[t]{0.11\columnwidth}\centering\strut
\textbf{F11}
\strut\end{minipage} &
\begin{minipage}[t]{0.07\columnwidth}\raggedleft\strut
1
\strut\end{minipage} &
\begin{minipage}[t]{0.08\columnwidth}\raggedleft\strut
1
\strut\end{minipage} &
\begin{minipage}[t]{0.09\columnwidth}\raggedleft\strut
1
\strut\end{minipage} &
\begin{minipage}[t]{0.10\columnwidth}\raggedleft\strut
1
\strut\end{minipage} &
\begin{minipage}[t]{0.11\columnwidth}\raggedleft\strut
1
\strut\end{minipage} &
\begin{minipage}[t]{0.07\columnwidth}\raggedleft\strut
33
\strut\end{minipage}\tabularnewline
\begin{minipage}[t]{0.11\columnwidth}\centering\strut
\textbf{F12}
\strut\end{minipage} &
\begin{minipage}[t]{0.07\columnwidth}\raggedleft\strut
1
\strut\end{minipage} &
\begin{minipage}[t]{0.08\columnwidth}\raggedleft\strut
1
\strut\end{minipage} &
\begin{minipage}[t]{0.09\columnwidth}\raggedleft\strut
1
\strut\end{minipage} &
\begin{minipage}[t]{0.10\columnwidth}\raggedleft\strut
1
\strut\end{minipage} &
\begin{minipage}[t]{0.11\columnwidth}\raggedleft\strut
1
\strut\end{minipage} &
\begin{minipage}[t]{0.07\columnwidth}\raggedleft\strut
34
\strut\end{minipage}\tabularnewline
\begin{minipage}[t]{0.11\columnwidth}\centering\strut
\textbf{F13}
\strut\end{minipage} &
\begin{minipage}[t]{0.07\columnwidth}\raggedleft\strut
1
\strut\end{minipage} &
\begin{minipage}[t]{0.08\columnwidth}\raggedleft\strut
1
\strut\end{minipage} &
\begin{minipage}[t]{0.09\columnwidth}\raggedleft\strut
1
\strut\end{minipage} &
\begin{minipage}[t]{0.10\columnwidth}\raggedleft\strut
1
\strut\end{minipage} &
\begin{minipage}[t]{0.11\columnwidth}\raggedleft\strut
1
\strut\end{minipage} &
\begin{minipage}[t]{0.07\columnwidth}\raggedleft\strut
34
\strut\end{minipage}\tabularnewline
\begin{minipage}[t]{0.11\columnwidth}\centering\strut
\textbf{F14}
\strut\end{minipage} &
\begin{minipage}[t]{0.07\columnwidth}\raggedleft\strut
1
\strut\end{minipage} &
\begin{minipage}[t]{0.08\columnwidth}\raggedleft\strut
1
\strut\end{minipage} &
\begin{minipage}[t]{0.09\columnwidth}\raggedleft\strut
1
\strut\end{minipage} &
\begin{minipage}[t]{0.10\columnwidth}\raggedleft\strut
1
\strut\end{minipage} &
\begin{minipage}[t]{0.11\columnwidth}\raggedleft\strut
1
\strut\end{minipage} &
\begin{minipage}[t]{0.07\columnwidth}\raggedleft\strut
33
\strut\end{minipage}\tabularnewline
\begin{minipage}[t]{0.11\columnwidth}\centering\strut
\textbf{F15}
\strut\end{minipage} &
\begin{minipage}[t]{0.07\columnwidth}\raggedleft\strut
1
\strut\end{minipage} &
\begin{minipage}[t]{0.08\columnwidth}\raggedleft\strut
1
\strut\end{minipage} &
\begin{minipage}[t]{0.09\columnwidth}\raggedleft\strut
1
\strut\end{minipage} &
\begin{minipage}[t]{0.10\columnwidth}\raggedleft\strut
1
\strut\end{minipage} &
\begin{minipage}[t]{0.11\columnwidth}\raggedleft\strut
1
\strut\end{minipage} &
\begin{minipage}[t]{0.07\columnwidth}\raggedleft\strut
26
\strut\end{minipage}\tabularnewline
\begin{minipage}[t]{0.11\columnwidth}\centering\strut
\textbf{F16}
\strut\end{minipage} &
\begin{minipage}[t]{0.07\columnwidth}\raggedleft\strut
0.01
\strut\end{minipage} &
\begin{minipage}[t]{0.08\columnwidth}\raggedleft\strut
0
\strut\end{minipage} &
\begin{minipage}[t]{0.09\columnwidth}\raggedleft\strut
0
\strut\end{minipage} &
\begin{minipage}[t]{0.10\columnwidth}\raggedleft\strut
0
\strut\end{minipage} &
\begin{minipage}[t]{0.11\columnwidth}\raggedleft\strut
0
\strut\end{minipage} &
\begin{minipage}[t]{0.07\columnwidth}\raggedleft\strut
13
\strut\end{minipage}\tabularnewline
\begin{minipage}[t]{0.11\columnwidth}\centering\strut
\textbf{F17}
\strut\end{minipage} &
\begin{minipage}[t]{0.07\columnwidth}\raggedleft\strut
0.77
\strut\end{minipage} &
\begin{minipage}[t]{0.08\columnwidth}\raggedleft\strut
0.74
\strut\end{minipage} &
\begin{minipage}[t]{0.09\columnwidth}\raggedleft\strut
0.72
\strut\end{minipage} &
\begin{minipage}[t]{0.10\columnwidth}\raggedleft\strut
0.71
\strut\end{minipage} &
\begin{minipage}[t]{0.11\columnwidth}\raggedleft\strut
0.67
\strut\end{minipage} &
\begin{minipage}[t]{0.07\columnwidth}\raggedleft\strut
12
\strut\end{minipage}\tabularnewline
\begin{minipage}[t]{0.11\columnwidth}\centering\strut
\textbf{F18}
\strut\end{minipage} &
\begin{minipage}[t]{0.07\columnwidth}\raggedleft\strut
0.83
\strut\end{minipage} &
\begin{minipage}[t]{0.08\columnwidth}\raggedleft\strut
0.83
\strut\end{minipage} &
\begin{minipage}[t]{0.09\columnwidth}\raggedleft\strut
0.83
\strut\end{minipage} &
\begin{minipage}[t]{0.10\columnwidth}\raggedleft\strut
0.8
\strut\end{minipage} &
\begin{minipage}[t]{0.11\columnwidth}\raggedleft\strut
0.8
\strut\end{minipage} &
\begin{minipage}[t]{0.07\columnwidth}\raggedleft\strut
16
\strut\end{minipage}\tabularnewline
\begin{minipage}[t]{0.11\columnwidth}\centering\strut
\textbf{F19}
\strut\end{minipage} &
\begin{minipage}[t]{0.07\columnwidth}\raggedleft\strut
0.44
\strut\end{minipage} &
\begin{minipage}[t]{0.08\columnwidth}\raggedleft\strut
0.44
\strut\end{minipage} &
\begin{minipage}[t]{0.09\columnwidth}\raggedleft\strut
0.42
\strut\end{minipage} &
\begin{minipage}[t]{0.10\columnwidth}\raggedleft\strut
0.42
\strut\end{minipage} &
\begin{minipage}[t]{0.11\columnwidth}\raggedleft\strut
0.4
\strut\end{minipage} &
\begin{minipage}[t]{0.07\columnwidth}\raggedleft\strut
13
\strut\end{minipage}\tabularnewline
\begin{minipage}[t]{0.11\columnwidth}\centering\strut
\textbf{F20}
\strut\end{minipage} &
\begin{minipage}[t]{0.07\columnwidth}\raggedleft\strut
0.15
\strut\end{minipage} &
\begin{minipage}[t]{0.08\columnwidth}\raggedleft\strut
0.14
\strut\end{minipage} &
\begin{minipage}[t]{0.09\columnwidth}\raggedleft\strut
0.14
\strut\end{minipage} &
\begin{minipage}[t]{0.10\columnwidth}\raggedleft\strut
0.14
\strut\end{minipage} &
\begin{minipage}[t]{0.11\columnwidth}\raggedleft\strut
0.12
\strut\end{minipage} &
\begin{minipage}[t]{0.07\columnwidth}\raggedleft\strut
12
\strut\end{minipage}\tabularnewline
\bottomrule
\end{longtable}

As a fourth measure, which wasn't introduced by the CEC committee, but
used in the original nmmso implementation (Fieldsend 2014) the Number of
Swarms was chosen. Since this is a continuous measure and therefore no
calculation is needed this measure is pictured as graphs. The graphs can
be found in Figure 1. The show the development of \(number of swarms\)
kept by nmmso.R over all iterations. Important to notice here is that
\(iterations\) is different from the \(evaluations\) referenced in the
other measures. Iterations are calls to start single runs of nmmso.R and
is therefore different from the evaluations taken within the program.

Additionally a fifth measure was introduced which denotes the runtime of
nmmso.R for the single functions. These times were taken on the ZIVHPC a
scientific High Perfomance Computing Cluster by Westfälische
Wilhelms-Universität Münster. Since the nmmso.R is a strictly sequential
algorithm the runtimes for single runs will comparable on common
computers. The ZIVHPC was only used to parallelize the single runs.

\begin{figure}[htbp]
\centering
\includegraphics{figure/trend\%20curve\%20of\%20kept\%20swarms\%20over\%20all\%2020\%20functions.\%20The\%20red\%20curves\%20show\%20the\%20number\%20of\%20swarms\%20kept\%20for\%20each\%20single\%20run.\%20The\%20black\%20line\%20shows\%20the\%20mean\%20of\%20kept\%20swarms\%20over\%20these\%20runs.-1.pdf}
\caption{plot of chunk trend curve of kept swarms over all 20 functions.
The red curves show the number of swarms kept for each single run. The
black line shows the mean of kept swarms over these runs.}
\end{figure}

When comparing those measures with the ones given in the original paper
(Fieldsend 2014) it can be seen that the reimplementation nmmso.R is an
overall good resemblance of the original algorithm. The three CEC
measures are close to the original taken measures and the trend curves
for the number of kept swarms have similar trends.

The biggest differences between the benchmarking results of the two
implementations can be seen in the general results of function 14, 15,
16 and 18, as well as in the number of created swarms for the
n-dimensional functions:

\begin{enumerate}
\def\labelenumi{(\arabic{enumi})}
\item
  Function 14 and 15 have a \(Success\ Ratio\) of \(1\) aswell as as
  \(Peak\) \(Ratio\) of one \(1\) for all accuracy levels. Additionally
  nmmso.R sometimes found all global optima for Function 18. In
  contradiction to that the evaluation of all thre function almost never
  result in the finding of global optima in the evaluation of the
  original implementation. Only at the lowest accuracy the original
  implementation is able to find all global optima for Function 14
  (Fieldsend 2014, p. 16). It is hard to say if this difference is equal
  to an error in the implementation of nmmso.R or if an error in the
  original implementation was fixed. Also this could be a difference in
  the reimplementation of the CEC Benchmarking Tool. Nevertheless, this
  is interesting point of discussion and worth evaluating.
\item
  nmmso.R performs noticeable worse for Function 16 than the original
  function. While nmmso.R has a \(Peak\) \(Ratio\) of \(0.01\) for an
  accuracy of \(0.1\) and \(0\) for all others, the original
  implementation reaches a \(Peak\) \(Ratio\) of around \(0.6\) for all
  accuracies. This might be to an implementation error in the CEC
  Benchmarking Tool. Since it is so significantly worse that it is
  unlikely that this difference would only occur in one test function.
\item
  Almost all algorithm runs on high-dimensional functions (F12-F20)
  result in a high number of swarms, while all other results regarding
  this functions are comparable to the original results. This difference
  becomes very clear in the case of Functions 17-20. In the paper
  addressing the original paper the x-axis rank from 0-40,000
  iterations, while for the reimplementation limit of 4,000 for Function
  17, of 20,000 for Function 18, 6,000 for Function 19 and of 30,000 for
  Function 20 is enough to show all data sets. This is connected to the
  creation of much more swarms, which leads to an earlier depletion of
  the maximum allowed number of evaluations.
\end{enumerate}

\section{Conclusion}\label{conclusion}

test

\section{Acknowledgements}\label{acknowledgements}

We want to thank Dr.~Jonathan Fieldsend for his continuous help via mail
during this seminar. Also the committee of the CEC was always available
for questions and concerns during our work. Furthermore a special thanks
goes to all employees of the chair for `Information Systems and
Statistics' including Dr.~Mike Preuß, Jakob Bossek and Pascal Kerschke
who were available for any questions regarding the implementation and
this report at all times. \newpage

\hypertarget{refs}{}
\hypertarget{ref-epitropakisux5f2013}{}
Epitropakis, M. G., Li, X., and Burke, E. K. 2013. ``A dynamic archive
niching differential evolution algorithm for multimodal optimization,''
in \emph{Evolutionary computation (cEC), 2013 iEEE congress on}, IEEE,
pp. 79--86.

\hypertarget{ref-fieldsendux5f2014}{}
Fieldsend, J. E. 2014. ``Running up those hills: Multi-modal search with
the niching migratory multi-swarm optimiser,'' in \emph{Evolutionary
computation (cEC), 2014 iEEE congress on}, IEEE, pp. 2593--2600.

\hypertarget{ref-liux5f2013}{}
Li, X., Engelbrecht, A., and Epitropakis, M. G. 2013. ``Benchmark
functions for cEC'2013 special session and competition on niching
methods for multimodal function optimization,'' \emph{RMIT University,
Evolutionary Computation and Machine Learning Group, Australia, Tech.
Rep}.

\hypertarget{ref-preussux5f2012}{}
Preuss, M. 2012. ``Improved topological niching for real-valued global
optimization,'' in \emph{Applications of evolutionary computation},
Springer, pp. 386--395.

\hypertarget{ref-yangux5f2009}{}
Yang, X.-S. 2009. ``Firefly algorithms for multimodal optimization,'' in
\emph{Stochastic algorithms: Foundations and applications}, Springer,
pp. 169--178.


\end{document}
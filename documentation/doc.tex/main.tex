%!TEX TS-program = /usr/texbin/pdflatex
%!GEDIT texbin = /usr/local/texlive/current/bin/x86_64-linux/pdflatex
%
%   ''Becker-Vorlage'' style LaTeX template

%   This template is  MIT licensed, authors:
%   Jan Betzing <jan.betzing@ercis.uni-muenster.de> *corresponding author
%   Dominik Lekse <dominik@lekse.de>
%
%   Basic file to demonstrate the usage of this LaTeX template.
%   You can build your own paper/thesis on top of this file.
%   Simply adjust the document class and all metadata and start working.
%
\documentclass[
    language=english, % set to english or german
    type=seminar % set to bachelor, master or seminar
]{isthesis}

% Graphics rendering using TikZ
% See: https://en.wikibooks.org/wiki/LaTeX/PGF/TikZ
\usepackage{tikz}
% Include required TikZ libraries here, some exemplary libraries are pre-included
\usetikzlibrary{calc}
\usetikzlibrary{matrix}
\usetikzlibrary{positioning}
\usetikzlibrary{shapes.geometric}

% Import acronyms
\input{acronyms}

% Import symbols
\input{symbols}

% Document meta information
\isthesis{
    title={Implementation of nmmso},
    author={Daniel Carriola},             
    author-email={student@uni-muenster.de},
    author-phone={+49 251 8338100}, % Use international numbers format
    author-matriculation={123456},
    author-address={Schlossplatz 2},
    author-zip={48149},
    author-city={M\"{u}nster},
    author_2={Daniel Carriola},             
    author_2-email={student@uni-muenster.de},
    author_2-phone={+49 251 8338100}, % Use international numbers format
    author_2-matriculation={123456},
    author_2-address={Schlossplatz 2},
    author_2-zip={48149},
    author_2-city={M\"{u}nster},
    principal-supervisor={Prof Dr. Heike Trautmann}, % This has to be a professor
    associate-supervisor={Mike Preuss}, % This is your main supervisor, i.e., a post doc or PhD student
    tutor-supervisor={}, % If required, define an additional supervisor resp. tutor here
    group={Something, something Statistics},
    group-institute={Westphalian Wilhelms-University, M\"unster},
    %associate-group={}, % When the thesis is done in cooperation with another chair, add it here
    %associate-group-institute={}, % add cooperating institute or university here
    seminar={Statistical Programming in R}, % The title of your seminar
    submission-date={2015-01-05} % The date you handed in your document
    %primary-logo={}, % Uses the WWU logo by default
    %primary-logo-height={}, % Uses 16mm as default height
    %secondary-logo={}, % Logo of the secondary institution (cooperating chair/university), USES Faculty logo by default
    %secondary-logo-height={} % Uses 16mm as default height
}
\begin{document}
    % Title page
    \maketitle

    % Quote
    % You can put an optional quote page in front of your content
   \quotepage[author={Arthur C. Clarke}]{
            Any sufficiently advanced technology is indistinguishable from magic.
   }

    % Table of contents
    \tableofcontents

    % List of figures (if you have figures)
    \listoffigures

    % List of tables (if you have tables)
    \listoftables
    
    % List of listings (if you have listings)
    \lstlistoflistings

    % List of abbreviations (if you use acronyms)
    \listofabbreviations

    % List of symbols (if you use symbols)
    \listofsymbols

    % Abstract
    %
    % Comment out this part, if you don't require an abstract
    % \begin{abstract}
    %   \input{abstract}
    % \end{abstract}

    % Content
    \begin{content}
        % LaTeX Template for Project Report, Version 2.0
% (Abstracted from a Major Project Report at CSED, NIT Calicut but can be
% modified easily to use for other reports also.)
%
% Released under Creative Commons Attribution license (CC-BY)
% Info: http://creativecommons.org/licenses/by/3.0/
%
% Created by: Kartik Singhal
% BTech CSE Batch of 2009-13
% NIT Calicut
% Contact Info: kartiksinghal@gmail.com
%
% It is advisable to learn the basics of LaTeX before using this template.
% A good resource to start with is http://en.wikibooks.org/wiki/LaTeX/
%
% All template fields are marked with a pair of angular brackets e.g. <title here>
% except for the ones defining citation names in ref.tex.
%
% Empty space after chapter/section/subsection titles can be used to insert text.
%
% Just compile this file using pdflatex after making all required changes.

\documentclass[12pt,a4paper]{article}

\RequirePackage{fancyhdr}
\RequirePackage{lastpage}

\setlength\textwidth{165mm}
\setlength\textheight{240mm}
\setlength\topmargin{-10mm}
\setlength\oddsidemargin{0mm}
\setlength\parindent{0pt}
\setlength\parskip{1.7\medskipamount}

\sloppy\pagestyle{fancy}


%% rest

\usepackage{url} %for proper url entries
\usepackage[numbers]{natbib}
% for nice tables
\usepackage{longtable}
\usepackage{booktabs}
\usepackage{floatrow}
\floatsetup[table]{capposition=bottom}
% for nice code

\usepackage{amssymb,amsmath}
\usepackage{ifxetex,ifluatex}
\ifxetex
  \usepackage{fontspec,xltxtra,xunicode}
  \defaultfontfeatures{Mapping=tex-text,Scale=MatchLowercase}
\else
  \ifluatex
    \usepackage{fontspec}
    \defaultfontfeatures{Mapping=tex-text,Scale=MatchLowercase}
  \else
    \usepackage[utf8]{inputenc}
  \fi
\fi

\usepackage{color}
\usepackage{fancyvrb}
\DefineShortVerb[commandchars=\\\{\}]{\|}
\DefineVerbatimEnvironment{Highlighting}{Verbatim}{commandchars=\\\{\}}
% Add ',fontsize=\small' for more characters per line
\newenvironment{Shaded}{}{}
\newcommand{\KeywordTok}[1]{\textcolor[rgb]{0.00,0.44,0.13}{\textbf{{#1}}}}
\newcommand{\DataTypeTok}[1]{\textcolor[rgb]{0.56,0.13,0.00}{{#1}}}
\newcommand{\DecValTok}[1]{\textcolor[rgb]{0.25,0.63,0.44}{{#1}}}
\newcommand{\BaseNTok}[1]{\textcolor[rgb]{0.25,0.63,0.44}{{#1}}}
\newcommand{\FloatTok}[1]{\textcolor[rgb]{0.25,0.63,0.44}{{#1}}}
\newcommand{\CharTok}[1]{\textcolor[rgb]{0.25,0.44,0.63}{{#1}}}
\newcommand{\StringTok}[1]{\textcolor[rgb]{0.25,0.44,0.63}{{#1}}}
\newcommand{\CommentTok}[1]{\textcolor[rgb]{0.38,0.63,0.69}{\textit{{#1}}}}
\newcommand{\OtherTok}[1]{\textcolor[rgb]{0.00,0.44,0.13}{{#1}}}
\newcommand{\AlertTok}[1]{\textcolor[rgb]{1.00,0.00,0.00}{\textbf{{#1}}}}
\newcommand{\FunctionTok}[1]{\textcolor[rgb]{0.02,0.16,0.49}{{#1}}}
\newcommand{\RegionMarkerTok}[1]{{#1}}
\newcommand{\ErrorTok}[1]{\textcolor[rgb]{1.00,0.00,0.00}{\textbf{{#1}}}}
\newcommand{\NormalTok}[1]{{#1}}
\usepackage[pdftex]{graphicx}
\setkeys{Gin}{width=\textwidth}
\usepackage[Export]{adjustbox}
\usepackage[unicode=true]{hyperref}
\hypersetup{breaklinks=true, pdfborder={0 0 0}}
\setlength{\parindent}{0pt}
\setlength{\parskip}{6pt plus 2pt minus 1pt}
\setlength{\emergencystretch}{3em}  % prevent overfull lines
%\setcounter{secnumdepth}{0}


\begin{document}
\renewcommand\refname{References} %Renames "Bibliography" to "References" on ref page

%include other pages
\begin{titlepage}

\begin{center}

\textup{\small {\bf Statistical Computing in R} \\ Report}\\[0.2in]

% Title
\Large \textbf {Implementation of NMMSO in R}\\[0.5in]

 % Submitted by
\normalsize Submitted by \\
\begin{table}[h]
\centering
\begin{tabular}{lr}
425699 & Jannik Hoffjann \\
425699 & Daniel Carriola \\ 
\end{tabular}
\end{table}

\vspace{.1in}
Under the guidance of\\
{\textbf{Dr. Mike Preuss}}\\[0.2in]

\vfill

% Bottom of the page
\includegraphics[width=0.18\textwidth]{./assets/wwu-logo}\\[0.1in]
\Large{Information Systems and Statistics}\\
\normalsize
\textsc{ERCIS}\\
Münster - NRW - Germany \\
\vspace{0.2cm}
Winter Semester 2015/16

\end{center}

\end{titlepage}

\newpage
\pagenumbering{roman} %numbering before main content starts
\tableofcontents
\newpage
\listoffigures

\newpage
\pagenumbering{arabic} %reset numbering to normal for the main content

\section{Introduction}\label{introduction}

In the recent years R has become the statistical programming language of
choice for many scientist. The strength of R of being a domain specific
language has also become one of its weaknesses. Since new research
findings in statistical computing are split up over several languages
like R, Matlab or SciPy\footnote{SciPy is a common library for the
  Python Programming language which brings Statistical Computing
  capabilities to the language. \newpage} it often becomes difficult to
compare new methods with established ones. Since it is also hard to
interface those languages due to different architectures, data storage
mechanisms there is often no other way than to reimplement new methods
in a different programming language to create a common scope.

An example for a well perceived new finding in statistical computing is
the NMMSO-Algorithm by Jonathan E. Fieldsend (Fieldsend 2014). It won
the niching competition in 2015 held by the CEC and is only written in
Matlab. Since the chair `Information Systems and Statistics' at the
Westfälische Wilhelms-Universität Münster, Germany is mainly
concentrating its work on Statistical Computing in R an implementation
of this algorithm became interesting.

As part of this Seminar Project in the context of the Seminar
`Statistical Computing in R' a reimplementation of the NMMSO algorithm
in R will be presented. During this technical documentation, the general
function of the algorithm and the used test cases by the CEC will be
shown. Afterwards the structure and used techniques and libraries, as
well as problems and pitfalls due to the different behaviours of R and
Matlab, will be shown. The documentation will be closed by the
benchmarking results and different test cases.

It was the goal of this project to keep up high comparability with the
original code, to ensure the correct functionality and easily implement
changes to the original codebase in this program.

\textbf{\ldots{} write a bit more here.}

\section{General Function}\label{general-function}

The starting point of the project was the paper provided by Dr.~Jonathen
E. Fieldsend (Fieldsend 2014) on the Niching Migratory Multi-Swarm
Optimiser (NMMSO) algorithm. NMMSO is a multi-modal optimiser which
relies heavily on multiple swarms which are generated on the landscape
of an function in order to find the global optimum. It is build around
three main pillars: (1) dynamic in the numbers of dimensions, (2)
self-adaptive without any special preparation and (3) exploitative local
search to quickly find peak estimates (Fieldsend 2014, 1).

Multi-modal optimisation in general is not that different from well
known and widely discussed single-objective optimisation, but in
difference to it the goal of the algorithms in the multi-modal is not to
find just one single optimising point but all possible points (Fieldsend
2014, 1). In order to do so, many early multi-modal optimisation
algorithms needed highly defined parameters {[}TODO: quote needed{]}.

\textbf{maybe it would be interesting to write a few more lines about
the history of evolutionary algorithms here?}

Newer algorithms fall in the field of self-tuning and try to use
different mathematical paradigms like nearest-best clustering with
covariance matrices (Preuss 2010) and strategies like storing the so far
best found global optima estimators to provide them as parameters for
new optimisation runs (Epitropakis, Li, and Burke 2013). Contradictory
to that NMMSO goes another way and uses the the swarm strategy in order
to find which store their current (Fieldsend 2014)

In order to do so NMMSO follow a strict structure which can be seen in
the following pseudo-code

\begin{verbatim}
nmmso(max_evals, tol, n, max_inc, c_1, c_2, chi, w)
    S: initialise_swarm(1)
    evaluations := 1
    while evaluations < max_evals:
        while flagged_swarms(S) == true:
            {S, m} := attempt_merge(S, n, tol)
            evals := evals + m
        S := increment(S, n, max_inc, c_1, c_2, chi, w)
        evals := evals + min(|S|, max_inc)
        {S, k} := attempt_separation(S, tol)
        evals := evals + k
        S := add_new_swarm(S)
        evals := evals + 1
    {X*, Y*} := extract_gebsest(S)
    return X*,Y*
\end{verbatim}

This structure wasn't modified during the reimplementation of NMMSO to
keep comparability and the possibility to fix bugs at a high level. The
only newly introduced setting was the possibility to modify the c\_1,
c\_2, chi, w as parameters from the outside. In the original version
those parameters are part of the program code.

\textbf{What else about the algorithm need to be explained that isn't
explicitly part of the implementation?}

\begin{center}\rule{0.5\linewidth}{\linethickness}\end{center}

\section{CEC Algrithms}\label{cec-algrithms}

\subsection{CEC}\label{cec}

The IEEE Congress of Evolutionary Computation (CEC) is one of the
largest, most important and recognised conferences within Evolutionary
Computation (EC). It is organised by the IEEE Computational Intelligence
Society in cooperation with the Evolutionary Programming Society, and
covers most of the subtopics of the EC.

In order to validate the potential of the NMMSO algorithm, it was
submitted to the IEEE CEC 2013 held in Cancun, Mexico. Here,
Dr.~Fieldsend was provided with some multimodal benchmark test functions
with different dimension sizes and characteristics, for evaluating
niching algorithms developed by Dr.~Xiaodong Li, Dr.~Andries Engelbrecht
and Dr.~Michael G. Epitropakis (Epitropakis, Li, and Burke 2013). They
state that even if several niching methods have been around for many
years, further advances in this area have been hindered by several
obstacles; most of the studies focus on very low dimensional multimodal
problems (2 or 3 dimensions) making this more complicated to asses
theses methods' scalability to high dimensions with better performance.
The benchmark tool includes 20 test functions (in some cases the same
function but with different dimension sizes), which includes 10 simple,
well-known and widely used benchmark functions, based on recent studies,
and more complex functions following the paradigm of composition
functions. In the following section, they will be briefly explained:

\begin{verbatim}
•   F1: Five-Uneven-Peak Trap (1D)
•   F2: Equal Maxima (1D)
•   F3: Uneven Decreasing Maxima (1D)
•   F4: Himmelblau (2D)
•   F5: Six-Hump Camel Back (2D)
•   F6: Shubert (2D, 3D)
•   F7: Vincent (2D, 3D)
•   F8: Modified Rastrigin - All Global Optima (2D)
•   F9: Composition Function 1 (2D)
•   F10: Composition Function 2 (2D)
•   F11: Composition Function 3 (2D, 3D, 5D, 10D)
•   F12: Composition Function 4 (3D, 5D, 10D, 20D)
\end{verbatim}

All of the test functions are formulated as maximisation problems. F1,
F2 and F3 are simple 1D multimodal functions, while F4 and F5 are simple
2D functions and not scalable. F6 to F8 are scalable multimodal
functions. The number of global optima for F6 and F7 are determined by
the dimension. However, for F8, the number of global optima is
independent from the dimension, therefore it can be controlled by the
user. F9 to F12 are scalable multimodal functions constructed by several
basic functions with different properties (Sphere function, Grienwank,
Rastrigin, Weierstrass and the Expanded Griewank's plus Rosenbrock's
function). F9 and F10 are separable, and non-symmetric, while F11 and
F12 are non-separable, non-symmetric complex multimodal functions. The
number of global optima in all of the composition functions is
independent from the number of dimensions, therefore can be controlled
by the user (Epitropakis, Li, and Burke 2013).

\textbf{Maybe write each math equation or the R code}

https://en.wikipedia.org/wiki/IEEE\_Congress\_on\_Evolutionary\_Computation

\subsection{Implementation and
Pitfalls}\label{implementation-and-pitfalls}

\textbf{write also about the count\_goptima and so on}

\begin{center}\rule{0.5\linewidth}{\linethickness}\end{center}

\section{The Implementation}\label{the-implementation}

\subsection{Structure of the project}\label{structure-of-the-project}

After analysing the algorithm provided in Matlab by Dr.~Fieldsend, it
was decided to first translate each of the functions into the R
programming language. At first instance, this task seemed to be simple
because most of the functions were basically managing matrices and
vectors, but later this became a problem that will be addressed in the
pitfalls' section of this paper.

Once all the NMMSO functions existed in R and having the input data, the
testing phase started. It has be said, that one of the biggest problems
when you code an already existing program into another programming
language, is the different behaviours corresponding to each object (in
case of an object-oriented language) or its main structure. The first
runs came with several errors regarding the matrix generation and
handling, slowing down the project in a near future. Using GitHub, it
was easier to attack these problems in parallel, having one developer
reviewing different functions and the other one, fixing other bugs and
continue the testing phase. Also, this was achieved in an easier way,
thanks that each function was coded in an independent R file, making
easier and faster the debugging and the fixing of each problem.

During the developing time, an issue raised with the CEC benchmark tool.
In order to compare the R implementation of the NMMSO algorithm with the
original one, it was mandatory to use this tool to test each of its
functions with the new algorithm and compare results. After several
complications with the original test suite (these complications will be
addressed in the pitfalls' section), it was decided to recode each of
the functions as an independent R package to avoid any further
complication and having an easier and more trustworthy comparison of the
NMMSO algorithm in R.

\subsection{Pitfalls and Problems}\label{pitfalls-and-problems}

test

\section{Benchmark and Comparison}\label{benchmark-and-comparison}

To compare the nmmsoR with the original NMMSO the CEC test cases were
used to run the same benchmarks as in the original submission (Fieldsend
2014). There 4 different Ratios were used to measure the performance of
certain algorithms. Three of those measures (Peak Ratio, Success Ratio
and Convergence Speed) have been introduced in (Epitropakis, Li, and
Burke 2013, 6--7) to create a common point of comparison. The fourth
ratio is special for the nmmso algorithm since it tracks the number of
swarms over the iterations of the algorithm. Nmmso.R uses the same
measures to reach the highest comparability possible.

\begin{longtable}[c]{@{}crrrrrr@{}}
\caption{Success Ratio over given runs}\tabularnewline
\toprule
\begin{minipage}[b]{0.11\columnwidth}\centering\strut
~
\strut\end{minipage} &
\begin{minipage}[b]{0.07\columnwidth}\raggedleft\strut
0.1
\strut\end{minipage} &
\begin{minipage}[b]{0.08\columnwidth}\raggedleft\strut
0.01
\strut\end{minipage} &
\begin{minipage}[b]{0.09\columnwidth}\raggedleft\strut
0.001
\strut\end{minipage} &
\begin{minipage}[b]{0.10\columnwidth}\raggedleft\strut
0.0001
\strut\end{minipage} &
\begin{minipage}[b]{0.11\columnwidth}\raggedleft\strut
0.00001
\strut\end{minipage} &
\begin{minipage}[b]{0.07\columnwidth}\raggedleft\strut
runs
\strut\end{minipage}\tabularnewline
\midrule
\endfirsthead
\toprule
\begin{minipage}[b]{0.11\columnwidth}\centering\strut
~
\strut\end{minipage} &
\begin{minipage}[b]{0.07\columnwidth}\raggedleft\strut
0.1
\strut\end{minipage} &
\begin{minipage}[b]{0.08\columnwidth}\raggedleft\strut
0.01
\strut\end{minipage} &
\begin{minipage}[b]{0.09\columnwidth}\raggedleft\strut
0.001
\strut\end{minipage} &
\begin{minipage}[b]{0.10\columnwidth}\raggedleft\strut
0.0001
\strut\end{minipage} &
\begin{minipage}[b]{0.11\columnwidth}\raggedleft\strut
0.00001
\strut\end{minipage} &
\begin{minipage}[b]{0.07\columnwidth}\raggedleft\strut
runs
\strut\end{minipage}\tabularnewline
\midrule
\endhead
\begin{minipage}[t]{0.11\columnwidth}\centering\strut
\textbf{F1}
\strut\end{minipage} &
\begin{minipage}[t]{0.07\columnwidth}\raggedleft\strut
1
\strut\end{minipage} &
\begin{minipage}[t]{0.08\columnwidth}\raggedleft\strut
1
\strut\end{minipage} &
\begin{minipage}[t]{0.09\columnwidth}\raggedleft\strut
1
\strut\end{minipage} &
\begin{minipage}[t]{0.10\columnwidth}\raggedleft\strut
1
\strut\end{minipage} &
\begin{minipage}[t]{0.11\columnwidth}\raggedleft\strut
1
\strut\end{minipage} &
\begin{minipage}[t]{0.07\columnwidth}\raggedleft\strut
32
\strut\end{minipage}\tabularnewline
\begin{minipage}[t]{0.11\columnwidth}\centering\strut
\textbf{F2}
\strut\end{minipage} &
\begin{minipage}[t]{0.07\columnwidth}\raggedleft\strut
1
\strut\end{minipage} &
\begin{minipage}[t]{0.08\columnwidth}\raggedleft\strut
1
\strut\end{minipage} &
\begin{minipage}[t]{0.09\columnwidth}\raggedleft\strut
1
\strut\end{minipage} &
\begin{minipage}[t]{0.10\columnwidth}\raggedleft\strut
1
\strut\end{minipage} &
\begin{minipage}[t]{0.11\columnwidth}\raggedleft\strut
1
\strut\end{minipage} &
\begin{minipage}[t]{0.07\columnwidth}\raggedleft\strut
30
\strut\end{minipage}\tabularnewline
\begin{minipage}[t]{0.11\columnwidth}\centering\strut
\textbf{F3}
\strut\end{minipage} &
\begin{minipage}[t]{0.07\columnwidth}\raggedleft\strut
1
\strut\end{minipage} &
\begin{minipage}[t]{0.08\columnwidth}\raggedleft\strut
1
\strut\end{minipage} &
\begin{minipage}[t]{0.09\columnwidth}\raggedleft\strut
1
\strut\end{minipage} &
\begin{minipage}[t]{0.10\columnwidth}\raggedleft\strut
1
\strut\end{minipage} &
\begin{minipage}[t]{0.11\columnwidth}\raggedleft\strut
1
\strut\end{minipage} &
\begin{minipage}[t]{0.07\columnwidth}\raggedleft\strut
32
\strut\end{minipage}\tabularnewline
\begin{minipage}[t]{0.11\columnwidth}\centering\strut
\textbf{F4}
\strut\end{minipage} &
\begin{minipage}[t]{0.07\columnwidth}\raggedleft\strut
1
\strut\end{minipage} &
\begin{minipage}[t]{0.08\columnwidth}\raggedleft\strut
1
\strut\end{minipage} &
\begin{minipage}[t]{0.09\columnwidth}\raggedleft\strut
1
\strut\end{minipage} &
\begin{minipage}[t]{0.10\columnwidth}\raggedleft\strut
1
\strut\end{minipage} &
\begin{minipage}[t]{0.11\columnwidth}\raggedleft\strut
1
\strut\end{minipage} &
\begin{minipage}[t]{0.07\columnwidth}\raggedleft\strut
32
\strut\end{minipage}\tabularnewline
\begin{minipage}[t]{0.11\columnwidth}\centering\strut
\textbf{F5}
\strut\end{minipage} &
\begin{minipage}[t]{0.07\columnwidth}\raggedleft\strut
1
\strut\end{minipage} &
\begin{minipage}[t]{0.08\columnwidth}\raggedleft\strut
1
\strut\end{minipage} &
\begin{minipage}[t]{0.09\columnwidth}\raggedleft\strut
1
\strut\end{minipage} &
\begin{minipage}[t]{0.10\columnwidth}\raggedleft\strut
1
\strut\end{minipage} &
\begin{minipage}[t]{0.11\columnwidth}\raggedleft\strut
1
\strut\end{minipage} &
\begin{minipage}[t]{0.07\columnwidth}\raggedleft\strut
29
\strut\end{minipage}\tabularnewline
\begin{minipage}[t]{0.11\columnwidth}\centering\strut
\textbf{F6}
\strut\end{minipage} &
\begin{minipage}[t]{0.07\columnwidth}\raggedleft\strut
1
\strut\end{minipage} &
\begin{minipage}[t]{0.08\columnwidth}\raggedleft\strut
1
\strut\end{minipage} &
\begin{minipage}[t]{0.09\columnwidth}\raggedleft\strut
1
\strut\end{minipage} &
\begin{minipage}[t]{0.10\columnwidth}\raggedleft\strut
1
\strut\end{minipage} &
\begin{minipage}[t]{0.11\columnwidth}\raggedleft\strut
0
\strut\end{minipage} &
\begin{minipage}[t]{0.07\columnwidth}\raggedleft\strut
29
\strut\end{minipage}\tabularnewline
\begin{minipage}[t]{0.11\columnwidth}\centering\strut
\textbf{F7}
\strut\end{minipage} &
\begin{minipage}[t]{0.07\columnwidth}\raggedleft\strut
1
\strut\end{minipage} &
\begin{minipage}[t]{0.08\columnwidth}\raggedleft\strut
1
\strut\end{minipage} &
\begin{minipage}[t]{0.09\columnwidth}\raggedleft\strut
1
\strut\end{minipage} &
\begin{minipage}[t]{0.10\columnwidth}\raggedleft\strut
1
\strut\end{minipage} &
\begin{minipage}[t]{0.11\columnwidth}\raggedleft\strut
1
\strut\end{minipage} &
\begin{minipage}[t]{0.07\columnwidth}\raggedleft\strut
29
\strut\end{minipage}\tabularnewline
\begin{minipage}[t]{0.11\columnwidth}\centering\strut
\textbf{F8}
\strut\end{minipage} &
\begin{minipage}[t]{0.07\columnwidth}\raggedleft\strut
1
\strut\end{minipage} &
\begin{minipage}[t]{0.08\columnwidth}\raggedleft\strut
1
\strut\end{minipage} &
\begin{minipage}[t]{0.09\columnwidth}\raggedleft\strut
1
\strut\end{minipage} &
\begin{minipage}[t]{0.10\columnwidth}\raggedleft\strut
0.91
\strut\end{minipage} &
\begin{minipage}[t]{0.11\columnwidth}\raggedleft\strut
0.73
\strut\end{minipage} &
\begin{minipage}[t]{0.07\columnwidth}\raggedleft\strut
11
\strut\end{minipage}\tabularnewline
\begin{minipage}[t]{0.11\columnwidth}\centering\strut
\textbf{F9}
\strut\end{minipage} &
\begin{minipage}[t]{0.07\columnwidth}\raggedleft\strut
1
\strut\end{minipage} &
\begin{minipage}[t]{0.08\columnwidth}\raggedleft\strut
1
\strut\end{minipage} &
\begin{minipage}[t]{0.09\columnwidth}\raggedleft\strut
1
\strut\end{minipage} &
\begin{minipage}[t]{0.10\columnwidth}\raggedleft\strut
1
\strut\end{minipage} &
\begin{minipage}[t]{0.11\columnwidth}\raggedleft\strut
1
\strut\end{minipage} &
\begin{minipage}[t]{0.07\columnwidth}\raggedleft\strut
15
\strut\end{minipage}\tabularnewline
\begin{minipage}[t]{0.11\columnwidth}\centering\strut
\textbf{F10}
\strut\end{minipage} &
\begin{minipage}[t]{0.07\columnwidth}\raggedleft\strut
1
\strut\end{minipage} &
\begin{minipage}[t]{0.08\columnwidth}\raggedleft\strut
1
\strut\end{minipage} &
\begin{minipage}[t]{0.09\columnwidth}\raggedleft\strut
1
\strut\end{minipage} &
\begin{minipage}[t]{0.10\columnwidth}\raggedleft\strut
1
\strut\end{minipage} &
\begin{minipage}[t]{0.11\columnwidth}\raggedleft\strut
1
\strut\end{minipage} &
\begin{minipage}[t]{0.07\columnwidth}\raggedleft\strut
29
\strut\end{minipage}\tabularnewline
\begin{minipage}[t]{0.11\columnwidth}\centering\strut
\textbf{F11}
\strut\end{minipage} &
\begin{minipage}[t]{0.07\columnwidth}\raggedleft\strut
1
\strut\end{minipage} &
\begin{minipage}[t]{0.08\columnwidth}\raggedleft\strut
1
\strut\end{minipage} &
\begin{minipage}[t]{0.09\columnwidth}\raggedleft\strut
1
\strut\end{minipage} &
\begin{minipage}[t]{0.10\columnwidth}\raggedleft\strut
1
\strut\end{minipage} &
\begin{minipage}[t]{0.11\columnwidth}\raggedleft\strut
1
\strut\end{minipage} &
\begin{minipage}[t]{0.07\columnwidth}\raggedleft\strut
29
\strut\end{minipage}\tabularnewline
\begin{minipage}[t]{0.11\columnwidth}\centering\strut
\textbf{F12}
\strut\end{minipage} &
\begin{minipage}[t]{0.07\columnwidth}\raggedleft\strut
1
\strut\end{minipage} &
\begin{minipage}[t]{0.08\columnwidth}\raggedleft\strut
1
\strut\end{minipage} &
\begin{minipage}[t]{0.09\columnwidth}\raggedleft\strut
1
\strut\end{minipage} &
\begin{minipage}[t]{0.10\columnwidth}\raggedleft\strut
1
\strut\end{minipage} &
\begin{minipage}[t]{0.11\columnwidth}\raggedleft\strut
1
\strut\end{minipage} &
\begin{minipage}[t]{0.07\columnwidth}\raggedleft\strut
28
\strut\end{minipage}\tabularnewline
\begin{minipage}[t]{0.11\columnwidth}\centering\strut
\textbf{F13}
\strut\end{minipage} &
\begin{minipage}[t]{0.07\columnwidth}\raggedleft\strut
1
\strut\end{minipage} &
\begin{minipage}[t]{0.08\columnwidth}\raggedleft\strut
1
\strut\end{minipage} &
\begin{minipage}[t]{0.09\columnwidth}\raggedleft\strut
1
\strut\end{minipage} &
\begin{minipage}[t]{0.10\columnwidth}\raggedleft\strut
1
\strut\end{minipage} &
\begin{minipage}[t]{0.11\columnwidth}\raggedleft\strut
1
\strut\end{minipage} &
\begin{minipage}[t]{0.07\columnwidth}\raggedleft\strut
28
\strut\end{minipage}\tabularnewline
\begin{minipage}[t]{0.11\columnwidth}\centering\strut
\textbf{F14}
\strut\end{minipage} &
\begin{minipage}[t]{0.07\columnwidth}\raggedleft\strut
1
\strut\end{minipage} &
\begin{minipage}[t]{0.08\columnwidth}\raggedleft\strut
1
\strut\end{minipage} &
\begin{minipage}[t]{0.09\columnwidth}\raggedleft\strut
1
\strut\end{minipage} &
\begin{minipage}[t]{0.10\columnwidth}\raggedleft\strut
1
\strut\end{minipage} &
\begin{minipage}[t]{0.11\columnwidth}\raggedleft\strut
1
\strut\end{minipage} &
\begin{minipage}[t]{0.07\columnwidth}\raggedleft\strut
27
\strut\end{minipage}\tabularnewline
\begin{minipage}[t]{0.11\columnwidth}\centering\strut
\textbf{F15}
\strut\end{minipage} &
\begin{minipage}[t]{0.07\columnwidth}\raggedleft\strut
1
\strut\end{minipage} &
\begin{minipage}[t]{0.08\columnwidth}\raggedleft\strut
1
\strut\end{minipage} &
\begin{minipage}[t]{0.09\columnwidth}\raggedleft\strut
1
\strut\end{minipage} &
\begin{minipage}[t]{0.10\columnwidth}\raggedleft\strut
1
\strut\end{minipage} &
\begin{minipage}[t]{0.11\columnwidth}\raggedleft\strut
1
\strut\end{minipage} &
\begin{minipage}[t]{0.07\columnwidth}\raggedleft\strut
20
\strut\end{minipage}\tabularnewline
\begin{minipage}[t]{0.11\columnwidth}\centering\strut
\textbf{F16}
\strut\end{minipage} &
\begin{minipage}[t]{0.07\columnwidth}\raggedleft\strut
0
\strut\end{minipage} &
\begin{minipage}[t]{0.08\columnwidth}\raggedleft\strut
0
\strut\end{minipage} &
\begin{minipage}[t]{0.09\columnwidth}\raggedleft\strut
0
\strut\end{minipage} &
\begin{minipage}[t]{0.10\columnwidth}\raggedleft\strut
0
\strut\end{minipage} &
\begin{minipage}[t]{0.11\columnwidth}\raggedleft\strut
0
\strut\end{minipage} &
\begin{minipage}[t]{0.07\columnwidth}\raggedleft\strut
8
\strut\end{minipage}\tabularnewline
\begin{minipage}[t]{0.11\columnwidth}\centering\strut
\textbf{F17}
\strut\end{minipage} &
\begin{minipage}[t]{0.07\columnwidth}\raggedleft\strut
0.14
\strut\end{minipage} &
\begin{minipage}[t]{0.08\columnwidth}\raggedleft\strut
0
\strut\end{minipage} &
\begin{minipage}[t]{0.09\columnwidth}\raggedleft\strut
0
\strut\end{minipage} &
\begin{minipage}[t]{0.10\columnwidth}\raggedleft\strut
0
\strut\end{minipage} &
\begin{minipage}[t]{0.11\columnwidth}\raggedleft\strut
0
\strut\end{minipage} &
\begin{minipage}[t]{0.07\columnwidth}\raggedleft\strut
7
\strut\end{minipage}\tabularnewline
\begin{minipage}[t]{0.11\columnwidth}\centering\strut
\textbf{F18}
\strut\end{minipage} &
\begin{minipage}[t]{0.07\columnwidth}\raggedleft\strut
0.4
\strut\end{minipage} &
\begin{minipage}[t]{0.08\columnwidth}\raggedleft\strut
0.4
\strut\end{minipage} &
\begin{minipage}[t]{0.09\columnwidth}\raggedleft\strut
0.4
\strut\end{minipage} &
\begin{minipage}[t]{0.10\columnwidth}\raggedleft\strut
0.4
\strut\end{minipage} &
\begin{minipage}[t]{0.11\columnwidth}\raggedleft\strut
0.4
\strut\end{minipage} &
\begin{minipage}[t]{0.07\columnwidth}\raggedleft\strut
10
\strut\end{minipage}\tabularnewline
\begin{minipage}[t]{0.11\columnwidth}\centering\strut
\textbf{F19}
\strut\end{minipage} &
\begin{minipage}[t]{0.07\columnwidth}\raggedleft\strut
0
\strut\end{minipage} &
\begin{minipage}[t]{0.08\columnwidth}\raggedleft\strut
0
\strut\end{minipage} &
\begin{minipage}[t]{0.09\columnwidth}\raggedleft\strut
0
\strut\end{minipage} &
\begin{minipage}[t]{0.10\columnwidth}\raggedleft\strut
0
\strut\end{minipage} &
\begin{minipage}[t]{0.11\columnwidth}\raggedleft\strut
0
\strut\end{minipage} &
\begin{minipage}[t]{0.07\columnwidth}\raggedleft\strut
7
\strut\end{minipage}\tabularnewline
\begin{minipage}[t]{0.11\columnwidth}\centering\strut
\textbf{F20}
\strut\end{minipage} &
\begin{minipage}[t]{0.07\columnwidth}\raggedleft\strut
0
\strut\end{minipage} &
\begin{minipage}[t]{0.08\columnwidth}\raggedleft\strut
0
\strut\end{minipage} &
\begin{minipage}[t]{0.09\columnwidth}\raggedleft\strut
0
\strut\end{minipage} &
\begin{minipage}[t]{0.10\columnwidth}\raggedleft\strut
0
\strut\end{minipage} &
\begin{minipage}[t]{0.11\columnwidth}\raggedleft\strut
0
\strut\end{minipage} &
\begin{minipage}[t]{0.07\columnwidth}\raggedleft\strut
6
\strut\end{minipage}\tabularnewline
\bottomrule
\end{longtable}

\begin{longtable}[c]{@{}crrrrrr@{}}
\caption{Convergence Rates over given runs}\tabularnewline
\toprule
\begin{minipage}[b]{0.11\columnwidth}\centering\strut
~
\strut\end{minipage} &
\begin{minipage}[b]{0.08\columnwidth}\raggedleft\strut
0.1
\strut\end{minipage} &
\begin{minipage}[b]{0.08\columnwidth}\raggedleft\strut
0.01
\strut\end{minipage} &
\begin{minipage}[b]{0.09\columnwidth}\raggedleft\strut
0.001
\strut\end{minipage} &
\begin{minipage}[b]{0.10\columnwidth}\raggedleft\strut
0.0001
\strut\end{minipage} &
\begin{minipage}[b]{0.11\columnwidth}\raggedleft\strut
0.00001
\strut\end{minipage} &
\begin{minipage}[b]{0.07\columnwidth}\raggedleft\strut
runs
\strut\end{minipage}\tabularnewline
\midrule
\endfirsthead
\toprule
\begin{minipage}[b]{0.11\columnwidth}\centering\strut
~
\strut\end{minipage} &
\begin{minipage}[b]{0.08\columnwidth}\raggedleft\strut
0.1
\strut\end{minipage} &
\begin{minipage}[b]{0.08\columnwidth}\raggedleft\strut
0.01
\strut\end{minipage} &
\begin{minipage}[b]{0.09\columnwidth}\raggedleft\strut
0.001
\strut\end{minipage} &
\begin{minipage}[b]{0.10\columnwidth}\raggedleft\strut
0.0001
\strut\end{minipage} &
\begin{minipage}[b]{0.11\columnwidth}\raggedleft\strut
0.00001
\strut\end{minipage} &
\begin{minipage}[b]{0.07\columnwidth}\raggedleft\strut
runs
\strut\end{minipage}\tabularnewline
\midrule
\endhead
\begin{minipage}[t]{0.11\columnwidth}\centering\strut
\textbf{F1}
\strut\end{minipage} &
\begin{minipage}[t]{0.08\columnwidth}\raggedleft\strut
641
\strut\end{minipage} &
\begin{minipage}[t]{0.08\columnwidth}\raggedleft\strut
839
\strut\end{minipage} &
\begin{minipage}[t]{0.09\columnwidth}\raggedleft\strut
1050
\strut\end{minipage} &
\begin{minipage}[t]{0.10\columnwidth}\raggedleft\strut
1228
\strut\end{minipage} &
\begin{minipage}[t]{0.11\columnwidth}\raggedleft\strut
1449
\strut\end{minipage} &
\begin{minipage}[t]{0.07\columnwidth}\raggedleft\strut
32
\strut\end{minipage}\tabularnewline
\begin{minipage}[t]{0.11\columnwidth}\centering\strut
\textbf{F2}
\strut\end{minipage} &
\begin{minipage}[t]{0.08\columnwidth}\raggedleft\strut
179
\strut\end{minipage} &
\begin{minipage}[t]{0.08\columnwidth}\raggedleft\strut
256
\strut\end{minipage} &
\begin{minipage}[t]{0.09\columnwidth}\raggedleft\strut
394
\strut\end{minipage} &
\begin{minipage}[t]{0.10\columnwidth}\raggedleft\strut
534
\strut\end{minipage} &
\begin{minipage}[t]{0.11\columnwidth}\raggedleft\strut
636
\strut\end{minipage} &
\begin{minipage}[t]{0.07\columnwidth}\raggedleft\strut
30
\strut\end{minipage}\tabularnewline
\begin{minipage}[t]{0.11\columnwidth}\centering\strut
\textbf{F3}
\strut\end{minipage} &
\begin{minipage}[t]{0.08\columnwidth}\raggedleft\strut
38
\strut\end{minipage} &
\begin{minipage}[t]{0.08\columnwidth}\raggedleft\strut
182
\strut\end{minipage} &
\begin{minipage}[t]{0.09\columnwidth}\raggedleft\strut
277
\strut\end{minipage} &
\begin{minipage}[t]{0.10\columnwidth}\raggedleft\strut
386
\strut\end{minipage} &
\begin{minipage}[t]{0.11\columnwidth}\raggedleft\strut
511
\strut\end{minipage} &
\begin{minipage}[t]{0.07\columnwidth}\raggedleft\strut
32
\strut\end{minipage}\tabularnewline
\begin{minipage}[t]{0.11\columnwidth}\centering\strut
\textbf{F4}
\strut\end{minipage} &
\begin{minipage}[t]{0.08\columnwidth}\raggedleft\strut
501
\strut\end{minipage} &
\begin{minipage}[t]{0.08\columnwidth}\raggedleft\strut
736
\strut\end{minipage} &
\begin{minipage}[t]{0.09\columnwidth}\raggedleft\strut
969
\strut\end{minipage} &
\begin{minipage}[t]{0.10\columnwidth}\raggedleft\strut
1173
\strut\end{minipage} &
\begin{minipage}[t]{0.11\columnwidth}\raggedleft\strut
1434
\strut\end{minipage} &
\begin{minipage}[t]{0.07\columnwidth}\raggedleft\strut
32
\strut\end{minipage}\tabularnewline
\begin{minipage}[t]{0.11\columnwidth}\centering\strut
\textbf{F5}
\strut\end{minipage} &
\begin{minipage}[t]{0.08\columnwidth}\raggedleft\strut
84
\strut\end{minipage} &
\begin{minipage}[t]{0.08\columnwidth}\raggedleft\strut
200
\strut\end{minipage} &
\begin{minipage}[t]{0.09\columnwidth}\raggedleft\strut
322
\strut\end{minipage} &
\begin{minipage}[t]{0.10\columnwidth}\raggedleft\strut
503
\strut\end{minipage} &
\begin{minipage}[t]{0.11\columnwidth}\raggedleft\strut
786
\strut\end{minipage} &
\begin{minipage}[t]{0.07\columnwidth}\raggedleft\strut
29
\strut\end{minipage}\tabularnewline
\begin{minipage}[t]{0.11\columnwidth}\centering\strut
\textbf{F6}
\strut\end{minipage} &
\begin{minipage}[t]{0.08\columnwidth}\raggedleft\strut
19548
\strut\end{minipage} &
\begin{minipage}[t]{0.08\columnwidth}\raggedleft\strut
24640
\strut\end{minipage} &
\begin{minipage}[t]{0.09\columnwidth}\raggedleft\strut
30646
\strut\end{minipage} &
\begin{minipage}[t]{0.10\columnwidth}\raggedleft\strut
42534
\strut\end{minipage} &
\begin{minipage}[t]{0.11\columnwidth}\raggedleft\strut
2e+05
\strut\end{minipage} &
\begin{minipage}[t]{0.07\columnwidth}\raggedleft\strut
29
\strut\end{minipage}\tabularnewline
\begin{minipage}[t]{0.11\columnwidth}\centering\strut
\textbf{F7}
\strut\end{minipage} &
\begin{minipage}[t]{0.08\columnwidth}\raggedleft\strut
8368
\strut\end{minipage} &
\begin{minipage}[t]{0.08\columnwidth}\raggedleft\strut
9059
\strut\end{minipage} &
\begin{minipage}[t]{0.09\columnwidth}\raggedleft\strut
10272
\strut\end{minipage} &
\begin{minipage}[t]{0.10\columnwidth}\raggedleft\strut
12212
\strut\end{minipage} &
\begin{minipage}[t]{0.11\columnwidth}\raggedleft\strut
13613
\strut\end{minipage} &
\begin{minipage}[t]{0.07\columnwidth}\raggedleft\strut
29
\strut\end{minipage}\tabularnewline
\begin{minipage}[t]{0.11\columnwidth}\centering\strut
\textbf{F8}
\strut\end{minipage} &
\begin{minipage}[t]{0.08\columnwidth}\raggedleft\strut
176727
\strut\end{minipage} &
\begin{minipage}[t]{0.08\columnwidth}\raggedleft\strut
215977
\strut\end{minipage} &
\begin{minipage}[t]{0.09\columnwidth}\raggedleft\strut
253443
\strut\end{minipage} &
\begin{minipage}[t]{0.10\columnwidth}\raggedleft\strut
307968
\strut\end{minipage} &
\begin{minipage}[t]{0.11\columnwidth}\raggedleft\strut
341918
\strut\end{minipage} &
\begin{minipage}[t]{0.07\columnwidth}\raggedleft\strut
11
\strut\end{minipage}\tabularnewline
\begin{minipage}[t]{0.11\columnwidth}\centering\strut
\textbf{F9}
\strut\end{minipage} &
\begin{minipage}[t]{0.08\columnwidth}\raggedleft\strut
175729
\strut\end{minipage} &
\begin{minipage}[t]{0.08\columnwidth}\raggedleft\strut
179787
\strut\end{minipage} &
\begin{minipage}[t]{0.09\columnwidth}\raggedleft\strut
192204
\strut\end{minipage} &
\begin{minipage}[t]{0.10\columnwidth}\raggedleft\strut
2e+05
\strut\end{minipage} &
\begin{minipage}[t]{0.11\columnwidth}\raggedleft\strut
213030
\strut\end{minipage} &
\begin{minipage}[t]{0.07\columnwidth}\raggedleft\strut
15
\strut\end{minipage}\tabularnewline
\begin{minipage}[t]{0.11\columnwidth}\centering\strut
\textbf{F10}
\strut\end{minipage} &
\begin{minipage}[t]{0.08\columnwidth}\raggedleft\strut
899
\strut\end{minipage} &
\begin{minipage}[t]{0.08\columnwidth}\raggedleft\strut
1347
\strut\end{minipage} &
\begin{minipage}[t]{0.09\columnwidth}\raggedleft\strut
1727
\strut\end{minipage} &
\begin{minipage}[t]{0.10\columnwidth}\raggedleft\strut
2255
\strut\end{minipage} &
\begin{minipage}[t]{0.11\columnwidth}\raggedleft\strut
2773
\strut\end{minipage} &
\begin{minipage}[t]{0.07\columnwidth}\raggedleft\strut
29
\strut\end{minipage}\tabularnewline
\begin{minipage}[t]{0.11\columnwidth}\centering\strut
\textbf{F11}
\strut\end{minipage} &
\begin{minipage}[t]{0.08\columnwidth}\raggedleft\strut
3517
\strut\end{minipage} &
\begin{minipage}[t]{0.08\columnwidth}\raggedleft\strut
5516
\strut\end{minipage} &
\begin{minipage}[t]{0.09\columnwidth}\raggedleft\strut
7099
\strut\end{minipage} &
\begin{minipage}[t]{0.10\columnwidth}\raggedleft\strut
8069
\strut\end{minipage} &
\begin{minipage}[t]{0.11\columnwidth}\raggedleft\strut
8741
\strut\end{minipage} &
\begin{minipage}[t]{0.07\columnwidth}\raggedleft\strut
29
\strut\end{minipage}\tabularnewline
\begin{minipage}[t]{0.11\columnwidth}\centering\strut
\textbf{F12}
\strut\end{minipage} &
\begin{minipage}[t]{0.08\columnwidth}\raggedleft\strut
17330
\strut\end{minipage} &
\begin{minipage}[t]{0.08\columnwidth}\raggedleft\strut
25136
\strut\end{minipage} &
\begin{minipage}[t]{0.09\columnwidth}\raggedleft\strut
37468
\strut\end{minipage} &
\begin{minipage}[t]{0.10\columnwidth}\raggedleft\strut
44542
\strut\end{minipage} &
\begin{minipage}[t]{0.11\columnwidth}\raggedleft\strut
50752
\strut\end{minipage} &
\begin{minipage}[t]{0.07\columnwidth}\raggedleft\strut
28
\strut\end{minipage}\tabularnewline
\begin{minipage}[t]{0.11\columnwidth}\centering\strut
\textbf{F13}
\strut\end{minipage} &
\begin{minipage}[t]{0.08\columnwidth}\raggedleft\strut
11555
\strut\end{minipage} &
\begin{minipage}[t]{0.08\columnwidth}\raggedleft\strut
16033
\strut\end{minipage} &
\begin{minipage}[t]{0.09\columnwidth}\raggedleft\strut
20058
\strut\end{minipage} &
\begin{minipage}[t]{0.10\columnwidth}\raggedleft\strut
23771
\strut\end{minipage} &
\begin{minipage}[t]{0.11\columnwidth}\raggedleft\strut
27500
\strut\end{minipage} &
\begin{minipage}[t]{0.07\columnwidth}\raggedleft\strut
28
\strut\end{minipage}\tabularnewline
\begin{minipage}[t]{0.11\columnwidth}\centering\strut
\textbf{F14}
\strut\end{minipage} &
\begin{minipage}[t]{0.08\columnwidth}\raggedleft\strut
29504
\strut\end{minipage} &
\begin{minipage}[t]{0.08\columnwidth}\raggedleft\strut
35431
\strut\end{minipage} &
\begin{minipage}[t]{0.09\columnwidth}\raggedleft\strut
50219
\strut\end{minipage} &
\begin{minipage}[t]{0.10\columnwidth}\raggedleft\strut
58919
\strut\end{minipage} &
\begin{minipage}[t]{0.11\columnwidth}\raggedleft\strut
68803
\strut\end{minipage} &
\begin{minipage}[t]{0.07\columnwidth}\raggedleft\strut
27
\strut\end{minipage}\tabularnewline
\begin{minipage}[t]{0.11\columnwidth}\centering\strut
\textbf{F15}
\strut\end{minipage} &
\begin{minipage}[t]{0.08\columnwidth}\raggedleft\strut
97539
\strut\end{minipage} &
\begin{minipage}[t]{0.08\columnwidth}\raggedleft\strut
120679
\strut\end{minipage} &
\begin{minipage}[t]{0.09\columnwidth}\raggedleft\strut
145910
\strut\end{minipage} &
\begin{minipage}[t]{0.10\columnwidth}\raggedleft\strut
175327
\strut\end{minipage} &
\begin{minipage}[t]{0.11\columnwidth}\raggedleft\strut
185881
\strut\end{minipage} &
\begin{minipage}[t]{0.07\columnwidth}\raggedleft\strut
20
\strut\end{minipage}\tabularnewline
\begin{minipage}[t]{0.11\columnwidth}\centering\strut
\textbf{F16}
\strut\end{minipage} &
\begin{minipage}[t]{0.08\columnwidth}\raggedleft\strut
4e+05
\strut\end{minipage} &
\begin{minipage}[t]{0.08\columnwidth}\raggedleft\strut
4e+05
\strut\end{minipage} &
\begin{minipage}[t]{0.09\columnwidth}\raggedleft\strut
4e+05
\strut\end{minipage} &
\begin{minipage}[t]{0.10\columnwidth}\raggedleft\strut
4e+05
\strut\end{minipage} &
\begin{minipage}[t]{0.11\columnwidth}\raggedleft\strut
4e+05
\strut\end{minipage} &
\begin{minipage}[t]{0.07\columnwidth}\raggedleft\strut
8
\strut\end{minipage}\tabularnewline
\begin{minipage}[t]{0.11\columnwidth}\centering\strut
\textbf{F17}
\strut\end{minipage} &
\begin{minipage}[t]{0.08\columnwidth}\raggedleft\strut
362540
\strut\end{minipage} &
\begin{minipage}[t]{0.08\columnwidth}\raggedleft\strut
4e+05
\strut\end{minipage} &
\begin{minipage}[t]{0.09\columnwidth}\raggedleft\strut
4e+05
\strut\end{minipage} &
\begin{minipage}[t]{0.10\columnwidth}\raggedleft\strut
4e+05
\strut\end{minipage} &
\begin{minipage}[t]{0.11\columnwidth}\raggedleft\strut
4e+05
\strut\end{minipage} &
\begin{minipage}[t]{0.07\columnwidth}\raggedleft\strut
7
\strut\end{minipage}\tabularnewline
\begin{minipage}[t]{0.11\columnwidth}\centering\strut
\textbf{F18}
\strut\end{minipage} &
\begin{minipage}[t]{0.08\columnwidth}\raggedleft\strut
288605
\strut\end{minipage} &
\begin{minipage}[t]{0.08\columnwidth}\raggedleft\strut
288631
\strut\end{minipage} &
\begin{minipage}[t]{0.09\columnwidth}\raggedleft\strut
288822
\strut\end{minipage} &
\begin{minipage}[t]{0.10\columnwidth}\raggedleft\strut
288905
\strut\end{minipage} &
\begin{minipage}[t]{0.11\columnwidth}\raggedleft\strut
289382
\strut\end{minipage} &
\begin{minipage}[t]{0.07\columnwidth}\raggedleft\strut
10
\strut\end{minipage}\tabularnewline
\begin{minipage}[t]{0.11\columnwidth}\centering\strut
\textbf{F19}
\strut\end{minipage} &
\begin{minipage}[t]{0.08\columnwidth}\raggedleft\strut
4e+05
\strut\end{minipage} &
\begin{minipage}[t]{0.08\columnwidth}\raggedleft\strut
4e+05
\strut\end{minipage} &
\begin{minipage}[t]{0.09\columnwidth}\raggedleft\strut
4e+05
\strut\end{minipage} &
\begin{minipage}[t]{0.10\columnwidth}\raggedleft\strut
4e+05
\strut\end{minipage} &
\begin{minipage}[t]{0.11\columnwidth}\raggedleft\strut
4e+05
\strut\end{minipage} &
\begin{minipage}[t]{0.07\columnwidth}\raggedleft\strut
7
\strut\end{minipage}\tabularnewline
\begin{minipage}[t]{0.11\columnwidth}\centering\strut
\textbf{F20}
\strut\end{minipage} &
\begin{minipage}[t]{0.08\columnwidth}\raggedleft\strut
4e+05
\strut\end{minipage} &
\begin{minipage}[t]{0.08\columnwidth}\raggedleft\strut
4e+05
\strut\end{minipage} &
\begin{minipage}[t]{0.09\columnwidth}\raggedleft\strut
4e+05
\strut\end{minipage} &
\begin{minipage}[t]{0.10\columnwidth}\raggedleft\strut
4e+05
\strut\end{minipage} &
\begin{minipage}[t]{0.11\columnwidth}\raggedleft\strut
4e+05
\strut\end{minipage} &
\begin{minipage}[t]{0.07\columnwidth}\raggedleft\strut
6
\strut\end{minipage}\tabularnewline
\bottomrule
\end{longtable}

\begin{longtable}[c]{@{}crrrrrr@{}}
\caption{Mean Peak Ratio over given runs}\tabularnewline
\toprule
\begin{minipage}[b]{0.11\columnwidth}\centering\strut
~
\strut\end{minipage} &
\begin{minipage}[b]{0.07\columnwidth}\raggedleft\strut
0.1
\strut\end{minipage} &
\begin{minipage}[b]{0.08\columnwidth}\raggedleft\strut
0.01
\strut\end{minipage} &
\begin{minipage}[b]{0.09\columnwidth}\raggedleft\strut
0.001
\strut\end{minipage} &
\begin{minipage}[b]{0.10\columnwidth}\raggedleft\strut
0.0001
\strut\end{minipage} &
\begin{minipage}[b]{0.11\columnwidth}\raggedleft\strut
0.00001
\strut\end{minipage} &
\begin{minipage}[b]{0.07\columnwidth}\raggedleft\strut
runs
\strut\end{minipage}\tabularnewline
\midrule
\endfirsthead
\toprule
\begin{minipage}[b]{0.11\columnwidth}\centering\strut
~
\strut\end{minipage} &
\begin{minipage}[b]{0.07\columnwidth}\raggedleft\strut
0.1
\strut\end{minipage} &
\begin{minipage}[b]{0.08\columnwidth}\raggedleft\strut
0.01
\strut\end{minipage} &
\begin{minipage}[b]{0.09\columnwidth}\raggedleft\strut
0.001
\strut\end{minipage} &
\begin{minipage}[b]{0.10\columnwidth}\raggedleft\strut
0.0001
\strut\end{minipage} &
\begin{minipage}[b]{0.11\columnwidth}\raggedleft\strut
0.00001
\strut\end{minipage} &
\begin{minipage}[b]{0.07\columnwidth}\raggedleft\strut
runs
\strut\end{minipage}\tabularnewline
\midrule
\endhead
\begin{minipage}[t]{0.11\columnwidth}\centering\strut
\textbf{F1}
\strut\end{minipage} &
\begin{minipage}[t]{0.07\columnwidth}\raggedleft\strut
1
\strut\end{minipage} &
\begin{minipage}[t]{0.08\columnwidth}\raggedleft\strut
1
\strut\end{minipage} &
\begin{minipage}[t]{0.09\columnwidth}\raggedleft\strut
1
\strut\end{minipage} &
\begin{minipage}[t]{0.10\columnwidth}\raggedleft\strut
1
\strut\end{minipage} &
\begin{minipage}[t]{0.11\columnwidth}\raggedleft\strut
1
\strut\end{minipage} &
\begin{minipage}[t]{0.07\columnwidth}\raggedleft\strut
32
\strut\end{minipage}\tabularnewline
\begin{minipage}[t]{0.11\columnwidth}\centering\strut
\textbf{F2}
\strut\end{minipage} &
\begin{minipage}[t]{0.07\columnwidth}\raggedleft\strut
1
\strut\end{minipage} &
\begin{minipage}[t]{0.08\columnwidth}\raggedleft\strut
1
\strut\end{minipage} &
\begin{minipage}[t]{0.09\columnwidth}\raggedleft\strut
1
\strut\end{minipage} &
\begin{minipage}[t]{0.10\columnwidth}\raggedleft\strut
1
\strut\end{minipage} &
\begin{minipage}[t]{0.11\columnwidth}\raggedleft\strut
1
\strut\end{minipage} &
\begin{minipage}[t]{0.07\columnwidth}\raggedleft\strut
30
\strut\end{minipage}\tabularnewline
\begin{minipage}[t]{0.11\columnwidth}\centering\strut
\textbf{F3}
\strut\end{minipage} &
\begin{minipage}[t]{0.07\columnwidth}\raggedleft\strut
1
\strut\end{minipage} &
\begin{minipage}[t]{0.08\columnwidth}\raggedleft\strut
1
\strut\end{minipage} &
\begin{minipage}[t]{0.09\columnwidth}\raggedleft\strut
1
\strut\end{minipage} &
\begin{minipage}[t]{0.10\columnwidth}\raggedleft\strut
1
\strut\end{minipage} &
\begin{minipage}[t]{0.11\columnwidth}\raggedleft\strut
1
\strut\end{minipage} &
\begin{minipage}[t]{0.07\columnwidth}\raggedleft\strut
32
\strut\end{minipage}\tabularnewline
\begin{minipage}[t]{0.11\columnwidth}\centering\strut
\textbf{F4}
\strut\end{minipage} &
\begin{minipage}[t]{0.07\columnwidth}\raggedleft\strut
1
\strut\end{minipage} &
\begin{minipage}[t]{0.08\columnwidth}\raggedleft\strut
1
\strut\end{minipage} &
\begin{minipage}[t]{0.09\columnwidth}\raggedleft\strut
1
\strut\end{minipage} &
\begin{minipage}[t]{0.10\columnwidth}\raggedleft\strut
1
\strut\end{minipage} &
\begin{minipage}[t]{0.11\columnwidth}\raggedleft\strut
1
\strut\end{minipage} &
\begin{minipage}[t]{0.07\columnwidth}\raggedleft\strut
32
\strut\end{minipage}\tabularnewline
\begin{minipage}[t]{0.11\columnwidth}\centering\strut
\textbf{F5}
\strut\end{minipage} &
\begin{minipage}[t]{0.07\columnwidth}\raggedleft\strut
1
\strut\end{minipage} &
\begin{minipage}[t]{0.08\columnwidth}\raggedleft\strut
1
\strut\end{minipage} &
\begin{minipage}[t]{0.09\columnwidth}\raggedleft\strut
1
\strut\end{minipage} &
\begin{minipage}[t]{0.10\columnwidth}\raggedleft\strut
1
\strut\end{minipage} &
\begin{minipage}[t]{0.11\columnwidth}\raggedleft\strut
1
\strut\end{minipage} &
\begin{minipage}[t]{0.07\columnwidth}\raggedleft\strut
29
\strut\end{minipage}\tabularnewline
\begin{minipage}[t]{0.11\columnwidth}\centering\strut
\textbf{F6}
\strut\end{minipage} &
\begin{minipage}[t]{0.07\columnwidth}\raggedleft\strut
1
\strut\end{minipage} &
\begin{minipage}[t]{0.08\columnwidth}\raggedleft\strut
1
\strut\end{minipage} &
\begin{minipage}[t]{0.09\columnwidth}\raggedleft\strut
1
\strut\end{minipage} &
\begin{minipage}[t]{0.10\columnwidth}\raggedleft\strut
1
\strut\end{minipage} &
\begin{minipage}[t]{0.11\columnwidth}\raggedleft\strut
0
\strut\end{minipage} &
\begin{minipage}[t]{0.07\columnwidth}\raggedleft\strut
29
\strut\end{minipage}\tabularnewline
\begin{minipage}[t]{0.11\columnwidth}\centering\strut
\textbf{F7}
\strut\end{minipage} &
\begin{minipage}[t]{0.07\columnwidth}\raggedleft\strut
1
\strut\end{minipage} &
\begin{minipage}[t]{0.08\columnwidth}\raggedleft\strut
1
\strut\end{minipage} &
\begin{minipage}[t]{0.09\columnwidth}\raggedleft\strut
1
\strut\end{minipage} &
\begin{minipage}[t]{0.10\columnwidth}\raggedleft\strut
1
\strut\end{minipage} &
\begin{minipage}[t]{0.11\columnwidth}\raggedleft\strut
1
\strut\end{minipage} &
\begin{minipage}[t]{0.07\columnwidth}\raggedleft\strut
29
\strut\end{minipage}\tabularnewline
\begin{minipage}[t]{0.11\columnwidth}\centering\strut
\textbf{F8}
\strut\end{minipage} &
\begin{minipage}[t]{0.07\columnwidth}\raggedleft\strut
1
\strut\end{minipage} &
\begin{minipage}[t]{0.08\columnwidth}\raggedleft\strut
1
\strut\end{minipage} &
\begin{minipage}[t]{0.09\columnwidth}\raggedleft\strut
1
\strut\end{minipage} &
\begin{minipage}[t]{0.10\columnwidth}\raggedleft\strut
1
\strut\end{minipage} &
\begin{minipage}[t]{0.11\columnwidth}\raggedleft\strut
0.98
\strut\end{minipage} &
\begin{minipage}[t]{0.07\columnwidth}\raggedleft\strut
11
\strut\end{minipage}\tabularnewline
\begin{minipage}[t]{0.11\columnwidth}\centering\strut
\textbf{F9}
\strut\end{minipage} &
\begin{minipage}[t]{0.07\columnwidth}\raggedleft\strut
1
\strut\end{minipage} &
\begin{minipage}[t]{0.08\columnwidth}\raggedleft\strut
1
\strut\end{minipage} &
\begin{minipage}[t]{0.09\columnwidth}\raggedleft\strut
1
\strut\end{minipage} &
\begin{minipage}[t]{0.10\columnwidth}\raggedleft\strut
1
\strut\end{minipage} &
\begin{minipage}[t]{0.11\columnwidth}\raggedleft\strut
1
\strut\end{minipage} &
\begin{minipage}[t]{0.07\columnwidth}\raggedleft\strut
15
\strut\end{minipage}\tabularnewline
\begin{minipage}[t]{0.11\columnwidth}\centering\strut
\textbf{F10}
\strut\end{minipage} &
\begin{minipage}[t]{0.07\columnwidth}\raggedleft\strut
1
\strut\end{minipage} &
\begin{minipage}[t]{0.08\columnwidth}\raggedleft\strut
1
\strut\end{minipage} &
\begin{minipage}[t]{0.09\columnwidth}\raggedleft\strut
1
\strut\end{minipage} &
\begin{minipage}[t]{0.10\columnwidth}\raggedleft\strut
1
\strut\end{minipage} &
\begin{minipage}[t]{0.11\columnwidth}\raggedleft\strut
1
\strut\end{minipage} &
\begin{minipage}[t]{0.07\columnwidth}\raggedleft\strut
29
\strut\end{minipage}\tabularnewline
\begin{minipage}[t]{0.11\columnwidth}\centering\strut
\textbf{F11}
\strut\end{minipage} &
\begin{minipage}[t]{0.07\columnwidth}\raggedleft\strut
1
\strut\end{minipage} &
\begin{minipage}[t]{0.08\columnwidth}\raggedleft\strut
1
\strut\end{minipage} &
\begin{minipage}[t]{0.09\columnwidth}\raggedleft\strut
1
\strut\end{minipage} &
\begin{minipage}[t]{0.10\columnwidth}\raggedleft\strut
1
\strut\end{minipage} &
\begin{minipage}[t]{0.11\columnwidth}\raggedleft\strut
1
\strut\end{minipage} &
\begin{minipage}[t]{0.07\columnwidth}\raggedleft\strut
29
\strut\end{minipage}\tabularnewline
\begin{minipage}[t]{0.11\columnwidth}\centering\strut
\textbf{F12}
\strut\end{minipage} &
\begin{minipage}[t]{0.07\columnwidth}\raggedleft\strut
1
\strut\end{minipage} &
\begin{minipage}[t]{0.08\columnwidth}\raggedleft\strut
1
\strut\end{minipage} &
\begin{minipage}[t]{0.09\columnwidth}\raggedleft\strut
1
\strut\end{minipage} &
\begin{minipage}[t]{0.10\columnwidth}\raggedleft\strut
1
\strut\end{minipage} &
\begin{minipage}[t]{0.11\columnwidth}\raggedleft\strut
1
\strut\end{minipage} &
\begin{minipage}[t]{0.07\columnwidth}\raggedleft\strut
28
\strut\end{minipage}\tabularnewline
\begin{minipage}[t]{0.11\columnwidth}\centering\strut
\textbf{F13}
\strut\end{minipage} &
\begin{minipage}[t]{0.07\columnwidth}\raggedleft\strut
1
\strut\end{minipage} &
\begin{minipage}[t]{0.08\columnwidth}\raggedleft\strut
1
\strut\end{minipage} &
\begin{minipage}[t]{0.09\columnwidth}\raggedleft\strut
1
\strut\end{minipage} &
\begin{minipage}[t]{0.10\columnwidth}\raggedleft\strut
1
\strut\end{minipage} &
\begin{minipage}[t]{0.11\columnwidth}\raggedleft\strut
1
\strut\end{minipage} &
\begin{minipage}[t]{0.07\columnwidth}\raggedleft\strut
28
\strut\end{minipage}\tabularnewline
\begin{minipage}[t]{0.11\columnwidth}\centering\strut
\textbf{F14}
\strut\end{minipage} &
\begin{minipage}[t]{0.07\columnwidth}\raggedleft\strut
1
\strut\end{minipage} &
\begin{minipage}[t]{0.08\columnwidth}\raggedleft\strut
1
\strut\end{minipage} &
\begin{minipage}[t]{0.09\columnwidth}\raggedleft\strut
1
\strut\end{minipage} &
\begin{minipage}[t]{0.10\columnwidth}\raggedleft\strut
1
\strut\end{minipage} &
\begin{minipage}[t]{0.11\columnwidth}\raggedleft\strut
1
\strut\end{minipage} &
\begin{minipage}[t]{0.07\columnwidth}\raggedleft\strut
27
\strut\end{minipage}\tabularnewline
\begin{minipage}[t]{0.11\columnwidth}\centering\strut
\textbf{F15}
\strut\end{minipage} &
\begin{minipage}[t]{0.07\columnwidth}\raggedleft\strut
1
\strut\end{minipage} &
\begin{minipage}[t]{0.08\columnwidth}\raggedleft\strut
1
\strut\end{minipage} &
\begin{minipage}[t]{0.09\columnwidth}\raggedleft\strut
1
\strut\end{minipage} &
\begin{minipage}[t]{0.10\columnwidth}\raggedleft\strut
1
\strut\end{minipage} &
\begin{minipage}[t]{0.11\columnwidth}\raggedleft\strut
1
\strut\end{minipage} &
\begin{minipage}[t]{0.07\columnwidth}\raggedleft\strut
20
\strut\end{minipage}\tabularnewline
\begin{minipage}[t]{0.11\columnwidth}\centering\strut
\textbf{F16}
\strut\end{minipage} &
\begin{minipage}[t]{0.07\columnwidth}\raggedleft\strut
0.02
\strut\end{minipage} &
\begin{minipage}[t]{0.08\columnwidth}\raggedleft\strut
0
\strut\end{minipage} &
\begin{minipage}[t]{0.09\columnwidth}\raggedleft\strut
0
\strut\end{minipage} &
\begin{minipage}[t]{0.10\columnwidth}\raggedleft\strut
0
\strut\end{minipage} &
\begin{minipage}[t]{0.11\columnwidth}\raggedleft\strut
0
\strut\end{minipage} &
\begin{minipage}[t]{0.07\columnwidth}\raggedleft\strut
8
\strut\end{minipage}\tabularnewline
\begin{minipage}[t]{0.11\columnwidth}\centering\strut
\textbf{F17}
\strut\end{minipage} &
\begin{minipage}[t]{0.07\columnwidth}\raggedleft\strut
0.8
\strut\end{minipage} &
\begin{minipage}[t]{0.08\columnwidth}\raggedleft\strut
0.79
\strut\end{minipage} &
\begin{minipage}[t]{0.09\columnwidth}\raggedleft\strut
0.79
\strut\end{minipage} &
\begin{minipage}[t]{0.10\columnwidth}\raggedleft\strut
0.79
\strut\end{minipage} &
\begin{minipage}[t]{0.11\columnwidth}\raggedleft\strut
0.71
\strut\end{minipage} &
\begin{minipage}[t]{0.07\columnwidth}\raggedleft\strut
7
\strut\end{minipage}\tabularnewline
\begin{minipage}[t]{0.11\columnwidth}\centering\strut
\textbf{F18}
\strut\end{minipage} &
\begin{minipage}[t]{0.07\columnwidth}\raggedleft\strut
0.82
\strut\end{minipage} &
\begin{minipage}[t]{0.08\columnwidth}\raggedleft\strut
0.82
\strut\end{minipage} &
\begin{minipage}[t]{0.09\columnwidth}\raggedleft\strut
0.82
\strut\end{minipage} &
\begin{minipage}[t]{0.10\columnwidth}\raggedleft\strut
0.8
\strut\end{minipage} &
\begin{minipage}[t]{0.11\columnwidth}\raggedleft\strut
0.8
\strut\end{minipage} &
\begin{minipage}[t]{0.07\columnwidth}\raggedleft\strut
10
\strut\end{minipage}\tabularnewline
\begin{minipage}[t]{0.11\columnwidth}\centering\strut
\textbf{F19}
\strut\end{minipage} &
\begin{minipage}[t]{0.07\columnwidth}\raggedleft\strut
0.36
\strut\end{minipage} &
\begin{minipage}[t]{0.08\columnwidth}\raggedleft\strut
0.36
\strut\end{minipage} &
\begin{minipage}[t]{0.09\columnwidth}\raggedleft\strut
0.33
\strut\end{minipage} &
\begin{minipage}[t]{0.10\columnwidth}\raggedleft\strut
0.33
\strut\end{minipage} &
\begin{minipage}[t]{0.11\columnwidth}\raggedleft\strut
0.31
\strut\end{minipage} &
\begin{minipage}[t]{0.07\columnwidth}\raggedleft\strut
7
\strut\end{minipage}\tabularnewline
\begin{minipage}[t]{0.11\columnwidth}\centering\strut
\textbf{F20}
\strut\end{minipage} &
\begin{minipage}[t]{0.07\columnwidth}\raggedleft\strut
0.21
\strut\end{minipage} &
\begin{minipage}[t]{0.08\columnwidth}\raggedleft\strut
0.19
\strut\end{minipage} &
\begin{minipage}[t]{0.09\columnwidth}\raggedleft\strut
0.19
\strut\end{minipage} &
\begin{minipage}[t]{0.10\columnwidth}\raggedleft\strut
0.19
\strut\end{minipage} &
\begin{minipage}[t]{0.11\columnwidth}\raggedleft\strut
0.17
\strut\end{minipage} &
\begin{minipage}[t]{0.07\columnwidth}\raggedleft\strut
6
\strut\end{minipage}\tabularnewline
\bottomrule
\end{longtable}

\begin{verbatim}
## Warning in scan(file, what, nmax, sep, dec, quote, skip, nlines,
## na.strings, : number of items read is not a multiple of the number of
## columns
\end{verbatim}

\begin{verbatim}
## Warning: Removed 1 rows containing missing values (stat_summary).
\end{verbatim}

\begin{verbatim}
## Warning: Removed 1 rows containing missing values (geom_path).
\end{verbatim}

\begin{figure}[htbp]
\centering
\includegraphics{figure/trend\%20curve\%20of\%20kept\%20swarms\%20over\%20all\%2020\%20functions.-1.pdf}
\caption{plot of chunk trend curve of kept swarms over all 20
functions.}
\end{figure}

\section{Discussion}\label{discussion}

test

\section{Conclusion}\label{conclusion}

test

\section{Acknowledgements}\label{acknowledgements}

Thanks to everybody! \newpage

\hypertarget{refs}{}
\hypertarget{ref-epitropakisux5f2013}{}
Epitropakis, Michael G, Xiaodong Li, and Edmund K Burke. 2013. ``A
dynamic archive niching differential evolution algorithm for multimodal
optimization.'' In \emph{Evolutionary computation (cEC), 2013 iEEE
congress on}, IEEE, p. 79--86.

\hypertarget{ref-fieldsendux5f2014}{}
Fieldsend, Jonathan E. 2014. ``Running up those hills: Multi-modal
search with the niching migratory multi-swarm optimiser.'' In
\emph{Evolutionary computation (cEC), 2014 iEEE congress on}, IEEE, p.
2593--2600.

\hypertarget{ref-preussux5f2010}{}
Preuss, Mike. 2010. ``Niching the cMA-eS via nearest-better
clustering.'' In \emph{Proceedings of the 12th annual conference
companion on genetic and evolutionary computation}, GECCO '10, New York,
NY, USA: ACM, p. 1711--1718.
\url{http://doi.acm.org/10.1145/1830761.1830793}.


\end{document}
        % Add your content files here
    \end{content}

    % Appendix
     % \begin{appendix}
     %     \input{appendix}
     % \end{appendix}

    % References
    \references{library}

    % Declaration of authorship
    % \authorshipstatement[pagenumbering=false]
    \authorshipstatement[pagenumbering=true]
    % \authorshipstatement[pagenumbering=only]
    
    % Bonus: Wordcount
    % cd %FOLDER WHERE THE .tex FILES ARE IN %
    % clear
    % texcount -total -q -col -sum *.tex
    
\end{document}